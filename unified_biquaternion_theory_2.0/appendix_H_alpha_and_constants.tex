
\appendix
\section*{Appendix H: Fine-Structure Constant $\alpha$ and Fundamental Constants in the UBT Framework}
\addcontentsline{toc}{section}{Appendix H: Fine-Structure Constant $\alpha$ and Fundamental Constants in the UBT Framework}

\subsection*{H.1 Motivation and Scope}
This appendix consolidates the derivations and consistency relations for the fine-structure constant $\alpha$ and selected fundamental constants within the Unified Biquaternion Theory (UBT). 
We integrate (i) standard QED renormalization of $\alpha(Q^2)$, (ii) a geometric--modular ansatz based on Jacobi theta functions (see Appendix~E), and (iii) the complex-time and biquaternionic extensions introduced in Appendices~A--D.
The goal is to provide a consistent parameterization that reproduces standard physics when $\partial_\psi(\cdot)=0$, yet allows for controlled, testable corrections in the UBT regime.

\subsection*{H.2 Preliminaries: Definitions and Normalizations}
We adopt Heaviside--Lorentz units with $\hbar=c=1$ unless stated otherwise. 
The fine-structure constant is
\begin{equation}
\alpha \equiv \frac{e^2}{4\pi}, \qquad e>0, 
\end{equation}
and at a reference scale $\mu$ we write $\alpha(\mu)\equiv \alpha_\mu$.
In SI units, $\alpha = e^2/(4\pi\epsilon_0 \hbar c)$.
The QED covariant derivative and field strength are
\begin{equation}
D_\mu=\partial_\mu+i e A_\mu, \qquad F_{\mu\nu}=\partial_\mu A_\nu - \partial_\nu A_\mu.
\end{equation}

\subsection*{H.3 QED Running of $\alpha$ (Baseline)}
At one loop (fermions of charge $Q_f$ and mass $m_f$), the QED beta function yields
\begin{equation}
\mu \frac{\mathrm{d}\alpha}{\mathrm{d}\mu} \;=\; \frac{2}{3\pi}\,\alpha^2 \sum_f Q_f^2 \;+\; \mathcal{O}(\alpha^3).
\end{equation}
Equivalently,
\begin{equation}
\frac{1}{\alpha(Q)} \;=\; \frac{1}{\alpha(\mu)} \;-\; \frac{2}{3\pi}\sum_f Q_f^2 \ln\!\frac{Q}{\mu} \;+\; \mathcal{O}(\alpha).
\end{equation}
Threshold matching at $Q\sim m_f$ implements decoupling of heavy fermions.
Higher-loop corrections and hadronic vacuum polarization can be included if needed; these give the standard precision values in the $\partial_\psi(\cdot)=0$ limit.

\subsection*{H.4 Modular--Geometric Ansatz for $\alpha$}
Appendix~E established Jacobi theta functions $\theta_j(z,\tau)$ and their modular properties on complex tori. 
At $z=0$, define the theta \emph{constants}
\begin{equation}
\vartheta_2(\tau) \equiv \theta_2(0,\tau), \qquad
\vartheta_3(\tau) \equiv \theta_3(0,\tau), \qquad
\vartheta_4(\tau) \equiv \theta_4(0,\tau),
\end{equation}
and the elliptic modulus $\lambda(\tau)$ (modular lambda) by
\begin{equation}
\lambda(\tau) \;=\; k^2(\tau) \;=\; \frac{\vartheta_2^4(\tau)}{\vartheta_3^4(\tau)}, \qquad 1-\lambda(\tau)=\frac{\vartheta_4^4(\tau)}{\vartheta_3^4(\tau)}.
\end{equation}
The \emph{UBT ansatz} relates $\alpha$ to a modular invariant built from theta constants evaluated at an \emph{effective complex time parameter} $\tau_\psi$,
\begin{equation}
\tau_\psi \;\equiv\; \tau_0 + i\,\xi_\psi, \qquad \tau_0 \in \mathbb{H},\ \ \xi_\psi \ll 1,
\end{equation}
where $\xi_\psi$ encodes small, controlled $\psi$-sector effects. 
A minimal choice is
\begin{equation}
\boxed{\quad \alpha_{\mathrm{UBT}} \;=\; \alpha_0 \,\Bigg[ 1 \;+\; \varepsilon_\lambda\,
\frac{\lambda(\tau_\psi)-\lambda(\tau_0)}{\lambda(\tau_0)} \;+\; \varepsilon_\eta\,
\frac{\eta(\tau_\psi)}{\eta(\tau_0)} \;+\; \cdots \Bigg] \quad}
\label{eq:alpha_modular}
\end{equation}
with small coefficients $\varepsilon_{\lambda},\varepsilon_{\eta}$ and Dedekind eta $\eta(\tau)$, ensuring that in the limit $\xi_\psi \to 0$ one recovers $\alpha_{\mathrm{UBT}}\to \alpha_0$.
This form preserves modular covariance and keeps corrections perturbative and tunable.

\subsection*{H.5 Complex Time and Biquaternionic Couplings}
The UBT complex time is $\tau=t+i\psi$ (Appendix~A). 
In the presence of slow $\psi$-dynamics (Appendix~F), the EM sector inherits small modulations through the covariant derivatives extended along $\psi$ and via psychon couplings,
\begin{equation}
\mathcal{L} \supset \frac{1}{2}\kappa_\psi\,(\partial_\psi A_\mu)(\partial_\psi A^\mu) \;-\; g_{\chi A}\,\chi\,F_{\mu\nu}F^{\mu\nu} \;-\; g_{\chi\tilde A}\,\chi\,F_{\mu\nu}\tilde F^{\mu\nu} \;-\; g_{\chi \psi}\,\chi\,\bar{\Psi}\Psi.
\end{equation}
Integrating out slow $\psi$-fluctuations and $\chi$ at tree level leads to an EM effective action with small multiplicative renormalizations,
\begin{equation}
-\frac{1}{4} Z_F(\psi)\,F_{\mu\nu}F^{\mu\nu}, 
\qquad Z_F(\psi) \approx 1 + \delta Z_F(\psi),
\end{equation}
which is equivalent to a tiny shift of the effective charge $e \to e_{\mathrm{eff}}=e/\sqrt{Z_F}$ and hence of $\alpha$. 
Matching this with Eq.~\eqref{eq:alpha_modular} fixes the small parameters $(\varepsilon_\lambda,\varepsilon_\eta)$ in terms of $(\kappa_\psi,g_{\chi\cdot},\ldots)$ and the statistics of $\psi$-fluctuations.

\subsection*{H.6 Reconstruction Strategy from Precision Data}
To remain compatible with precision QED:
\begin{enumerate}
\item Fix $\alpha_0$ and $\tau_0$ by matching to a reference-scale value (e.g., low-energy $\alpha(0)$).
\item Impose that $\varepsilon_{\lambda},\varepsilon_{\eta}$ generate corrections below current sensitivity in all benchmark observables (Lamb shift, $a_e$, etc.).
\item Calibrate $(\varepsilon_{\lambda},\varepsilon_{\eta})$ in controlled resonator experiments (Appendix~E) via sideband asymmetries and phase shifts.
\end{enumerate}
In practice, this is a two-stage fit: (i) standard QED running to translate experimental scales to $\mu$, and (ii) a tiny modular correction constrained by laboratory protocols.

\subsection*{H.7 Fine-Structure Constant from a Geometric Ratio}
On compact $\psi$ (period $2\pi$), resonant solutions in Appendix~E suggest a ratio of periods/lengths as a natural source of a small dimensionless number. 
Let $L_t$ and $L_\psi$ denote effective periodicities in $(t,\psi)$ for a stationary eigenmode. Then, to leading order,
\begin{equation}
\alpha_{\mathrm{geom}} \;\sim\; \frac{L_\psi}{4\pi\,L_t}\,\Bigg[ 1 + \mathcal{O}\!\left(\frac{L_\psi^2}{L_t^2}\right)\Bigg], 
\qquad \frac{L_\psi}{L_t}\ \longleftrightarrow\ \text{function of } \lambda(\tau_\psi).
\end{equation}
Identifying $L_\psi/L_t$ with a modular function (e.g., $\lambda$ or a ratio of theta constants) yields a route from toroidal geometry to $\alpha$.
This geometric interpretation is consistent with the theta-based mode expansions and their modular invariants.

\subsection*{H.8 Other Constants: $G$, $\hbar$, $c$, and Mass Scales}
In UBT, $c$ and $\hbar$ are fixed by unit conventions; departures would signal Lorentz or quantum-kinematic anomalies (not present here).
The Newton constant $G$ arises from the biquaternionic gravity sector (Appendix~A/B) via an effective coupling in the geometric action; schematically,
\begin{equation}
S_{\mathrm{grav}} \;=\; \frac{1}{16\pi G_{\mathrm{eff}}}\!\int\! \mathrm{d}^4x\,\sqrt{-g}\, R \;+\; \cdots,\qquad 
G_{\mathrm{eff}} \;=\; G_0 \,[1+\delta_G(\psi)],
\end{equation}
where $\delta_G(\psi)$ is suppressed for slow $\psi$-dynamics.
Mass scales (e.g., $m_e$) receive tiny phase-sector corrections $\delta m \sim g_{\chi\psi}\langle \chi\rangle$, consistent with Appendix~F and subject to strong experimental bounds.

\subsection*{H.9 Testable Consequences and Bounds}
\begin{itemize}
\item \textbf{Laboratory resonators (Appendix E):} minute phase-dependent shifts in effective $\alpha$ inferred from dispersion and sidebands, parameterized by $(\varepsilon_\lambda,\varepsilon_\eta)$.
\item \textbf{Spectroscopy:} no conflict with precision QED provided $\varepsilon_{\lambda,\eta}\!\to\!0$ in standard conditions; dedicated protocols set upper bounds.
\item \textbf{Astrophysics:} environmental $\psi$-fields could in principle bias $\tau_\psi$; constraints follow from null variation of constants in diverse environments.
\end{itemize}

\subsection*{H.10 Summary}
We presented a modular--geometric ansatz consistent with QED running and UBT's complex-time/biquaternion structure. 
The resulting formula for $\alpha$,
\begin{equation}
\alpha_{\mathrm{UBT}} \;=\; \alpha_0\,\big[ 1 + \delta\alpha_{\mathrm{mod}}(\tau_\psi;\varepsilon_\lambda,\varepsilon_\eta) \big],
\end{equation}
reduces to standard physics in the $\partial_\psi(\cdot)=0$ limit and predicts only tiny, structured deviations under controlled phase protocols.
Informally, one might say that \emph{theta} acts as a guiding principle: the modular anatomy of $\alpha$ mirrors the toroidal, phase-coupled fabric of UBT.
