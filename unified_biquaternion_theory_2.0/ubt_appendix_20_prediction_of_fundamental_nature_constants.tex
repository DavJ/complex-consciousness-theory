\documentclass[12pt, a4paper]{article}
\usepackage[utf8]{inputenc}
\usepackage[english]{babel}
\usepackage{amsmath, amssymb}
\usepackage{geometry}
\usepackage{siunitx} % For proper unit formatting

\geometry{a4paper, margin=1in}

\begin{document}

\appendix
\section*{Appendix 20: Semiclassical Calculation of the Electron's Self-Energy}
\addcontentsline{toc}{section}{Appendix 20: Semiclassical Calculation of the Electron's Self-Energy}

\subsection*{20.1 Overview}
This appendix details the semiclassical derivation of the electron's electromagnetic self-energy (\(\delta m_e\)). This approach provides an intuitive physical picture and a first-order estimate for the correction required by our dual-mass hypothesis. The result demonstrates that the effective radius \(R\) of the charge distribution is not a free parameter but is determined by the underlying topological solution \(\Theta_1\).

\subsection*{20.2 Self-Energy of a Smeared Charge Distribution}

We start from the classical expression for the electrostatic self-energy, which represents the energy stored in the electric field generated by the charge distribution:
\begin{equation}
    \delta m_e c^2 = \frac{1}{2} \int \rho(\vec{x}) \phi(\vec{x})\, d^3x
\end{equation}
To make the problem analytically tractable, we assume the charge distribution \( \rho(r) \) corresponding to the \(\Theta_1\) Hopfion can be approximated by a spherically symmetric Gaussian function:
\begin{equation}
    \rho(r) = \frac{e}{\pi^{3/2} R^3} \exp\left(-\frac{r^2}{R^2}\right)
\end{equation}
where \(R\) is the effective radius of the distribution.

\subsection*{20.3 Electrostatic Potential and Final Integration}
The electrostatic potential \( \phi(r) \) for this Gaussian source is found by solving the Poisson equation \( \nabla^2 \phi = -\rho/\epsilon_0 \). The solution is:
\begin{equation}
    \phi(r) = \frac{e}{4\pi\epsilon_0 r} \operatorname{erf}\left( \frac{r}{R} \right)
\end{equation}
where \( \operatorname{erf} \) is the error function. Substituting \( \rho(r) \) and \( \phi(r) \) back into the self-energy integral yields the result:
\begin{equation}
    \delta m_e = \frac{e^2}{2\sqrt{\pi} (4\pi\epsilon_0) R c^2} = \frac{\alpha \hbar}{\sqrt{\pi} R c}
\end{equation}
In natural units (\(\hbar=c=1\), \( e^2 = 4\pi\alpha \)), this simplifies to \( \delta m_e = \frac{\sqrt{4\pi}\alpha}{R} \).
\textit{(Poznámka: Ve vašem souboru byl vzorec bez \(4\pi\epsilon_0\), což odpovídá jinému systému jednotek. Zde jsem použil standardní SI-odvozenou formu, která je explicitnější.)}

\subsection*{20.4 The Nature of the Radius R}
Crucially, the parameter \(R\) is not a free constant. As shown in the document `electron_mass_from_self_energy.tex`, it is uniquely determined by the topological field solution \(\Theta_1\). For instance, by calculating the root-mean-square radius of the field's energy density distribution:
\begin{equation}
    R^2 = \frac{\int r^2 |\nabla \Theta_1|^2 \, d^3x}{\int |\nabla \Theta_1|^2 \, d^3x}
\end{equation}
This calculation links the effective size \(R\) directly to the fundamental structure of the theory, thus completing the prediction of the electron mass from first principles.

\end{document}
