
\appendix
\section*{Appendix P: Prime-Based Universes, Fine-Structure Constant, and $p$-Adic Extension}

In Appendix F we introduced the $p$-adic formulation of the Unified Biquaternion Theory (UBT), in which the complex-time manifold $\mathbb{C}^5$ admits an alternative valuation defined over the field $\mathbb{Q}_p$ for a prime $p$. Here, we explore a specific physical implication of this structure: the possibility that the fine-structure constant $\alpha$ is determined by a prime number parameter governing the topological winding of the fundamental field $\Theta(q,\tau)$.

\subsection*{Prime constraint and physical motivation}
The topological derivation of $\alpha$ in UBT, given in Appendix A and further refined in Appendix G, relies on a quantization condition involving the number of windings $N$ of a fundamental phase around a compactified cycle of the toroidal $\tau$-manifold. Physical consistency (gauge invariance, anomaly cancellation) imposes that $N$ must be an integer, and stability analysis suggests that $N$ must be prime to avoid factorization into unstable sub-cycles. The prime $N$ then appears in the leading-order expression for $\alpha^{-1}$.

In our universe, the optimal value found is $N = 137$, yielding $\alpha \approx 1/137.035999$. However, other prime values may correspond to alternative universes with similar but distinct physical constants. For example:
\begin{itemize}
\item $N = 131$ could yield a slightly stronger electromagnetic interaction, shifting atomic spectra and chemistry.
\item $N = 139$ would weaken the interaction slightly, possibly still allowing stable matter but with altered nuclear binding energies.
\end{itemize}

\subsection*{Generalized formula for $\alpha$}
The generalized leading-order formula from the topological derivation is:
\begin{equation}
\alpha^{-1}(N) \approx N + \delta_{\mathrm{QED}}(N) + \delta_{\mathrm{grav}}(N),
\end{equation}
where $\delta_{\mathrm{QED}}$ and $\delta_{\mathrm{grav}}$ are radiative and gravitational corrections derived in Appendices H and I.

The $p$-adic interpretation replaces $N$ by a $p$-adic norm $|N|_p^{-1}$, where $p$ is the same prime as $N$ for the self-consistent universe, leading to a quantization of $\alpha^{-1}$ in discrete prime steps. This yields:
\begin{equation}
\alpha^{-1}(p) \approx p + \delta(p), \quad p \ \text{prime}.
\end{equation}

\subsection*{Predictions for alternative prime-based universes}
Using the same correction model as for $p=137$, we find:
\begin{align*}
\alpha^{-1}(131) &\approx 131.031, & \alpha(131) &\approx 0.0076330, \\
\alpha^{-1}(137) &\approx 137.036, & \alpha(137) &\approx 0.00729735, \\
\alpha^{-1}(139) &\approx 139.039, & \alpha(139) &\approx 0.00719421.
\end{align*}

\subsection*{Physical implications}
Small changes in $\alpha$ modify atomic and molecular energy levels, fusion rates, and even stellar lifetimes. The $p$-adic formulation suggests that the prime number acts as a discrete "dial" for universes, with stability of complex matter only possible for certain primes. This framework naturally integrates with the drift-diffusion dynamics in complex time, as described in Appendix F.

\subsection*{Conclusion}
The $p$-adic extension of UBT not only provides a mathematically consistent embedding for discrete prime parameters, but also predicts that the fine-structure constant is quantized in prime-based steps. Our universe, with $p=137$, is optimally balanced for stability and complexity, but nearby prime universes ($p=131$, $p=139$) might still host recognizable physics, albeit with altered constants.
