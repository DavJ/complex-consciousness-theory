
\appendix{G: Dark Matter as Topological Hopfions in Unified Biquaternion Theory}

\section*{Introduction}
In the Unified Biquaternion Theory (UBT), dark matter is modeled as topologically stable field configurations – \emph{hopfions} – arising as non-trivial mappings between compactified spatial slices \(S^3\) and internal field space \(S^2\). These configurations are solutions to the UBT field equations for the tensor–spinor field \(\Theta(q,\tau)\) with non-zero Hopf invariant \(Q_H \in \mathbb{Z}\).

Unlike conventional particle dark matter candidates (e.g., WIMPs, axions), hopfions do not require direct coupling to the Standard Model fields; their stability is protected by topological conservation laws.

\section*{Mathematical Formulation}
Let the UBT field \(\Theta: S^3 \to S^2\) be expressed in normalized complex components \(\Theta = (\phi_1, \phi_2) \in \mathbb{C}^2\), satisfying \(|\phi_1|^2 + |\phi_2|^2 = 1\). The Hopf invariant is defined as:

\begin{equation}
Q_H = \frac{1}{(4\pi)^2} \int_{S^3} \mathbf{A} \wedge \mathbf{F},
\end{equation}
where \(\mathbf{F} = d\mathbf{A}\) is the field strength associated with the pullback of the area form on \(S^2\).

The energy functional in the biquaternionic Lagrangian takes the form:
\begin{equation}
E[\Theta] = \int_{\mathbb{R}^3} \left( \alpha |\nabla \Theta|^2 + \beta |\mathbf{F}|^2 \right) \, d^3x,
\end{equation}
with \(\alpha, \beta\) determined by the underlying UBT coupling constants.

\section*{Hopfion Solutions and Stability}
Minimization of \(E[\Theta]\) under fixed \(Q_H\) yields knotted soliton solutions. In the UBT framework, these solutions are generalized to include the complex-time coordinate \(\tau = t + i\psi\), giving rise to \emph{phase–linked hopfions} where the imaginary time \(\psi\) modulates their internal twist.

Stability is guaranteed by the integer-valued Hopf invariant; any continuous deformation preserving field smoothness cannot change \(Q_H\).

\section*{Dark Matter Interpretation}
Hopfions interact only gravitationally and through topology–mediated phase effects with normal matter. Their contribution to the energy–momentum tensor in the UBT metric is given by:
\begin{equation}
T_{\mu\nu}^{\text{hopfion}} = 2\alpha \partial_\mu \Theta \cdot \partial_\nu \Theta^* + 2\beta F_{\mu\lambda} F_{\nu}{}^\lambda - g_{\mu\nu} \mathcal{L}_{\text{hopfion}}.
\end{equation}

Cosmologically, a relic density of hopfions could arise naturally during symmetry-breaking phase transitions in the early Universe, analogously to cosmic strings or monopoles, but without leading to observational inconsistencies thanks to their low interaction cross-section.

\section*{Observational Signatures}
Possible observational consequences include:
\begin{itemize}
\item Microlensing events without electromagnetic counterparts.
\item Non-Gaussian features in the CMB due to large-scale hopfion structures.
\item Anomalous gravitational wave dispersion from hopfion–induced metric perturbations.
\end{itemize}

\section*{Conclusion}
The UBT hopfion model provides a purely topological candidate for dark matter, naturally arising from the field \(\Theta(q,\tau)\) without introducing ad hoc particles. This aligns with the theory’s aim to unify physical phenomena through a single geometric–algebraic framework.
