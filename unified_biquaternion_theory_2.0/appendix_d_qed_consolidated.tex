
\appendix
\section*{Appendix D: Quantum Electrodynamics (QED)}

\subsection*{Introduction}
Quantum Electrodynamics (QED) is the relativistic quantum field theory of the electromagnetic interaction. 
It describes the dynamics of charged fermions (such as electrons and positrons) and photons within a fully 
Lorentz-invariant and gauge-invariant framework. In the context of the Unified Biquaternion Theory (UBT), 
QED is embedded as the $U(1)$ sector of the full gauge symmetry, with possible extensions involving 
complex time $\tau = t + i\psi$ and the $\Theta(q,\tau)$ field.

\subsection*{Mathematical Foundations}
The classical Lagrangian density of QED is:
\begin{equation}
\mathcal{L}_{\text{QED}} = \bar{\psi}(i\gamma^\mu D_\mu - m)\psi - \frac{1}{4}F_{\mu\nu}F^{\mu\nu},
\end{equation}
where:
\begin{equation}
D_\mu = \partial_\mu + ieA_\mu
\end{equation}
is the covariant derivative, $A_\mu$ is the electromagnetic potential, and 
$F_{\mu\nu} = \partial_\mu A_\nu - \partial_\nu A_\mu$ is the electromagnetic field tensor.

The Dirac equation in the presence of an electromagnetic field is obtained from the Euler–Lagrange equations:
\begin{equation}
(i\gamma^\mu D_\mu - m)\psi = 0.
\end{equation}

\subsection*{Gauge Symmetry and $U(1)$ Invariance}
The QED Lagrangian is invariant under local $U(1)$ gauge transformations:
\begin{equation}
\psi \rightarrow e^{i e \alpha(x)}\psi, \quad A_\mu \rightarrow A_\mu - \partial_\mu \alpha(x).
\end{equation}
This symmetry enforces charge conservation and determines the form of the fermion–photon coupling.

\subsection*{Interaction and Feynman Rules}
In perturbative QED, the interaction term:
\begin{equation}
\mathcal{L}_{\text{int}} = -e \bar{\psi} \gamma^\mu \psi A_\mu
\end{equation}
generates all fermion–photon processes, including scattering and pair production. The 
Feynman rules for QED are derived from this interaction term and are consistent with both Lorentz and gauge invariance.

\subsection*{QED in the Unified Biquaternion Theory}
Within UBT, the electromagnetic four-potential $A_\mu$ is embedded in the biquaternionic $\Theta(q,\tau)$ field, 
and the $U(1)$ symmetry emerges naturally from the phase transformations in complex time. This formalism allows 
a direct link between quantum phase evolution (in $\psi$) and electromagnetic gauge transformations.

\subsection*{Summary and Outlook}
QED remains one of the most accurate and well-tested theories in physics. In the UBT framework, 
its formalism is preserved, but the embedding in complex time and biquaternionic geometry opens the 
possibility of additional couplings and modifications, particularly in regimes where $\psi$-phase dynamics become relevant.

Further details on the fine-structure constant $\alpha$ and its potential dynamical behavior in complex time 
are discussed in Appendix E.
