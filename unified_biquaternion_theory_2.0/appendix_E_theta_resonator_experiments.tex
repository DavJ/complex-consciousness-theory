
% Appendix E: Experimental Protocols and Calibration of the Theta Resonator
% This file is meant to be included into your main LaTeX project or compiled standalone with an article class.
% Requires: tikz, tikz-3dplot, circuitikz, amsmath, amssymb, physics, siunitx, caption
%
% Example preamble (if compiling standalone):
% \documentclass[11pt,a4paper]{article}
% \usepackage{amsmath,amssymb}
% \usepackage{physics}
% \usepackage{siunitx}
% \usepackage{tikz}
% \usepackage{tikz-3dplot}
% \usepackage[europeanresistors,americaninductors]{circuitikz}
% \usepackage{caption}
% \begin{document}
% 
% Appendix E: Experimental Protocols and Calibration of the Theta Resonator
% This file is meant to be included into your main LaTeX project or compiled standalone with an article class.
% Requires: tikz, tikz-3dplot, circuitikz, amsmath, amssymb, physics, siunitx, caption
%
% Example preamble (if compiling standalone):
% \documentclass[11pt,a4paper]{article}
% \usepackage{amsmath,amssymb}
% \usepackage{physics}
% \usepackage{siunitx}
% \usepackage{tikz}
% \usepackage{tikz-3dplot}
% \usepackage[europeanresistors,americaninductors]{circuitikz}
% \usepackage{caption}
% \begin{document}
% 
% Appendix E: Experimental Protocols and Calibration of the Theta Resonator
% This file is meant to be included into your main LaTeX project or compiled standalone with an article class.
% Requires: tikz, tikz-3dplot, circuitikz, amsmath, amssymb, physics, siunitx, caption
%
% Example preamble (if compiling standalone):
% \documentclass[11pt,a4paper]{article}
% \usepackage{amsmath,amssymb}
% \usepackage{physics}
% \usepackage{siunitx}
% \usepackage{tikz}
% \usepackage{tikz-3dplot}
% \usepackage[europeanresistors,americaninductors]{circuitikz}
% \usepackage{caption}
% \begin{document}
% 
% Appendix E: Experimental Protocols and Calibration of the Theta Resonator
% This file is meant to be included into your main LaTeX project or compiled standalone with an article class.
% Requires: tikz, tikz-3dplot, circuitikz, amsmath, amssymb, physics, siunitx, caption
%
% Example preamble (if compiling standalone):
% \documentclass[11pt,a4paper]{article}
% \usepackage{amsmath,amssymb}
% \usepackage{physics}
% \usepackage{siunitx}
% \usepackage{tikz}
% \usepackage{tikz-3dplot}
% \usepackage[europeanresistors,americaninductors]{circuitikz}
% \usepackage{caption}
% \begin{document}
% \input{appendix_E_theta_resonator_experiments.tex}
% \end{document}

\section*{Appendix E: Experimental Protocols and Calibration of the \emph{Theta Resonator}}
\addcontentsline{toc}{section}{Appendix E: Experimental Protocols and Calibration of the Theta Resonator}

\subsection*{E.1 Purpose and Scope}
This appendix specifies the experimental blueprint, calibration procedures, and data-analysis methods for the \emph{Theta Resonator} --- a toroidal EM device designed to probe phase-coupled dynamics predicted by the Unified Biquaternion Theory (UBT) with complex time $\tau = t + i\psi$.
The goal is to detect small, reproducible signatures of $\psi$-dependent phase modulation in electromagnetic observables while remaining fully compatible with standard QED in the limit $\partial_\psi(\cdot)=0$.

\subsection*{E.2 Mathematical Model (Concise)}
Let $A_\mu$ be the EM potential and $\Psi$ a Dirac spinor. The UBT-extended QED Lagrangian density (restricted to the EM sector plus a scalar psychon field $\chi$) is
\begin{align}
\mathcal{L} &= \bar{\Psi}(i\gamma^\mu D_\mu - m)\Psi - \frac{1}{4}F_{\mu\nu}F^{\mu\nu}
+ \frac{1}{2}\eta_\psi\,(\partial_\psi \Psi)^\dagger(\partial_\psi \Psi)
+ \frac{1}{2}\kappa_\psi\,(\partial_\psi A_\mu)(\partial_\psi A^\mu) \nonumber\\
&\quad + \frac{1}{2}(\partial_\mu \chi)(\partial^\mu \chi) + \frac{1}{2}(\partial_\psi \chi)^2 - U(\chi)
- g_{\chi A}\,\chi\,F_{\mu\nu}F^{\mu\nu} - g_{\chi\tilde{A}}\,\chi\,F_{\mu\nu}\tilde{F}^{\mu\nu}
- g_{\chi \psi}\,\chi\,\bar{\Psi}\Psi .
\end{align}
Setting $\partial_\psi(\cdot)=0$ recovers standard QED predictions. The \emph{Theta Resonator} targets regimes where $\partial_\psi$-terms are small but non-zero, producing phase-sidebands and minute dispersion shifts.

\subsection*{E.3 Experimental Setup (3D Schematic)}
Figure~\ref{fig:torus3d} provides a 3D schematic of the toroidal resonator, including the primary winding, a phase modulator section, pickup probes, and environmental shielding.
The drawing is not to scale but preserves the essential geometry for replication.

\begin{figure}[h!]
\centering
\tdplotsetmaincoords{70}{110}% elevation, azimuth
\begin{tikzpicture}[tdplot_main_coords,scale=1.1]
  % Torus parameters (visual only)
  \def\R{3.0}    % major radius
  \def\r{0.9}    % minor radius
  \def\Nw{18}    % winding segments (visual)
  % Draw a coarse torus surface (meridians)
  \foreach \t in {0,20,...,340}{
    \begin{scope}[tdplot_rotated_coords]
      \tdplotsetrotatedcoords{\t}{0}{0}
      \draw[blue!40!black!60, line width=0.3pt, opacity=0.6]
        plot[domain=0:360,samples=50,variable=\u]
        ({(\R+\r*cos(\u))*cos(0)},
         {(\R+\r*cos(\u))*sin(0)},
         {\r*sin(\u)});
    \end{scope}
  }
  % Draw a few parallels
  \foreach \u in {0,45,...,315}{
    \draw[blue!30!black!50, line width=0.3pt, opacity=0.6]
      plot[domain=0:360,samples=80,variable=\t]
      ({(\R+\r*cos(\u))*cos(\t)},
       {(\R+\r*cos(\u))*sin(\t)},
       {\r*sin(\u)});
  }
  % Primary winding (as discrete loops)
  \foreach \k in {0,...,\Nw}{
    \pgfmathsetmacro{\ang}{\k*360/\Nw}
    \draw[red!70!black, line width=1.0pt, opacity=0.9]
      plot[domain=0:360,samples=60,variable=\u]
      ({(\R+0.55*\r*cos(\u))*cos(\ang)},
       {(\R+0.55*\r*cos(\u))*sin(\ang)},
       {0.55*\r*sin(\u)});
  }
  % Phase modulator block
  \draw[fill=green!30, draw=green!50!black, opacity=0.8]
       (\R+1.1*\r, -0.7, 0.4) -- (\R+1.8*\r, -0.7, 0.4) -- (\R+1.8*\r, 0.7, 0.4) -- (\R+1.1*\r, 0.7, 0.4) -- cycle;
  \node[anchor=west] at (\R+1.85*\r,0.0,0.6) {\small Phase Modulator};

  % Pickup probes
  \fill[orange!70!black] (\R-0.2, 0.0, 1.1) circle (2pt) node[anchor=south]{\small Probe 1};
  \fill[orange!70!black] (\R-0.2, 0.0,-1.1) circle (2pt) node[anchor=north]{\small Probe 2};

  % Shielding (wireframe ring)
  \draw[gray!60, dashed]
    plot[domain=0:360,samples=80,variable=\t]
    ({(\R+1.7*\r)*cos(\t)}, {(\R+1.7*\r)*sin(\t)}, {0.0});
  \node at (0,-\R-1.2,0) {\small Toroidal Resonator (3D schematic)};
\end{tikzpicture}
\caption{3D schematic of the \emph{Theta Resonator} using \texttt{tikz-3dplot}. Major/minor radii and winding are indicative.}
\label{fig:torus3d}
\end{figure}

\subsection*{E.4 Electrical Interconnection (circuitikz)}
The block diagram in Fig.~\ref{fig:circuit} shows the generator, phase modulator, toroidal resonator, low-noise amplifier (LNA), lock-in detection, and DAQ/PC control.

\begin{figure}[h!]
\centering
\begin{circuitikz}[american voltages]
  % Generator
  \draw (0,0) to[sV,l=$\mathrm{RF\ Generator}$] (0,3) to[short] (2,3);
  % Modulator block
  \draw (2,4) node[draw,fill=green!20,minimum width=2.0cm,minimum height=1.2cm]{\footnotesize Phase\\Modulator};
  \draw (2,3) -- (2,4);
  \draw (4,4) node[draw,fill=blue!10,minimum width=2.0cm,minimum height=1.2cm]{\footnotesize Toroidal\\Resonator};
  \draw (3,4) -- (3.0,4);
  \draw (2.95,4) -- (3.95,4);
  \draw (5,4) -- (6,4);
  % LNA and Lock-in
  \draw (6,4) to[american current source,l=$\mathrm{Probe}$] (6,1);
  \draw (6,1) node[draw,fill=orange!15,minimum width=2.2cm,minimum height=1.2cm]{\footnotesize LNA};
  \draw (8,1) node[draw,fill=yellow!20,minimum width=3.0cm,minimum height=1.2cm]{\footnotesize Lock-in\ Detector};
  \draw (7.1,1) -- (6.9,1); \draw (8.9,1) -- (9.1,1);
  \draw (10.5,1) -- (12,1);
  % DAQ/PC
  \draw (12,2.0) node[draw,fill=gray!15,minimum width=2.8cm,minimum height=1.8cm]{\footnotesize DAQ / PC};
  \draw (12,1) -- (12,2);
\end{circuitikz}
\caption{Block-level interconnection of the experiment: RF generator $\to$ phase modulator $\to$ toroidal resonator; pickup $\to$ LNA $\to$ lock-in $\to$ DAQ/PC.}
\label{fig:circuit}
\end{figure}

\subsection*{E.5 Measurement Protocol (Step-by-Step)}
\paragraph{Preparation.} 
Thermally stabilize the setup ($\pm\SI{0.1}{K}$), EM-shield the chamber ($>\SI{80}{dB}$ at carrier), and verify grounding topology.
Calibrate the frequency source and lock-in time constants.

\paragraph{Baseline.}
With the torus open-circuit and no phase modulation, acquire baseline spectra $S_0(f)$, noise density $N_0(f)$, and phase reference $\phi_0(f)$.

\paragraph{Resonance Identification.}
Sweep frequency to locate toroidal eigenmodes $f_n$.
Record $Q$-factors and mode shapes via probe positioning.

\paragraph{Phase Modulation.}
Apply controlled phase drive $\varphi(t)=\varphi_0 \sin(2\pi f_m t)$.
Measure sidebands at $f_n \pm k f_m$ and their dependence on $\varphi_0$ and $f_m$.

\paragraph{UBT-sensitive Runs.}
Enable $\psi$-protocols (attention tasks, meditative steady-state, randomized controls). 
Record synchronized EEG/ECG/EOG if applicable (ethically approved).
Mark epochs for correlation analysis with EM observables.

\subsection*{E.6 Calibration Procedures}
\paragraph{Amplitude and Phase.}
Insert a reference loop with known coupling $k$.
Sweep a calibrated amplitude and retrieve transfer function $H(f)$ and phase $\phi(f)$.

\paragraph{Linearity.}
Vary input by $\pm\SI{20}{dB}$ and fit response to detect compression or intermodulation.

\paragraph{Environmental Subtraction.}
Measure with a dummy load (non-toroidal) to subtract environmental EM signatures.
Use \emph{dual-probe differential} to suppress common-mode pickup.

\subsection*{E.7 Data Analysis}
\paragraph{Filtering and FFT.} 
Apply zero-phase bandpass filtering around $f_n$ and $f_n\pm k f_m$.
Compute PSD/ASD and coherence; estimate sideband amplitudes and phases.

\paragraph{Statistical Tests.} 
Use permutation tests and false-discovery-rate (FDR) control across epochs.
Correlate EM features with task markers; report effect sizes and confidence intervals.

\paragraph{Theta Signature Search.} 
Search for modular/periodic patterns consistent with theta-basis expansions.
Document invariance (or lack thereof) under reparameterizations of $\psi$.

\subsection*{E.8 Safety and Reproducibility}
\paragraph{Safety.}
Respect RF exposure limits, thermal constraints, and electrical isolation.
EEG (if used) must meet medical safety standards and informed consent protocols.

\paragraph{Reproducibility.}
Publicly share CAD/PCB files, BOM, firmware, raw data, and analysis scripts.
Log all calibration constants and environmental conditions.

\subsection*{E.9 Predicted Observables (UBT)}
\begin{itemize}
\item Sidebands at $f_n \pm k f_m$ with amplitudes deviating from standard EM-only predictions by $\mathcal{O}(10^{-4}\text{--}10^{-6})$ in controlled $\psi$-protocols.
\item Tiny phase drifts in $\phi(f_n)$ correlated with $\psi$-epoch markers.
\item Robustness of patterns under modular transforms of the analysis window.
\end{itemize}

\subsection*{E.10 Closing Remark}
All protocols herein preserve compatibility with QED in the $\partial_\psi\!=\!0$ limit; any deviation must pass stringent controls.
Informally, one might say that \emph{``theta is the password of the universe''} --- or, more conservatively, its guiding mathematical principle reflected in the observed modular structure.

% End of Appendix E

% \end{document}

\section*{Appendix E: Experimental Protocols and Calibration of the \emph{Theta Resonator}}
\addcontentsline{toc}{section}{Appendix E: Experimental Protocols and Calibration of the Theta Resonator}

\subsection*{E.1 Purpose and Scope}
This appendix specifies the experimental blueprint, calibration procedures, and data-analysis methods for the \emph{Theta Resonator} --- a toroidal EM device designed to probe phase-coupled dynamics predicted by the Unified Biquaternion Theory (UBT) with complex time $\tau = t + i\psi$.
The goal is to detect small, reproducible signatures of $\psi$-dependent phase modulation in electromagnetic observables while remaining fully compatible with standard QED in the limit $\partial_\psi(\cdot)=0$.

\subsection*{E.2 Mathematical Model (Concise)}
Let $A_\mu$ be the EM potential and $\Psi$ a Dirac spinor. The UBT-extended QED Lagrangian density (restricted to the EM sector plus a scalar psychon field $\chi$) is
\begin{align}
\mathcal{L} &= \bar{\Psi}(i\gamma^\mu D_\mu - m)\Psi - \frac{1}{4}F_{\mu\nu}F^{\mu\nu}
+ \frac{1}{2}\eta_\psi\,(\partial_\psi \Psi)^\dagger(\partial_\psi \Psi)
+ \frac{1}{2}\kappa_\psi\,(\partial_\psi A_\mu)(\partial_\psi A^\mu) \nonumber\\
&\quad + \frac{1}{2}(\partial_\mu \chi)(\partial^\mu \chi) + \frac{1}{2}(\partial_\psi \chi)^2 - U(\chi)
- g_{\chi A}\,\chi\,F_{\mu\nu}F^{\mu\nu} - g_{\chi\tilde{A}}\,\chi\,F_{\mu\nu}\tilde{F}^{\mu\nu}
- g_{\chi \psi}\,\chi\,\bar{\Psi}\Psi .
\end{align}
Setting $\partial_\psi(\cdot)=0$ recovers standard QED predictions. The \emph{Theta Resonator} targets regimes where $\partial_\psi$-terms are small but non-zero, producing phase-sidebands and minute dispersion shifts.

\subsection*{E.3 Experimental Setup (3D Schematic)}
Figure~\ref{fig:torus3d} provides a 3D schematic of the toroidal resonator, including the primary winding, a phase modulator section, pickup probes, and environmental shielding.
The drawing is not to scale but preserves the essential geometry for replication.

\begin{figure}[h!]
\centering
\tdplotsetmaincoords{70}{110}% elevation, azimuth
\begin{tikzpicture}[tdplot_main_coords,scale=1.1]
  % Torus parameters (visual only)
  \def\R{3.0}    % major radius
  \def\r{0.9}    % minor radius
  \def\Nw{18}    % winding segments (visual)
  % Draw a coarse torus surface (meridians)
  \foreach \t in {0,20,...,340}{
    \begin{scope}[tdplot_rotated_coords]
      \tdplotsetrotatedcoords{\t}{0}{0}
      \draw[blue!40!black!60, line width=0.3pt, opacity=0.6]
        plot[domain=0:360,samples=50,variable=\u]
        ({(\R+\r*cos(\u))*cos(0)},
         {(\R+\r*cos(\u))*sin(0)},
         {\r*sin(\u)});
    \end{scope}
  }
  % Draw a few parallels
  \foreach \u in {0,45,...,315}{
    \draw[blue!30!black!50, line width=0.3pt, opacity=0.6]
      plot[domain=0:360,samples=80,variable=\t]
      ({(\R+\r*cos(\u))*cos(\t)},
       {(\R+\r*cos(\u))*sin(\t)},
       {\r*sin(\u)});
  }
  % Primary winding (as discrete loops)
  \foreach \k in {0,...,\Nw}{
    \pgfmathsetmacro{\ang}{\k*360/\Nw}
    \draw[red!70!black, line width=1.0pt, opacity=0.9]
      plot[domain=0:360,samples=60,variable=\u]
      ({(\R+0.55*\r*cos(\u))*cos(\ang)},
       {(\R+0.55*\r*cos(\u))*sin(\ang)},
       {0.55*\r*sin(\u)});
  }
  % Phase modulator block
  \draw[fill=green!30, draw=green!50!black, opacity=0.8]
       (\R+1.1*\r, -0.7, 0.4) -- (\R+1.8*\r, -0.7, 0.4) -- (\R+1.8*\r, 0.7, 0.4) -- (\R+1.1*\r, 0.7, 0.4) -- cycle;
  \node[anchor=west] at (\R+1.85*\r,0.0,0.6) {\small Phase Modulator};

  % Pickup probes
  \fill[orange!70!black] (\R-0.2, 0.0, 1.1) circle (2pt) node[anchor=south]{\small Probe 1};
  \fill[orange!70!black] (\R-0.2, 0.0,-1.1) circle (2pt) node[anchor=north]{\small Probe 2};

  % Shielding (wireframe ring)
  \draw[gray!60, dashed]
    plot[domain=0:360,samples=80,variable=\t]
    ({(\R+1.7*\r)*cos(\t)}, {(\R+1.7*\r)*sin(\t)}, {0.0});
  \node at (0,-\R-1.2,0) {\small Toroidal Resonator (3D schematic)};
\end{tikzpicture}
\caption{3D schematic of the \emph{Theta Resonator} using \texttt{tikz-3dplot}. Major/minor radii and winding are indicative.}
\label{fig:torus3d}
\end{figure}

\subsection*{E.4 Electrical Interconnection (circuitikz)}
The block diagram in Fig.~\ref{fig:circuit} shows the generator, phase modulator, toroidal resonator, low-noise amplifier (LNA), lock-in detection, and DAQ/PC control.

\begin{figure}[h!]
\centering
\begin{circuitikz}[american voltages]
  % Generator
  \draw (0,0) to[sV,l=$\mathrm{RF\ Generator}$] (0,3) to[short] (2,3);
  % Modulator block
  \draw (2,4) node[draw,fill=green!20,minimum width=2.0cm,minimum height=1.2cm]{\footnotesize Phase\\Modulator};
  \draw (2,3) -- (2,4);
  \draw (4,4) node[draw,fill=blue!10,minimum width=2.0cm,minimum height=1.2cm]{\footnotesize Toroidal\\Resonator};
  \draw (3,4) -- (3.0,4);
  \draw (2.95,4) -- (3.95,4);
  \draw (5,4) -- (6,4);
  % LNA and Lock-in
  \draw (6,4) to[american current source,l=$\mathrm{Probe}$] (6,1);
  \draw (6,1) node[draw,fill=orange!15,minimum width=2.2cm,minimum height=1.2cm]{\footnotesize LNA};
  \draw (8,1) node[draw,fill=yellow!20,minimum width=3.0cm,minimum height=1.2cm]{\footnotesize Lock-in\ Detector};
  \draw (7.1,1) -- (6.9,1); \draw (8.9,1) -- (9.1,1);
  \draw (10.5,1) -- (12,1);
  % DAQ/PC
  \draw (12,2.0) node[draw,fill=gray!15,minimum width=2.8cm,minimum height=1.8cm]{\footnotesize DAQ / PC};
  \draw (12,1) -- (12,2);
\end{circuitikz}
\caption{Block-level interconnection of the experiment: RF generator $\to$ phase modulator $\to$ toroidal resonator; pickup $\to$ LNA $\to$ lock-in $\to$ DAQ/PC.}
\label{fig:circuit}
\end{figure}

\subsection*{E.5 Measurement Protocol (Step-by-Step)}
\paragraph{Preparation.} 
Thermally stabilize the setup ($\pm\SI{0.1}{K}$), EM-shield the chamber ($>\SI{80}{dB}$ at carrier), and verify grounding topology.
Calibrate the frequency source and lock-in time constants.

\paragraph{Baseline.}
With the torus open-circuit and no phase modulation, acquire baseline spectra $S_0(f)$, noise density $N_0(f)$, and phase reference $\phi_0(f)$.

\paragraph{Resonance Identification.}
Sweep frequency to locate toroidal eigenmodes $f_n$.
Record $Q$-factors and mode shapes via probe positioning.

\paragraph{Phase Modulation.}
Apply controlled phase drive $\varphi(t)=\varphi_0 \sin(2\pi f_m t)$.
Measure sidebands at $f_n \pm k f_m$ and their dependence on $\varphi_0$ and $f_m$.

\paragraph{UBT-sensitive Runs.}
Enable $\psi$-protocols (attention tasks, meditative steady-state, randomized controls). 
Record synchronized EEG/ECG/EOG if applicable (ethically approved).
Mark epochs for correlation analysis with EM observables.

\subsection*{E.6 Calibration Procedures}
\paragraph{Amplitude and Phase.}
Insert a reference loop with known coupling $k$.
Sweep a calibrated amplitude and retrieve transfer function $H(f)$ and phase $\phi(f)$.

\paragraph{Linearity.}
Vary input by $\pm\SI{20}{dB}$ and fit response to detect compression or intermodulation.

\paragraph{Environmental Subtraction.}
Measure with a dummy load (non-toroidal) to subtract environmental EM signatures.
Use \emph{dual-probe differential} to suppress common-mode pickup.

\subsection*{E.7 Data Analysis}
\paragraph{Filtering and FFT.} 
Apply zero-phase bandpass filtering around $f_n$ and $f_n\pm k f_m$.
Compute PSD/ASD and coherence; estimate sideband amplitudes and phases.

\paragraph{Statistical Tests.} 
Use permutation tests and false-discovery-rate (FDR) control across epochs.
Correlate EM features with task markers; report effect sizes and confidence intervals.

\paragraph{Theta Signature Search.} 
Search for modular/periodic patterns consistent with theta-basis expansions.
Document invariance (or lack thereof) under reparameterizations of $\psi$.

\subsection*{E.8 Safety and Reproducibility}
\paragraph{Safety.}
Respect RF exposure limits, thermal constraints, and electrical isolation.
EEG (if used) must meet medical safety standards and informed consent protocols.

\paragraph{Reproducibility.}
Publicly share CAD/PCB files, BOM, firmware, raw data, and analysis scripts.
Log all calibration constants and environmental conditions.

\subsection*{E.9 Predicted Observables (UBT)}
\begin{itemize}
\item Sidebands at $f_n \pm k f_m$ with amplitudes deviating from standard EM-only predictions by $\mathcal{O}(10^{-4}\text{--}10^{-6})$ in controlled $\psi$-protocols.
\item Tiny phase drifts in $\phi(f_n)$ correlated with $\psi$-epoch markers.
\item Robustness of patterns under modular transforms of the analysis window.
\end{itemize}

\subsection*{E.10 Closing Remark}
All protocols herein preserve compatibility with QED in the $\partial_\psi\!=\!0$ limit; any deviation must pass stringent controls.
Informally, one might say that \emph{``theta is the password of the universe''} --- or, more conservatively, its guiding mathematical principle reflected in the observed modular structure.

% End of Appendix E

% \end{document}

\section*{Appendix E: Experimental Protocols and Calibration of the \emph{Theta Resonator}}
\addcontentsline{toc}{section}{Appendix E: Experimental Protocols and Calibration of the Theta Resonator}

\subsection*{E.1 Purpose and Scope}
This appendix specifies the experimental blueprint, calibration procedures, and data-analysis methods for the \emph{Theta Resonator} --- a toroidal EM device designed to probe phase-coupled dynamics predicted by the Unified Biquaternion Theory (UBT) with complex time $\tau = t + i\psi$.
The goal is to detect small, reproducible signatures of $\psi$-dependent phase modulation in electromagnetic observables while remaining fully compatible with standard QED in the limit $\partial_\psi(\cdot)=0$.

\subsection*{E.2 Mathematical Model (Concise)}
Let $A_\mu$ be the EM potential and $\Psi$ a Dirac spinor. The UBT-extended QED Lagrangian density (restricted to the EM sector plus a scalar psychon field $\chi$) is
\begin{align}
\mathcal{L} &= \bar{\Psi}(i\gamma^\mu D_\mu - m)\Psi - \frac{1}{4}F_{\mu\nu}F^{\mu\nu}
+ \frac{1}{2}\eta_\psi\,(\partial_\psi \Psi)^\dagger(\partial_\psi \Psi)
+ \frac{1}{2}\kappa_\psi\,(\partial_\psi A_\mu)(\partial_\psi A^\mu) \nonumber\\
&\quad + \frac{1}{2}(\partial_\mu \chi)(\partial^\mu \chi) + \frac{1}{2}(\partial_\psi \chi)^2 - U(\chi)
- g_{\chi A}\,\chi\,F_{\mu\nu}F^{\mu\nu} - g_{\chi\tilde{A}}\,\chi\,F_{\mu\nu}\tilde{F}^{\mu\nu}
- g_{\chi \psi}\,\chi\,\bar{\Psi}\Psi .
\end{align}
Setting $\partial_\psi(\cdot)=0$ recovers standard QED predictions. The \emph{Theta Resonator} targets regimes where $\partial_\psi$-terms are small but non-zero, producing phase-sidebands and minute dispersion shifts.

\subsection*{E.3 Experimental Setup (3D Schematic)}
Figure~\ref{fig:torus3d} provides a 3D schematic of the toroidal resonator, including the primary winding, a phase modulator section, pickup probes, and environmental shielding.
The drawing is not to scale but preserves the essential geometry for replication.

\begin{figure}[h!]
\centering
\tdplotsetmaincoords{70}{110}% elevation, azimuth
\begin{tikzpicture}[tdplot_main_coords,scale=1.1]
  % Torus parameters (visual only)
  \def\R{3.0}    % major radius
  \def\r{0.9}    % minor radius
  \def\Nw{18}    % winding segments (visual)
  % Draw a coarse torus surface (meridians)
  \foreach \t in {0,20,...,340}{
    \begin{scope}[tdplot_rotated_coords]
      \tdplotsetrotatedcoords{\t}{0}{0}
      \draw[blue!40!black!60, line width=0.3pt, opacity=0.6]
        plot[domain=0:360,samples=50,variable=\u]
        ({(\R+\r*cos(\u))*cos(0)},
         {(\R+\r*cos(\u))*sin(0)},
         {\r*sin(\u)});
    \end{scope}
  }
  % Draw a few parallels
  \foreach \u in {0,45,...,315}{
    \draw[blue!30!black!50, line width=0.3pt, opacity=0.6]
      plot[domain=0:360,samples=80,variable=\t]
      ({(\R+\r*cos(\u))*cos(\t)},
       {(\R+\r*cos(\u))*sin(\t)},
       {\r*sin(\u)});
  }
  % Primary winding (as discrete loops)
  \foreach \k in {0,...,\Nw}{
    \pgfmathsetmacro{\ang}{\k*360/\Nw}
    \draw[red!70!black, line width=1.0pt, opacity=0.9]
      plot[domain=0:360,samples=60,variable=\u]
      ({(\R+0.55*\r*cos(\u))*cos(\ang)},
       {(\R+0.55*\r*cos(\u))*sin(\ang)},
       {0.55*\r*sin(\u)});
  }
  % Phase modulator block
  \draw[fill=green!30, draw=green!50!black, opacity=0.8]
       (\R+1.1*\r, -0.7, 0.4) -- (\R+1.8*\r, -0.7, 0.4) -- (\R+1.8*\r, 0.7, 0.4) -- (\R+1.1*\r, 0.7, 0.4) -- cycle;
  \node[anchor=west] at (\R+1.85*\r,0.0,0.6) {\small Phase Modulator};

  % Pickup probes
  \fill[orange!70!black] (\R-0.2, 0.0, 1.1) circle (2pt) node[anchor=south]{\small Probe 1};
  \fill[orange!70!black] (\R-0.2, 0.0,-1.1) circle (2pt) node[anchor=north]{\small Probe 2};

  % Shielding (wireframe ring)
  \draw[gray!60, dashed]
    plot[domain=0:360,samples=80,variable=\t]
    ({(\R+1.7*\r)*cos(\t)}, {(\R+1.7*\r)*sin(\t)}, {0.0});
  \node at (0,-\R-1.2,0) {\small Toroidal Resonator (3D schematic)};
\end{tikzpicture}
\caption{3D schematic of the \emph{Theta Resonator} using \texttt{tikz-3dplot}. Major/minor radii and winding are indicative.}
\label{fig:torus3d}
\end{figure}

\subsection*{E.4 Electrical Interconnection (circuitikz)}
The block diagram in Fig.~\ref{fig:circuit} shows the generator, phase modulator, toroidal resonator, low-noise amplifier (LNA), lock-in detection, and DAQ/PC control.

\begin{figure}[h!]
\centering
\begin{circuitikz}[american voltages]
  % Generator
  \draw (0,0) to[sV,l=$\mathrm{RF\ Generator}$] (0,3) to[short] (2,3);
  % Modulator block
  \draw (2,4) node[draw,fill=green!20,minimum width=2.0cm,minimum height=1.2cm]{\footnotesize Phase\\Modulator};
  \draw (2,3) -- (2,4);
  \draw (4,4) node[draw,fill=blue!10,minimum width=2.0cm,minimum height=1.2cm]{\footnotesize Toroidal\\Resonator};
  \draw (3,4) -- (3.0,4);
  \draw (2.95,4) -- (3.95,4);
  \draw (5,4) -- (6,4);
  % LNA and Lock-in
  \draw (6,4) to[american current source,l=$\mathrm{Probe}$] (6,1);
  \draw (6,1) node[draw,fill=orange!15,minimum width=2.2cm,minimum height=1.2cm]{\footnotesize LNA};
  \draw (8,1) node[draw,fill=yellow!20,minimum width=3.0cm,minimum height=1.2cm]{\footnotesize Lock-in\ Detector};
  \draw (7.1,1) -- (6.9,1); \draw (8.9,1) -- (9.1,1);
  \draw (10.5,1) -- (12,1);
  % DAQ/PC
  \draw (12,2.0) node[draw,fill=gray!15,minimum width=2.8cm,minimum height=1.8cm]{\footnotesize DAQ / PC};
  \draw (12,1) -- (12,2);
\end{circuitikz}
\caption{Block-level interconnection of the experiment: RF generator $\to$ phase modulator $\to$ toroidal resonator; pickup $\to$ LNA $\to$ lock-in $\to$ DAQ/PC.}
\label{fig:circuit}
\end{figure}

\subsection*{E.5 Measurement Protocol (Step-by-Step)}
\paragraph{Preparation.} 
Thermally stabilize the setup ($\pm\SI{0.1}{K}$), EM-shield the chamber ($>\SI{80}{dB}$ at carrier), and verify grounding topology.
Calibrate the frequency source and lock-in time constants.

\paragraph{Baseline.}
With the torus open-circuit and no phase modulation, acquire baseline spectra $S_0(f)$, noise density $N_0(f)$, and phase reference $\phi_0(f)$.

\paragraph{Resonance Identification.}
Sweep frequency to locate toroidal eigenmodes $f_n$.
Record $Q$-factors and mode shapes via probe positioning.

\paragraph{Phase Modulation.}
Apply controlled phase drive $\varphi(t)=\varphi_0 \sin(2\pi f_m t)$.
Measure sidebands at $f_n \pm k f_m$ and their dependence on $\varphi_0$ and $f_m$.

\paragraph{UBT-sensitive Runs.}
Enable $\psi$-protocols (attention tasks, meditative steady-state, randomized controls). 
Record synchronized EEG/ECG/EOG if applicable (ethically approved).
Mark epochs for correlation analysis with EM observables.

\subsection*{E.6 Calibration Procedures}
\paragraph{Amplitude and Phase.}
Insert a reference loop with known coupling $k$.
Sweep a calibrated amplitude and retrieve transfer function $H(f)$ and phase $\phi(f)$.

\paragraph{Linearity.}
Vary input by $\pm\SI{20}{dB}$ and fit response to detect compression or intermodulation.

\paragraph{Environmental Subtraction.}
Measure with a dummy load (non-toroidal) to subtract environmental EM signatures.
Use \emph{dual-probe differential} to suppress common-mode pickup.

\subsection*{E.7 Data Analysis}
\paragraph{Filtering and FFT.} 
Apply zero-phase bandpass filtering around $f_n$ and $f_n\pm k f_m$.
Compute PSD/ASD and coherence; estimate sideband amplitudes and phases.

\paragraph{Statistical Tests.} 
Use permutation tests and false-discovery-rate (FDR) control across epochs.
Correlate EM features with task markers; report effect sizes and confidence intervals.

\paragraph{Theta Signature Search.} 
Search for modular/periodic patterns consistent with theta-basis expansions.
Document invariance (or lack thereof) under reparameterizations of $\psi$.

\subsection*{E.8 Safety and Reproducibility}
\paragraph{Safety.}
Respect RF exposure limits, thermal constraints, and electrical isolation.
EEG (if used) must meet medical safety standards and informed consent protocols.

\paragraph{Reproducibility.}
Publicly share CAD/PCB files, BOM, firmware, raw data, and analysis scripts.
Log all calibration constants and environmental conditions.

\subsection*{E.9 Predicted Observables (UBT)}
\begin{itemize}
\item Sidebands at $f_n \pm k f_m$ with amplitudes deviating from standard EM-only predictions by $\mathcal{O}(10^{-4}\text{--}10^{-6})$ in controlled $\psi$-protocols.
\item Tiny phase drifts in $\phi(f_n)$ correlated with $\psi$-epoch markers.
\item Robustness of patterns under modular transforms of the analysis window.
\end{itemize}

\subsection*{E.10 Closing Remark}
All protocols herein preserve compatibility with QED in the $\partial_\psi\!=\!0$ limit; any deviation must pass stringent controls.
Informally, one might say that \emph{``theta is the password of the universe''} --- or, more conservatively, its guiding mathematical principle reflected in the observed modular structure.

% End of Appendix E

% \end{document}

\section*{Appendix E: Experimental Protocols and Calibration of the \emph{Theta Resonator}}
\addcontentsline{toc}{section}{Appendix E: Experimental Protocols and Calibration of the Theta Resonator}

\subsection*{E.1 Purpose and Scope}
This appendix specifies the experimental blueprint, calibration procedures, and data-analysis methods for the \emph{Theta Resonator} --- a toroidal EM device designed to probe phase-coupled dynamics predicted by the Unified Biquaternion Theory (UBT) with complex time $\tau = t + i\psi$.
The goal is to detect small, reproducible signatures of $\psi$-dependent phase modulation in electromagnetic observables while remaining fully compatible with standard QED in the limit $\partial_\psi(\cdot)=0$.

\subsection*{E.2 Mathematical Model (Concise)}
Let $A_\mu$ be the EM potential and $\Psi$ a Dirac spinor. The UBT-extended QED Lagrangian density (restricted to the EM sector plus a scalar psychon field $\chi$) is
\begin{align}
\mathcal{L} &= \bar{\Psi}(i\gamma^\mu D_\mu - m)\Psi - \frac{1}{4}F_{\mu\nu}F^{\mu\nu}
+ \frac{1}{2}\eta_\psi\,(\partial_\psi \Psi)^\dagger(\partial_\psi \Psi)
+ \frac{1}{2}\kappa_\psi\,(\partial_\psi A_\mu)(\partial_\psi A^\mu) \nonumber\\
&\quad + \frac{1}{2}(\partial_\mu \chi)(\partial^\mu \chi) + \frac{1}{2}(\partial_\psi \chi)^2 - U(\chi)
- g_{\chi A}\,\chi\,F_{\mu\nu}F^{\mu\nu} - g_{\chi\tilde{A}}\,\chi\,F_{\mu\nu}\tilde{F}^{\mu\nu}
- g_{\chi \psi}\,\chi\,\bar{\Psi}\Psi .
\end{align}
Setting $\partial_\psi(\cdot)=0$ recovers standard QED predictions. The \emph{Theta Resonator} targets regimes where $\partial_\psi$-terms are small but non-zero, producing phase-sidebands and minute dispersion shifts.

\subsection*{E.3 Experimental Setup (3D Schematic)}
Figure~\ref{fig:torus3d} provides a 3D schematic of the toroidal resonator, including the primary winding, a phase modulator section, pickup probes, and environmental shielding.
The drawing is not to scale but preserves the essential geometry for replication.

\begin{figure}[h!]
\centering
\tdplotsetmaincoords{70}{110}% elevation, azimuth
\begin{tikzpicture}[tdplot_main_coords,scale=1.1]
  % Torus parameters (visual only)
  \def\R{3.0}    % major radius
  \def\r{0.9}    % minor radius
  \def\Nw{18}    % winding segments (visual)
  % Draw a coarse torus surface (meridians)
  \foreach \t in {0,20,...,340}{
    \begin{scope}[tdplot_rotated_coords]
      \tdplotsetrotatedcoords{\t}{0}{0}
      \draw[blue!40!black!60, line width=0.3pt, opacity=0.6]
        plot[domain=0:360,samples=50,variable=\u]
        ({(\R+\r*cos(\u))*cos(0)},
         {(\R+\r*cos(\u))*sin(0)},
         {\r*sin(\u)});
    \end{scope}
  }
  % Draw a few parallels
  \foreach \u in {0,45,...,315}{
    \draw[blue!30!black!50, line width=0.3pt, opacity=0.6]
      plot[domain=0:360,samples=80,variable=\t]
      ({(\R+\r*cos(\u))*cos(\t)},
       {(\R+\r*cos(\u))*sin(\t)},
       {\r*sin(\u)});
  }
  % Primary winding (as discrete loops)
  \foreach \k in {0,...,\Nw}{
    \pgfmathsetmacro{\ang}{\k*360/\Nw}
    \draw[red!70!black, line width=1.0pt, opacity=0.9]
      plot[domain=0:360,samples=60,variable=\u]
      ({(\R+0.55*\r*cos(\u))*cos(\ang)},
       {(\R+0.55*\r*cos(\u))*sin(\ang)},
       {0.55*\r*sin(\u)});
  }
  % Phase modulator block
  \draw[fill=green!30, draw=green!50!black, opacity=0.8]
       (\R+1.1*\r, -0.7, 0.4) -- (\R+1.8*\r, -0.7, 0.4) -- (\R+1.8*\r, 0.7, 0.4) -- (\R+1.1*\r, 0.7, 0.4) -- cycle;
  \node[anchor=west] at (\R+1.85*\r,0.0,0.6) {\small Phase Modulator};

  % Pickup probes
  \fill[orange!70!black] (\R-0.2, 0.0, 1.1) circle (2pt) node[anchor=south]{\small Probe 1};
  \fill[orange!70!black] (\R-0.2, 0.0,-1.1) circle (2pt) node[anchor=north]{\small Probe 2};

  % Shielding (wireframe ring)
  \draw[gray!60, dashed]
    plot[domain=0:360,samples=80,variable=\t]
    ({(\R+1.7*\r)*cos(\t)}, {(\R+1.7*\r)*sin(\t)}, {0.0});
  \node at (0,-\R-1.2,0) {\small Toroidal Resonator (3D schematic)};
\end{tikzpicture}
\caption{3D schematic of the \emph{Theta Resonator} using \texttt{tikz-3dplot}. Major/minor radii and winding are indicative.}
\label{fig:torus3d}
\end{figure}

\subsection*{E.4 Electrical Interconnection (circuitikz)}
The block diagram in Fig.~\ref{fig:circuit} shows the generator, phase modulator, toroidal resonator, low-noise amplifier (LNA), lock-in detection, and DAQ/PC control.

\begin{figure}[h!]
\centering
\begin{circuitikz}[american voltages]
  % Generator
  \draw (0,0) to[sV,l=$\mathrm{RF\ Generator}$] (0,3) to[short] (2,3);
  % Modulator block
  \draw (2,4) node[draw,fill=green!20,minimum width=2.0cm,minimum height=1.2cm]{\footnotesize Phase\\Modulator};
  \draw (2,3) -- (2,4);
  \draw (4,4) node[draw,fill=blue!10,minimum width=2.0cm,minimum height=1.2cm]{\footnotesize Toroidal\\Resonator};
  \draw (3,4) -- (3.0,4);
  \draw (2.95,4) -- (3.95,4);
  \draw (5,4) -- (6,4);
  % LNA and Lock-in
  \draw (6,4) to[american current source,l=$\mathrm{Probe}$] (6,1);
  \draw (6,1) node[draw,fill=orange!15,minimum width=2.2cm,minimum height=1.2cm]{\footnotesize LNA};
  \draw (8,1) node[draw,fill=yellow!20,minimum width=3.0cm,minimum height=1.2cm]{\footnotesize Lock-in\ Detector};
  \draw (7.1,1) -- (6.9,1); \draw (8.9,1) -- (9.1,1);
  \draw (10.5,1) -- (12,1);
  % DAQ/PC
  \draw (12,2.0) node[draw,fill=gray!15,minimum width=2.8cm,minimum height=1.8cm]{\footnotesize DAQ / PC};
  \draw (12,1) -- (12,2);
\end{circuitikz}
\caption{Block-level interconnection of the experiment: RF generator $\to$ phase modulator $\to$ toroidal resonator; pickup $\to$ LNA $\to$ lock-in $\to$ DAQ/PC.}
\label{fig:circuit}
\end{figure}

\subsection*{E.5 Measurement Protocol (Step-by-Step)}
\paragraph{Preparation.} 
Thermally stabilize the setup ($\pm\SI{0.1}{K}$), EM-shield the chamber ($>\SI{80}{dB}$ at carrier), and verify grounding topology.
Calibrate the frequency source and lock-in time constants.

\paragraph{Baseline.}
With the torus open-circuit and no phase modulation, acquire baseline spectra $S_0(f)$, noise density $N_0(f)$, and phase reference $\phi_0(f)$.

\paragraph{Resonance Identification.}
Sweep frequency to locate toroidal eigenmodes $f_n$.
Record $Q$-factors and mode shapes via probe positioning.

\paragraph{Phase Modulation.}
Apply controlled phase drive $\varphi(t)=\varphi_0 \sin(2\pi f_m t)$.
Measure sidebands at $f_n \pm k f_m$ and their dependence on $\varphi_0$ and $f_m$.

\paragraph{UBT-sensitive Runs.}
Enable $\psi$-protocols (attention tasks, meditative steady-state, randomized controls). 
Record synchronized EEG/ECG/EOG if applicable (ethically approved).
Mark epochs for correlation analysis with EM observables.

\subsection*{E.6 Calibration Procedures}
\paragraph{Amplitude and Phase.}
Insert a reference loop with known coupling $k$.
Sweep a calibrated amplitude and retrieve transfer function $H(f)$ and phase $\phi(f)$.

\paragraph{Linearity.}
Vary input by $\pm\SI{20}{dB}$ and fit response to detect compression or intermodulation.

\paragraph{Environmental Subtraction.}
Measure with a dummy load (non-toroidal) to subtract environmental EM signatures.
Use \emph{dual-probe differential} to suppress common-mode pickup.

\subsection*{E.7 Data Analysis}
\paragraph{Filtering and FFT.} 
Apply zero-phase bandpass filtering around $f_n$ and $f_n\pm k f_m$.
Compute PSD/ASD and coherence; estimate sideband amplitudes and phases.

\paragraph{Statistical Tests.} 
Use permutation tests and false-discovery-rate (FDR) control across epochs.
Correlate EM features with task markers; report effect sizes and confidence intervals.

\paragraph{Theta Signature Search.} 
Search for modular/periodic patterns consistent with theta-basis expansions.
Document invariance (or lack thereof) under reparameterizations of $\psi$.

\subsection*{E.8 Safety and Reproducibility}
\paragraph{Safety.}
Respect RF exposure limits, thermal constraints, and electrical isolation.
EEG (if used) must meet medical safety standards and informed consent protocols.

\paragraph{Reproducibility.}
Publicly share CAD/PCB files, BOM, firmware, raw data, and analysis scripts.
Log all calibration constants and environmental conditions.

\subsection*{E.9 Predicted Observables (UBT)}
\begin{itemize}
\item Sidebands at $f_n \pm k f_m$ with amplitudes deviating from standard EM-only predictions by $\mathcal{O}(10^{-4}\text{--}10^{-6})$ in controlled $\psi$-protocols.
\item Tiny phase drifts in $\phi(f_n)$ correlated with $\psi$-epoch markers.
\item Robustness of patterns under modular transforms of the analysis window.
\end{itemize}

\subsection*{E.10 Closing Remark}
All protocols herein preserve compatibility with QED in the $\partial_\psi\!=\!0$ limit; any deviation must pass stringent controls.
Informally, one might say that \emph{``theta is the password of the universe''} --- or, more conservatively, its guiding mathematical principle reflected in the observed modular structure.

% End of Appendix E
