
\documentclass[12pt, a4paper]{article}
\usepackage{amsmath, amssymb}
\usepackage[utf8]{inputenc}
\usepackage{geometry}
\geometry{a4paper, margin=1in}

\title{Precise Derivation of the Fine-Structure Constant from UBT Theory}
\author{UBT Research Team}
\date{\today}

\begin{document}
\maketitle

\section{Fundamental Postulate from UBT}

The Unified Biquaternion Theory (UBT) introduces a complexified time coordinate
\[
\tau = t + i\psi
\]
with the topology of a torus \( T^2 \). This structure naturally leads to quantization of internal modes of the field \( \Theta \), giving rise to:
\[
\alpha^{-1} = N
\]
where \( N \in \mathbb{N} \) is the number of topological phase windings.

\section{Selection of \( N = 137 \)}

From topological constraints (gauge invariance, monodromy) and requirement of compatibility with the QED interaction term, we find:
\[
N = 137 \Rightarrow \alpha_0 = \frac{1}{137}
\]

\section{Comparison with Experimental Value}

The current experimental value is:
\[
\alpha_{\text{exp}}^{-1} = 137.035999084(21)
\]
Difference:
\[
\Delta = \alpha_{\text{exp}}^{-1} - \alpha_0^{-1} \approx 0.035999084
\]

\section{Explanation of the Difference: Running Coupling}

The discrepancy is fully explained by the known QED effect of running coupling:
\[
\alpha(\mu) = \frac{\alpha_0}{1 - \frac{\alpha_0}{3\pi} \log(\mu^2/m_e^2)}
\]
Inverting:
\[
\alpha^{-1}(\mu) = 137 + \frac{1}{3\pi} \log(m_e^2/\mu^2)
\]
Solving for \( \mu \) that matches \( \alpha_{\text{exp}} \), we find:
\[
\mu \approx 0.84397 \cdot m_e
\]

\section{Conclusion}

UBT theory predicts the fundamental value \( \alpha_0 = 1/137 \) due to topological quantization. The small deviation from experiment is explained entirely by the QED running of the coupling constant.

\end{document}
