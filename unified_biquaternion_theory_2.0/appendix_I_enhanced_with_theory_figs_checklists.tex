
\appendix
\section*{Appendix I: Hopfions and Topological Field Configurations in the Unified Biquaternion Theory}


% === UBT-Theoretical Derivation (placed at the start) ===
\subsection*{I.0 UBT-Derived Theoretical Foundations}
\paragraph{Unified field and complex time.}
Let $\Theta(q,\tau)\in\mathbb{B}\otimes\mathbb{C}^n$ denote the biquaternionic master field on spacetime coordinates $q^\mu$ and complex time $\tau=t+i\psi$.
A minimal UBT sector relevant for topological excitations is captured by the effective action
\begin{equation}
S[\Theta] \;=\; \int d^4x \, \Big\{
\frac{1}{2}\mathrm{Tr}\big[(D_\mu \Theta)^\dagger (D^\mu \Theta)\big]
- V(\Theta) \Big\}
\;+\; S_{\text{top}}[\Theta]\;+\;S_{\psi}[\Theta],
\end{equation}
where $D_\mu$ is a generalized (gauge) covariant derivative, $V$ is a symmetry-breaking potential, $S_{\text{top}}$ adds topological terms, and $S_{\psi}$ encodes slow $\psi$-sector dynamics.

\paragraph{Projection to $S^2$ and topological current.}
Write a normalized doublet $z=\Pi[\Theta]\in \mathbb{C}^2$ with $z^\dagger z=1$ and define $n^a=z^\dagger\sigma^a z\in S^2$.
The $U(1)$ one-form $A_i=-i\,z^\dagger\partial_i z$ has curvature $F_{ij}=\partial_iA_j-\partial_jA_i$.
The conserved topological current density in $3$D is
\begin{equation}
j_{\text{top}}^i \;=\; \frac{1}{4\pi^2}\,\epsilon^{ijk}\, A_j F_{k\ell}\, \hat{e}^\ell,\qquad
Q_H \;=\; \int d^3x\, j^0_{\text{top}} \;=\; \frac{1}{4\pi^2}\int d^3x\, \epsilon^{ijk} A_i F_{jk}\in\mathbb{Z},
\end{equation}
equal to the Hopf invariant $Q_H$.

\paragraph{Effective energy and Euler--Lagrange equation.}
Projecting $\Theta\mapsto n$ yields the (generalized) Faddeev--Skyrme energy
\begin{equation}
E[n]=\int d^3x\;\left\{\alpha\,(\partial_i n)^2+\beta\,\big(\epsilon^{ijk}\,n\cdot\partial_i n\times\partial_j n\big)^2\right\},\quad n\cdot n=1,
\end{equation}
with $\alpha,\beta>0$ functions of UBT couplings.
Variation with a Lagrange multiplier $\lambda$ gives
\begin{equation}
-2\alpha\,\Delta n -4\beta\,\partial_i\Big[\big(n\cdot\partial_i n\times\partial_j n\big)\,\partial_j n\times n\Big] \;=\; \lambda(x)\, n,
\end{equation}
whose finite-energy solutions in fixed $Q_H$ sectors are hopfions.

\paragraph{Derrick scaling and stability bound.}
Under $n(\mathbf{x})\mapsto n(\lambda\mathbf{x})$, $E(\lambda)=\alpha\lambda I_2+\beta\lambda^{-1}I_4$ with $I_2=\int(\partial n)^2$, $I_4=\int(\cdot)^2$.
Stationarity gives $E_{\min}=2\sqrt{\alpha\beta\,I_2 I_4}$; geometric inequalities imply the Vakulenko--Kapitanski bound $E\ge c\,|Q_H|^{3/4}$ with $c=2\sqrt{\alpha\beta C}>0$.

\paragraph{Quantization.}
Canonical quantization around a classical hopfion $n_0$ yields fluctuation modes from the Hessian of $E[n]$, while the integer $Q_H$ plays the role of a superselection label.
Slow $\psi$-dynamics ($\partial_\psi\neq 0$) adds a kinetic term $\frac12\kappa_\psi(\partial_\psi n)^2$ to the Lagrangian, shifting frequencies but preserving the topological bound.

\addcontentsline{toc}{section}{Appendix I: Hopfions and Topological Field Configurations in the Unified Biquaternion Theory}

\subsection*{I.1 Introduction}
In the Unified Biquaternion Theory (UBT), \emph{hopfions} are finite-energy knotted field configurations characterized by a nontrivial Hopf invariant $Q_H\in\mathbb{Z}$. 
They emerge as topological solitons within the master field $\Theta(q,\tau)$, where $\tau=t+i\psi$. 
This appendix provides \emph{derivations} underlying the qualitative predictions: field equations, scaling laws, stability bounds, and leading-order laboratory/astrophysical signatures.

\subsection*{I.2 Hopf Invariant and Gauge Construction}
Let $z(\mathbf{x})\in\mathbb{C}^2$ with $z^\dagger z=1$ parametrize $S^3$ and define $n^a=z^\dagger\sigma^a z\in S^2$.
The $U(1)$ connection $A_i=-i\,z^\dagger\partial_i z$ has curvature $F_{ij}=\partial_i A_j-\partial_j A_i$. 
The Hopf invariant is
\begin{equation}
Q_H=\frac{1}{4\pi^2}\int_{\mathbb{R}^3} d^3x\; \epsilon^{ijk} A_i F_{jk}\in\mathbb{Z}.
\label{eq:HopfQ}
\end{equation}
Finite energy enforces $n(\mathbf{x})\to n_\infty$ at $|\mathbf{x}|\to\infty$, compactifying space to $S^3$ and ensuring $Q_H$ is topological. 
(For an explicit $Q_H{=}1$ map and estimates, see Appendix G.)

\subsection*{I.3 Energy Functional from UBT Lagrangian}
Projecting the $\Theta$-sector to a unit vector $n(\mathbf{x})\in S^2$ yields the (generalized) Faddeev--Skyrme energy
\begin{equation}
E[n]=\int d^3x\;\Big\{\alpha\,(\partial_i n)^2+\beta\, \big(\epsilon^{ijk}\,n\cdot\partial_i n\times\partial_j n\big)^2 \Big\},
\label{eq:FSenergy}
\end{equation}
with $\alpha,\beta>0$ functions of UBT couplings (Appendix H). 
The Euler--Lagrange equation reads
\begin{align}
0&=\frac{\delta E}{\delta n}-\lambda(\mathbf{x})\,n
= -2\alpha\,\Delta n -4\beta\,\partial_i\Big[ \big(n\cdot\partial_i n\times\partial_j n\big)\, \partial_j n\times n \Big]-\lambda n,\quad n\cdot n=1,
\label{eq:EL}
\end{align}
where $\lambda$ enforces the $|n|=1$ constraint. 
Stable hopfions solve \eqref{eq:EL} in a fixed $Q_H$ sector.

\subsection*{I.4 Derrick Scaling and Stability Bound}
Under spatial scaling $n(\mathbf{x})\mapsto n(\lambda\mathbf{x})$, one finds
\begin{equation}
E(\lambda)=\alpha\,\lambda\,I_2+\beta\,\lambda^{-1} I_4,\qquad
I_2=\int (\partial_i n)^2,\ \ I_4=\int \big(\epsilon^{ijk}n\cdot\partial_i n\times\partial_j n\big)^2.
\end{equation}
Stationarity $\partial_\lambda E|_{\lambda_\star}=0$ gives $\lambda_\star=\sqrt{\beta I_4/(\alpha I_2)}$ and
\begin{equation}
E_{\min}=2\sqrt{\alpha\beta\,I_2 I_4}.
\end{equation}
Using geometric inequalities $I_2 I_4 \ge C\,|Q_H|^{3/2}$ yields the Vakulenko--Kapitanski bound
\begin{equation}
E \;\ge\; c\, |Q_H|^{3/4},\qquad c=2\sqrt{\alpha\beta\,C},
\label{eq:VK}
\end{equation}
ensuring stability against collapse for fixed $Q_H$.

\subsection*{I.5 Complex-Time Corrections and Small $\psi$-Sector Effects}
Allow a slow $\psi$-dependence of coefficients: $\alpha\to\alpha(\psi)$, $\beta\to\beta(\psi)$, and include a weak kinetic term $\frac12\kappa_\psi(\partial_\psi n)^2$ in the effective Lagrangian. 
Linearizing around a static hopfion solution $n_0$ gives the fluctuation operator
\begin{equation}
\mathcal{H}_{\mathrm{fluct}} \sim -2\alpha\,\Delta + \text{Skyrme terms} + \kappa_\psi\,\partial_\psi^2 + \cdots,
\end{equation}
whose positive spectrum (modulo zero modes) ensures linear stability. 
The $\psi$-sector shifts $E$ and the size $R_H\sim\sqrt{\beta/\alpha}$ by $\mathcal{O}(\partial_\psi^2)$ terms but preserves the topological bound \eqref{eq:VK}.

\subsection*{I.6 Couplings to Electromagnetism and Suppression of Cross Sections}
At leading order, gauge-invariant low-energy couplings read
\begin{equation}
\mathcal{L}_{\mathrm{int}}=- g_{nA}\,(\partial_i n\cdot\partial_j n) F^{ij} - g_{nF}\, (F_{ij})^2 (\partial_k n)^2 + \cdots.
\label{eq:Lint}
\end{equation}
For an incoming photon of momentum $k$ scattering off a static hopfion of size $R_H$, the dominant amplitude scales as
\begin{equation}
\mathcal{M} \sim g_{nA}\, (k R_H)^2\quad\Rightarrow\quad \sigma_{\gamma H}\sim |\mathcal{M}|^2 \sim g_{nA}^2\,(k R_H)^4,
\end{equation}
implying strong suppression at long wavelengths ($kR_H\ll 1$). 
This underlies the \emph{dark} character (Appendix G) while allowing tiny polarization-dependent effects in extreme fields (Appendix D).

\subsection*{I.7 Cavity Frequency Shift (Laboratory Signature)}
In a toroidal superconducting cavity (Appendix E), the hopfion-induced renormalization of the EM energy density modifies eigenfrequencies. 
To first order in \eqref{eq:Lint},
\begin{equation}
\frac{\Delta f}{f} \;=\; -\frac{1}{2}\frac{\int d^3x\ \delta\epsilon_{\mathrm{eff}}(\mathbf{x})\, |E(\mathbf{x})|^2 + \delta\mu_{\mathrm{eff}}(\mathbf{x})\, |B(\mathbf{x})|^2}{\int d^3x\ \epsilon_0 |E|^2 + \mu_0^{-1}|B|^2}\,,
\end{equation}
with effective perturbations $\delta\epsilon_{\mathrm{eff}},\delta\mu_{\mathrm{eff}}\propto g_{nA}\,(\partial n)^2 + g_{nF}\,(\partial n)^4$. 
For a localized hopfion of size $R_H$ at a field antinode,
\begin{equation}
\left|\frac{\Delta f}{f}\right| \sim \mathcal{O}\!\big( g_{nA}\, \langle(\partial n)^2\rangle\, V_H/V_{\mathrm{mode}} \big)\ \ll\ 1,
\end{equation}
where $V_H\sim R_H^3$ and $V_{\mathrm{mode}}$ is the mode volume. This provides a quantitative target for high-$Q$ measurements.

\subsection*{I.8 Astrophysical Scaling (Lensing / Rotation Curves)}
An ensemble of hopfions with number density $n_H$ and mean energy $E_H$ contributes $\rho_{\mathrm{DM}}=n_H E_H$. 
In spherical halos, the circular velocity obeys $v_c^2(r)=G M(<r)/r$; fitting $M(<r)=\int_0^r 4\pi r'^2 \rho_{\mathrm{DM}}(r')dr'$ with $E_H(Q_H)$ from \eqref{eq:VK} yields constraints on $n_H(r)$ and $R_H(r)$. 
Weak-lensing convergence $\kappa(\theta)\propto\Sigma(\theta)$ is likewise sensitive to the hopfion density profile, enabling consistency checks with observed NFW-like profiles.

\subsection*{I.9 Summary of Derived Predictions}
\begin{itemize}
\item \textbf{Stability \& size:} $R_H\sim \sqrt{\beta/\alpha}$, $E\ge c|Q_H|^{3/4}$; small $\psi$-sector corrections do not spoil the bound.
\item \textbf{EM scattering:} $\sigma_{\gamma H}\propto (k R_H)^4$ (long-wavelength suppression), enabling \emph{dark} behavior.
\item \textbf{Cavity shift:} $\Delta f/f$ proportional to $g_{nA}\langle(\partial n)^2\rangle$ times volume fraction $V_H/V_{\mathrm{mode}}$ (target for Appendix E setups).
\item \textbf{Astrophysics:} halo mass profiles and lensing consistent with ensembles of topologically stable knots parameterized by $(\alpha,\beta,Q_H)$.
\end{itemize}

\subsection*{I.10 Notes and Cross-References}
Detailed constructions, $Q_H{=}1$ examples, and order-of-magnitude numerics are provided in Appendix G. 
Couplings to QED and constraints from precision experiments are discussed in Appendix D; laboratory protocols in Appendix E; $\psi$-sector dynamics in Appendix F; relations to constants in Appendix H.


\subsection*{I.11 3D Schematic of a $Q_H=1$ Hopfion (TikZ)}
\begin{figure}[h!]
\centering
\tdplotsetmaincoords{70}{120}
\begin{tikzpicture}[tdplot_main_coords,scale=1.0, line cap=round,line join=round]
  % Parameters
  \def\R{2.2}
  \def\r{0.9}

  % Linked circles (Hopf link)
  % Circle 1 in x-y plane
  \draw[very thick, blue!70!black]
    plot[domain=0:360, samples=220, variable=\t]
    ({\R*cos(\t)}, {\R*sin(\t)}, {0});
  % Circle 2 in x-z plane
  \draw[very thick, red!70!black]
    plot[domain=0:360, samples=220, variable=\u]
    ({0}, {\r*cos(\u)}, {\r*sin(\u)});

  % Suggestive field-line arcs
  \foreach \k in {0,30,...,330}{
    \draw[blue!40!black, opacity=0.35]
      plot[domain=0:360, samples=90, variable=\t]
      ({(\R+0.25*cos(\k))*cos(\t)}, {(\R+0.25*cos(\k))*sin(\t)}, {0.18*sin(\k)});
  }

  \node[anchor=south] at (0,-2.8,0) {\small Linked preimages depicting a $Q_H=1$ configuration (schematic)};
\end{tikzpicture}
\caption{Conceptual 3D Hopf link representing the topology of a $Q_H=1$ hopfion.}
\label{fig:hopfion_3d_tikz}
\end{figure}

\subsection*{I.12 Quick Reference: Parameters $\to$ Observables}
\begin{table}[h!]
\centering
\begin{tabular}{|c|c|c|c|c|c|}
\hline
Parameter & Meaning & Typical Scale & Derived Size $R_H$ & Energy $E$ & EM Cross-Section $\sigma_{\gamma H}$ \\
\hline
$\alpha$ & gradient stiffness & model-dependent & \multirow{2}{*}{$\displaystyle R_H \sim \sqrt{\beta/\alpha}$} & \multirow{2}{*}{$\displaystyle E \gtrsim c\,|Q_H|^{3/4}$} & \multirow{2}{*}{$\displaystyle \propto (kR_H)^4$} \\
$\beta$ & Skyrme term coeff. & model-dependent & & & \\
\hline
$Q_H$ & Hopf charge & $1,2,\dots$ & $-$ & $\propto |Q_H|^{3/4}$ & $-$ \\
\hline
$\kappa_\psi$ & phase-sector kinetic & small & slight shift of $R_H$ & slight shift of $E$ & negligible at leading order \\
\hline
$g_{nA}$ & $(\partial n)^2 F$ coupling & tiny & $-$ & $-$ & $\sigma_{\gamma H} \sim g_{nA}^2 (k R_H)^4$ \\
\hline
$g_{nF}$ & $(F)^2 (\partial n)^2$ coupling & tiny & $-$ & cavity shift via $E,B$ & modifies effective medium \\
\hline
\end{tabular}
\caption{Parameter-to-observable map for hopfions in UBT. Relations assume the low-energy, long-wavelength regime and small phase-sector corrections.}
\label{tab:params_observables_hopfion}
\end{table}

\subsection*{I.13 Numerical Example (Physically Plausible, SI Units)}
We choose parameters consistent with a superconducting toroidal cavity experiment (Appendix E) at $f\!=\!5~\mathrm{GHz}$ ($\lambda\!\approx\!6~\mathrm{cm}$, $k\!=\!2\pi/\lambda\!\approx\!105~\mathrm{m^{-1}}$). 
Model the effective Faddeev--Skyrme coefficients as
\begin{equation}
\alpha = 0.20~\mathrm{J/m},\qquad \beta = 5.0\times 10^{-8}~\mathrm{J\,m},\qquad Q_H=1,\qquad \kappa\simeq 1.
\end{equation}
Then
\begin{equation}
R_H \;\sim\; \sqrt{\frac{\beta}{\alpha}} \;=\; \sqrt{\frac{5.0\times 10^{-8}}{0.20}}~\mathrm{m} \;\approx\; 5.0\times 10^{-4}~\mathrm{m}\quad (0.5~\mathrm{mm}),
\end{equation}
\begin{equation}
E_{Q=1} \;\sim\; \kappa\,\sqrt{\alpha\beta} \;=\; \sqrt{0.20\times 5.0\times 10^{-8}}~\mathrm{J} \;\approx\; 1.0\times 10^{-4}~\mathrm{J}\quad (0.1~\mathrm{mJ}).
\end{equation}
For EM scattering at $5~\mathrm{GHz}$, $kR_H \approx 105 \times 5\times 10^{-4} \approx 5.3\times 10^{-2}$, hence $(kR_H)^4 \approx 7.8\times 10^{-6}$. 
With a tiny coupling $g_{nA}=10^{-6}$ and a geometric area factor $\pi R_H^2 \approx 7.9\times 10^{-7}~\mathrm{m^2}$, a crude cross-section estimate
\begin{equation}
\sigma_{\gamma H} \;\sim\; \pi R_H^2 \, g_{nA}^2 \, (kR_H)^4 \;\approx\; 5.9\times 10^{-24}~\mathrm{m^2},
\end{equation}
demonstrates strong long-wavelength suppression ($\propto k^4$).

\paragraph{Cavity Frequency Shift.}
Let $V_H \sim \tfrac{4}{3}\pi R_H^3 \approx 5.2\times 10^{-10}~\mathrm{m^3}$ and a mode volume $V_{\mathrm{mode}}\sim 10^{-3}~\mathrm{m^3}$.
With $\langle (\partial n)^2\rangle \sim R_H^{-2} \approx 4.0\times 10^6~\mathrm{m^{-2}}$ and $g_{nA}=10^{-6}$,
\begin{equation}
\left|\frac{\Delta f}{f}\right| \;\sim\; g_{nA}\,\langle (\partial n)^2\rangle\,\frac{V_H}{V_{\mathrm{mode}}} \;\sim\; 2\times 10^{-6}.
\end{equation}
For $f=5~\mathrm{GHz}$ this corresponds to $\Delta f \sim 10~\mathrm{kHz}$ in the optimistic placement at a field antinode; realistic alignment and smaller $g_{nA}$ push $\Delta f/f$ into the $10^{-8}$--$10^{-7}$ range, within reach of high-$Q$ metrology.


\subsection*{I.14 ISM-Band Experimental Analog (2.4/5.8/10 GHz)

\subsection*{I.15 Modulation Chain and Sensor Placement}
\begin{figure}[h!]
\centering
\begin{circuitikz}[american voltages, scale=1.0]
% Source and split
\node[draw, fill=gray!15, minimum width=2.6cm, minimum height=1.2cm] (ref) at (0,0) {10 MHz Ref (GPSDO/Rb)};
\node[draw, fill=gray!15, minimum width=3.2cm, minimum height=1.2cm, right=2.5cm of ref] (pll) {PLL/DDS $f_0$};
\draw[->] (ref) -- (pll);
\node[draw, fill=yellow!20, minimum width=2.8cm, minimum height=1.2cm, right=2.8cm of pll] (iq) {IQ Vector Modulator};
\draw[->] (pll) -- (iq);
\node[draw, fill=green!20, minimum width=2.4cm, minimum height=1.2cm, below left=1.6cm and -0.4cm of iq] (ph120) {Phase $+120^\circ$};
\node[draw, fill=green!20, minimum width=2.4cm, minimum height=1.2cm, above left=1.6cm and -0.4cm of iq] (ph0) {Phase $0^\circ$};
\node[draw, fill=green!20, minimum width=2.4cm, minimum height=1.2cm, below right=1.6cm and -0.4cm of iq] (ph240) {Phase $+240^\circ$};
\draw[->] (iq) -- (ph0);
\draw[->] (iq) -- (ph120);
\draw[->] (iq) -- (ph240);

% Power amps
\node[draw, fill=orange!20, minimum width=1.8cm, minimum height=1.0cm, right=2.0cm of ph0] (pa0) {PA};
\node[draw, fill=orange!20, minimum width=1.8cm, minimum height=1.0cm, right=2.0cm of ph120] (pa120) {PA};
\node[draw, fill=orange!20, minimum width=1.8cm, minimum height=1.0cm, right=2.0cm of ph240] (pa240) {PA};
\draw[->] (ph0) -- (pa0);
\draw[->] (ph120) -- (pa120);
\draw[->] (ph240) -- (pa240);

% Couplers and loops
\node[draw, fill=blue!10, minimum width=2.2cm, minimum height=1.0cm, right=2.2cm of pa0] (loopxy) {Loop $xy$};
\node[draw, fill=red!10, minimum width=2.2cm, minimum height=1.0cm, right=2.2cm of pa120] (loopxz) {Loop $xz$};
\node[draw, fill=green!10, minimum width=2.2cm, minimum height=1.0cm, right=2.2cm of pa240] (loopyz) {Loop $yz$};
\draw[->] (pa0) -- (loopxy);
\draw[->] (pa120) -- (loopxz);
\draw[->] (pa240) -- (loopyz);

% Pickup and sensing
\node[draw, fill=orange!15, minimum width=2.6cm, minimum height=1.0cm, above right=2.0cm and -1.0cm of loopxz] (pickup) {E/H Pickup Probe};
\node[draw, fill=yellow!15, minimum width=2.4cm, minimum height=1.0cm, right=2.2cm of pickup] (lna) {Low-Noise Amp};
\node[draw, fill=purple!15, minimum width=2.6cm, minimum height=1.0cm, right=2.6cm of lna] (vna) {VNA / Lock-in};
\draw[->] (pickup) -- (lna);
\draw[->] (lna) -- (vna);

% Notes
\node[align=left, anchor=west] at ([xshift=1.1cm] loopxy.east) {\small Place loops orthogonally at cavity center.};
\node[align=left, anchor=west] at ([xshift=0.6cm] pickup.east) {\small Pickup at field antinode; map for $\Delta f/f$ and sidebands.};

\end{circuitikz}
\caption{Modulation chain with IQ vector modulator and sensor placement for EM hopfion-analog measurements. Phase drivers feed three orthogonal loops; pickup probe feeds a low-noise chain into VNA/lock-in.}
\label{fig:mod_chain_sensor}
\end{figure}
}
\paragraph{Will it work?} The setup below \emph{does} produce a controlled 3D near-field with linked phase loops (EM hopfion analog). It tests:
(i) stability of a knotted field pattern under perturbations,
(ii) cavity frequency shifts $\Delta f/f$ and sidebands under phase drive,
(iii) polarization-dependent scattering. 
It does not by itself prove UBT-specific $\psi$-sector couplings; those require the high-$Q$ protocols in Appendix E. Null results are informative and set bounds.

\paragraph{What it does.} Three orthogonal loop antennas ($\hat{x},\hat{y},\hat{z}$) driven at phases $0^\circ,120^\circ,240^\circ$ in a shielded cavity create a toroidal/twisted near-field with linked field-line structure. A small pickup and VNA/lock-in measure $S_{21}$, $\Delta f/f$, and sidebands $f\pm k f_m$ under slow phase modulation.

\begin{figure}[h!]
\centering
\tdplotsetmaincoords{70}{110}
\begin{tikzpicture}[tdplot_main_coords,scale=1.0, line cap=round,line join=round]
  % Axes
  \draw[->,gray!70] (-3,0,0)--(3,0,0) node[anchor=west] {$x$};
  \draw[->,gray!70] (0,-3,0)--(0,3,0) node[anchor=south] {$y$};
  \draw[->,gray!70] (0,0,-3)--(0,0,3) node[anchor=south] {$z$};

  % Shield (wireframe cube)
  \draw[gray!50, dashed] (-2,-2,-2) -- (2,-2,-2) -- (2,2,-2) -- (-2,2,-2) -- cycle;
  \draw[gray!50, dashed] (-2,-2, 2) -- (2,-2, 2) -- (2,2, 2) -- (-2,2, 2) -- cycle;
  \draw[gray!50, dashed] (-2,-2,-2) -- (-2,-2,2);
  \draw[gray!50, dashed] ( 2,-2,-2) -- ( 2,-2,2);
  \draw[gray!50, dashed] ( 2, 2,-2) -- ( 2, 2,2);
  \draw[gray!50, dashed] (-2, 2,-2) -- (-2, 2,2);

  % Three orthogonal loop antennas (radius 1)
  \def\R{1.2}
  % Loop in x-y plane
  \draw[very thick, blue!70!black]
    plot[domain=0:360, samples=180, variable=\t]
    ({\R*cos(\t)}, {\R*sin(\t)}, {0});
  % Loop in x-z plane
  \draw[very thick, red!70!black]
    plot[domain=0:360, samples=180, variable=\t]
    ({\R*cos(\t)}, {0}, {\R*sin(\t)});
  % Loop in y-z plane
  \draw[very thick, green!60!black]
    plot[domain=0:360, samples=180, variable=\t]
    ({0}, {\R*cos(\t)}, {\R*sin(\t)});

  % Phase arrows
  \draw[->,blue!70!black] (1.2,0,0) -- (1.8,0,0) node[anchor=west] {$0^\circ$};
  \draw[->,red!70!black]  (0,0,1.2) -- (0,0,1.8) node[anchor=west] {$120^\circ$};
  \draw[->,green!60!black] (0,1.2,0) -- (0,1.8,0) node[anchor=south] {$240^\circ$};

  % Pickup probe
  \shade[ball color=orange!80!black] (0.6,0.6,0.6) circle (2pt);
  \node[anchor=west,scale=0.8] at (0.7,0.6,0.6) {pickup};

\end{tikzpicture}
\caption{Three orthogonal loop antennas with $120^\circ$ phase progression in a shielded cavity produce a linked near-field (EM hopfion analog).}
\label{fig:loop_hopfion_analog}
\end{figure}

\begin{figure}[h!]
\centering
\begin{circuitikz}[american voltages]
  % Three channels: DDS/PLL -> phase shifter -> PA -> loop
  \draw (0,0) node[draw,fill=gray!15,minimum width=2cm,minimum height=1.2cm]{DDS/PLL $f_0$};
  \draw (2.5,0) node[draw,fill=green!20,minimum width=2.2cm,minimum height=1.2cm]{Phase $0^\circ$};
  \draw (5.3,0) node[draw,fill=orange!20,minimum width=2.2cm,minimum height=1.2cm]{PA};
  \draw (8.1,0) node[draw,fill=blue!10,minimum width=2.2cm,minimum height=1.2cm]{Loop $xy$};

  \draw[->] (1,0) -- (1.9,0);
  \draw[->] (3.1,0) -- (4.7,0);
  \draw[->] (5.9,0) -- (7.5,0);

  \draw (2.5,-2) node[draw,fill=green!20,minimum width=2.2cm,minimum height=1.2cm]{Phase $120^\circ$};
  \draw (5.3,-2) node[draw,fill=orange!20,minimum width=2.2cm,minimum height=1.2cm]{PA};
  \draw (8.1,-2) node[draw,fill=red!10,minimum width=2.2cm,minimum height=1.2cm]{Loop $xz$};
  \draw[->] (1,-2) -- (1.9,-2);
  \draw[->] (3.1,-2) -- (4.7,-2);
  \draw[->] (5.9,-2) -- (7.5,-2);

  \draw (2.5,-4) node[draw,fill=green!20,minimum width=2.2cm,minimum height=1.2cm]{Phase $240^\circ$};
  \draw (5.3,-4) node[draw,fill=orange!20,minimum width=2.2cm,minimum height=1.2cm]{PA};
  \draw (8.1,-4) node[draw,fill=green!10,minimum width=2.2cm,minimum height=1.2cm]{Loop $yz$};
  \draw[->] (1,-4) -- (1.9,-4);
  \draw[->] (3.1,-4) -- (4.7,-4);
  \draw[->] (5.9,-4) -- (7.5,-4);

  % Splitter from DDS
  \draw (0,-2) node[draw,fill=gray!15,minimum width=2cm,minimum height=1.2cm]{Power Splitter};
  \draw[->] (0,0) -- (0,-1.4);
  \draw[->] (0,-2) -- (1,-2);
  \draw[->] (0,-2) -- (1,-4);
  \draw[->] (0,-2) -- (1,0);

  % Measurement chain
  \draw (10.8,-2) node[draw,fill=yellow!20,minimum width=2.6cm,minimum height=1.2cm]{Pickup \& LNA};
  \draw (13.8,-2) node[draw,fill=purple!15,minimum width=2.6cm,minimum height=1.2cm]{VNA / Lock-in};
  \draw[->] (9.2,-2) -- (12.5,-2);
  \draw[->] (12.5,-2) -- (12.5,-2);
\end{circuitikz}
\caption{Block diagram: DDS/PLL source $\to$ phase shifters (0/120/240$^\circ$) $\to$ power amps $\to$ three orthogonal loops; pickup $\to$ LNA $\to$ VNA/lock-in.}
\label{fig:loop_block}
\end{figure}

\paragraph{Band options.} Choose ISM bands: 2.4 GHz ($\lambda\!\approx$12.5 cm), 5.8 GHz ($\lambda\!\approx$5.2 cm), 10 GHz ($\lambda\!\approx$3 cm). Loop radii $\sim \lambda/(4\pi)$ for near-field dominance; start with $R\!\approx\!1$--$2$ cm at 5.8 GHz.




% === Figures Section ===


\subsection*{I.14 ISM-Band Experimental Analog (2.4/5.8/10 GHz)

\subsection*{I.15 Modulation Chain and Sensor Placement}
\begin{figure}[h!]
\centering
\begin{circuitikz}[american voltages, scale=1.0]
% Source and split
\node[draw, fill=gray!15, minimum width=2.6cm, minimum height=1.2cm] (ref) at (0,0) {10 MHz Ref (GPSDO/Rb)};
\node[draw, fill=gray!15, minimum width=3.2cm, minimum height=1.2cm, right=2.5cm of ref] (pll) {PLL/DDS $f_0$};
\draw[->] (ref) -- (pll);
\node[draw, fill=yellow!20, minimum width=2.8cm, minimum height=1.2cm, right=2.8cm of pll] (iq) {IQ Vector Modulator};
\draw[->] (pll) -- (iq);
\node[draw, fill=green!20, minimum width=2.4cm, minimum height=1.2cm, below left=1.6cm and -0.4cm of iq] (ph120) {Phase $+120^\circ$};
\node[draw, fill=green!20, minimum width=2.4cm, minimum height=1.2cm, above left=1.6cm and -0.4cm of iq] (ph0) {Phase $0^\circ$};
\node[draw, fill=green!20, minimum width=2.4cm, minimum height=1.2cm, below right=1.6cm and -0.4cm of iq] (ph240) {Phase $+240^\circ$};
\draw[->] (iq) -- (ph0);
\draw[->] (iq) -- (ph120);
\draw[->] (iq) -- (ph240);

% Power amps
\node[draw, fill=orange!20, minimum width=1.8cm, minimum height=1.0cm, right=2.0cm of ph0] (pa0) {PA};
\node[draw, fill=orange!20, minimum width=1.8cm, minimum height=1.0cm, right=2.0cm of ph120] (pa120) {PA};
\node[draw, fill=orange!20, minimum width=1.8cm, minimum height=1.0cm, right=2.0cm of ph240] (pa240) {PA};
\draw[->] (ph0) -- (pa0);
\draw[->] (ph120) -- (pa120);
\draw[->] (ph240) -- (pa240);

% Couplers and loops
\node[draw, fill=blue!10, minimum width=2.2cm, minimum height=1.0cm, right=2.2cm of pa0] (loopxy) {Loop $xy$};
\node[draw, fill=red!10, minimum width=2.2cm, minimum height=1.0cm, right=2.2cm of pa120] (loopxz) {Loop $xz$};
\node[draw, fill=green!10, minimum width=2.2cm, minimum height=1.0cm, right=2.2cm of pa240] (loopyz) {Loop $yz$};
\draw[->] (pa0) -- (loopxy);
\draw[->] (pa120) -- (loopxz);
\draw[->] (pa240) -- (loopyz);

% Pickup and sensing
\node[draw, fill=orange!15, minimum width=2.6cm, minimum height=1.0cm, above right=2.0cm and -1.0cm of loopxz] (pickup) {E/H Pickup Probe};
\node[draw, fill=yellow!15, minimum width=2.4cm, minimum height=1.0cm, right=2.2cm of pickup] (lna) {Low-Noise Amp};
\node[draw, fill=purple!15, minimum width=2.6cm, minimum height=1.0cm, right=2.6cm of lna] (vna) {VNA / Lock-in};
\draw[->] (pickup) -- (lna);
\draw[->] (lna) -- (vna);

% Notes
\node[align=left, anchor=west] at ([xshift=1.1cm] loopxy.east) {\small Place loops orthogonally at cavity center.};
\node[align=left, anchor=west] at ([xshift=0.6cm] pickup.east) {\small Pickup at field antinode; map for $\Delta f/f$ and sidebands.};

\end{circuitikz}
\caption{Modulation chain with IQ vector modulator and sensor placement for EM hopfion-analog measurements. Phase drivers feed three orthogonal loops; pickup probe feeds a low-noise chain into VNA/lock-in.}
\label{fig:mod_chain_sensor}
\end{figure}
}
\paragraph{Will it work?} The setup below \emph{does} produce a controlled 3D near-field with linked phase loops (EM hopfion analog). It tests:
(i) stability of a knotted field pattern under perturbations,
(ii) cavity frequency shifts $\Delta f/f$ and sidebands under phase drive,
(iii) polarization-dependent scattering. 
It does not by itself prove UBT-specific $\psi$-sector couplings; those require the high-$Q$ protocols in Appendix E. Null results are informative and set bounds.

\paragraph{What it does.} Three orthogonal loop antennas ($\hat{x},\hat{y},\hat{z}$) driven at phases $0^\circ,120^\circ,240^\circ$ in a shielded cavity create a toroidal/twisted near-field with linked field-line structure. A small pickup and VNA/lock-in measure $S_{21}$, $\Delta f/f$, and sidebands $f\pm k f_m$ under slow phase modulation.

\begin{figure}[h!]
\centering
\tdplotsetmaincoords{70}{110}
\begin{tikzpicture}[tdplot_main_coords,scale=1.0, line cap=round,line join=round]
  % Axes
  \draw[->,gray!70] (-3,0,0)--(3,0,0) node[anchor=west] {$x$};
  \draw[->,gray!70] (0,-3,0)--(0,3,0) node[anchor=south] {$y$};
  \draw[->,gray!70] (0,0,-3)--(0,0,3) node[anchor=south] {$z$};

  % Shield (wireframe cube)
  \draw[gray!50, dashed] (-2,-2,-2) -- (2,-2,-2) -- (2,2,-2) -- (-2,2,-2) -- cycle;
  \draw[gray!50, dashed] (-2,-2, 2) -- (2,-2, 2) -- (2,2, 2) -- (-2,2, 2) -- cycle;
  \draw[gray!50, dashed] (-2,-2,-2) -- (-2,-2,2);
  \draw[gray!50, dashed] ( 2,-2,-2) -- ( 2,-2,2);
  \draw[gray!50, dashed] ( 2, 2,-2) -- ( 2, 2,2);
  \draw[gray!50, dashed] (-2, 2,-2) -- (-2, 2,2);

  % Three orthogonal loop antennas (radius 1)
  \def\R{1.2}
  % Loop in x-y plane
  \draw[very thick, blue!70!black]
    plot[domain=0:360, samples=180, variable=\t]
    ({\R*cos(\t)}, {\R*sin(\t)}, {0});
  % Loop in x-z plane
  \draw[very thick, red!70!black]
    plot[domain=0:360, samples=180, variable=\t]
    ({\R*cos(\t)}, {0}, {\R*sin(\t)});
  % Loop in y-z plane
  \draw[very thick, green!60!black]
    plot[domain=0:360, samples=180, variable=\t]
    ({0}, {\R*cos(\t)}, {\R*sin(\t)});

  % Phase arrows
  \draw[->,blue!70!black] (1.2,0,0) -- (1.8,0,0) node[anchor=west] {$0^\circ$};
  \draw[->,red!70!black]  (0,0,1.2) -- (0,0,1.8) node[anchor=west] {$120^\circ$};
  \draw[->,green!60!black] (0,1.2,0) -- (0,1.8,0) node[anchor=south] {$240^\circ$};

  % Pickup probe
  \shade[ball color=orange!80!black] (0.6,0.6,0.6) circle (2pt);
  \node[anchor=west,scale=0.8] at (0.7,0.6,0.6) {pickup};

\end{tikzpicture}
\caption{Three orthogonal loop antennas with $120^\circ$ phase progression in a shielded cavity produce a linked near-field (EM hopfion analog).}
\label{fig:loop_hopfion_analog}
\end{figure}

\begin{figure}[h!]
\centering
\begin{circuitikz}[american voltages]
  % Three channels: DDS/PLL -> phase shifter -> PA -> loop
  \draw (0,0) node[draw,fill=gray!15,minimum width=2cm,minimum height=1.2cm]{DDS/PLL $f_0$};
  \draw (2.5,0) node[draw,fill=green!20,minimum width=2.2cm,minimum height=1.2cm]{Phase $0^\circ$};
  \draw (5.3,0) node[draw,fill=orange!20,minimum width=2.2cm,minimum height=1.2cm]{PA};
  \draw (8.1,0) node[draw,fill=blue!10,minimum width=2.2cm,minimum height=1.2cm]{Loop $xy$};

  \draw[->] (1,0) -- (1.9,0);
  \draw[->] (3.1,0) -- (4.7,0);
  \draw[->] (5.9,0) -- (7.5,0);

  \draw (2.5,-2) node[draw,fill=green!20,minimum width=2.2cm,minimum height=1.2cm]{Phase $120^\circ$};
  \draw (5.3,-2) node[draw,fill=orange!20,minimum width=2.2cm,minimum height=1.2cm]{PA};
  \draw (8.1,-2) node[draw,fill=red!10,minimum width=2.2cm,minimum height=1.2cm]{Loop $xz$};
  \draw[->] (1,-2) -- (1.9,-2);
  \draw[->] (3.1,-2) -- (4.7,-2);
  \draw[->] (5.9,-2) -- (7.5,-2);

  \draw (2.5,-4) node[draw,fill=green!20,minimum width=2.2cm,minimum height=1.2cm]{Phase $240^\circ$};
  \draw (5.3,-4) node[draw,fill=orange!20,minimum width=2.2cm,minimum height=1.2cm]{PA};
  \draw (8.1,-4) node[draw,fill=green!10,minimum width=2.2cm,minimum height=1.2cm]{Loop $yz$};
  \draw[->] (1,-4) -- (1.9,-4);
  \draw[->] (3.1,-4) -- (4.7,-4);
  \draw[->] (5.9,-4) -- (7.5,-4);

  % Splitter from DDS
  \draw (0,-2) node[draw,fill=gray!15,minimum width=2cm,minimum height=1.2cm]{Power Splitter};
  \draw[->] (0,0) -- (0,-1.4);
  \draw[->] (0,-2) -- (1,-2);
  \draw[->] (0,-2) -- (1,-4);
  \draw[->] (0,-2) -- (1,0);

  % Measurement chain
  \draw (10.8,-2) node[draw,fill=yellow!20,minimum width=2.6cm,minimum height=1.2cm]{Pickup \& LNA};
  \draw (13.8,-2) node[draw,fill=purple!15,minimum width=2.6cm,minimum height=1.2cm]{VNA / Lock-in};
  \draw[->] (9.2,-2) -- (12.5,-2);
  \draw[->] (12.5,-2) -- (12.5,-2);
\end{circuitikz}
\caption{Block diagram: DDS/PLL source $\to$ phase shifters (0/120/240$^\circ$) $\to$ power amps $\to$ three orthogonal loops; pickup $\to$ LNA $\to$ VNA/lock-in.}
\label{fig:loop_block}
\end{figure}

\paragraph{Band options.} Choose ISM bands: 2.4 GHz ($\lambda\!\approx$12.5 cm), 5.8 GHz ($\lambda\!\approx$5.2 cm), 10 GHz ($\lambda\!\approx$3 cm). Loop radii $\sim \lambda/(4\pi)$ for near-field dominance; start with $R\!\approx\!1$--$2$ cm at 5.8 GHz.



\subsection*{I.16 Bring-Up Checklists (2.4 / 5.8 / 10 GHz)}
\paragraph{Common steps (all bands).}
\begin{enumerate}
\item Verify shielded enclosure (Faraday cage) continuity and grounding.
\item Calibrate VNA (SOLT) at the feed points; verify reference oscillator lock (GPSDO/Rb).
\item Validate phase alignment: inject tone, measure relative phase at each loop port; adjust to $0^\circ/120^\circ/240^\circ\pm 1^\circ$.
\item Set initial power $\leq +10$ dBm per loop; verify no spurious above \(-40\) dBc.
\item Map near-field with pickup probe; locate antinodes; record baseline $S_{21}$ and $Q$.
\end{enumerate}

\paragraph{2.4 GHz (ISM).}
\begin{itemize}
\item Loop radius $R \approx 2.0$--$2.5$ cm; cable lengths equalized within $< 2$ mm.
\item Start with cavity dimension $\sim 20$--$30$ cm; absorbers on seams to suppress parasitic modes.
\item Expected $\Delta f/f$: $10^{-8}$--$10^{-7}$ with clean phase drive; sidebands at $f_0\pm kf_m$ visible for $f_m=1$--$50$ Hz.
\end{itemize}

\paragraph{5.8 GHz (ISM).}
\begin{itemize}
\item Loop radius $R \approx 1.0$--$1.5$ cm; attention to solder joints and SMA torque.
\item Cavity dimension $\sim 10$--$18$ cm; higher sensitivity to tolerances; keep screws symmetrical.
\item Expected $\Delta f/f$: $10^{-8}$--$10^{-6}$ (optimistic placement); stronger sidebands for same $f_m$.
\end{itemize}

\paragraph{10 GHz (X-band, license/per local rules).}
\begin{itemize}
\item Loop radius $R \approx 0.6$--$1.0$ cm; silver-plated surfaces recommended to reduce losses.
\item Cavity dimension $\sim 6$--$12$ cm; use precision spacers and dielectric trimmers.
\item Expected $\Delta f/f$: up to $10^{-6}$ in best-case alignment; stricter EMC and safety handling.
\end{itemize}

\paragraph{Shut-down and data hygiene.}
\begin{enumerate}
\item Reduce drive to \(-20\) dBm, mute PAs, then power down sources (reverse of bring-up).
\item Save raw $S$-parameters, time traces, and instrument states; back up with metadata (temp, humidity, phase offsets).
\item Mark configuration, loop positions, and probe coordinates for reproducibility (Appendix O workflow).
\end{enumerate}

