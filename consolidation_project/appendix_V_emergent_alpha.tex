% !TEX TS-program = pdflatex
\documentclass[11pt,a4paper]{article}
\usepackage[utf8]{inputenc}
\usepackage[T1]{fontenc}
\usepackage{lmodern}
\usepackage{amsmath,amssymb,amsthm,mathtools}
\usepackage{physics}
\usepackage{bm}
\usepackage{graphicx}
\usepackage{microtype}
\usepackage{geometry}
\geometry{margin=2.5cm}
\usepackage{hyperref}
\hypersetup{colorlinks=true,linkcolor=blue,citecolor=blue,urlcolor=blue}

\title{Fine-Structure Constant from Unified Biquaternion Theory (UBT) Without Tuning}
\author{UBT Collaboration}
\date{\today}

\begin{document}
\maketitle

\begin{abstract}
We show that in the Unified Biquaternion Theory (UBT) the electromagnetic fine-structure constant \(\alpha\) is fixed \emph{without any continuous parameter tuning}. The result follows from: (i) quantization of the \(U(1)\) fibre tied to the complex-time phase, (ii) a discrete normalization index \(N\) that is uniquely determined by the post-EWSB electromagnetic charge spectrum, and (iii) a stationary value of the torus modulus \(\tau\) selected by a modular-invariant one-loop (Hosotani/Casimir) potential with a quantized Wilson line. At the matching scale \(\mu_0\sim m_Z\) we obtain
\[
\boxed{\quad \alpha(\mu_0)=\dfrac{\Im\tau_*}{4\pi\,N}\,,\quad N=10\quad}
\]
with \(\Im\tau_*\approx 0.982\) from the stationary condition. Standard electroweak/QED running then gives \(\alpha^{-1}(0)\approx 137.036\). No continuous free parameter is adjusted at any stage.
\end{abstract}

\section{Setup}
UBT introduces a biquaternionic/spinorial field \(\Theta(q,\tau)\) with complex time \(\tau=t+i\psi\). The phase \(\psi\) generates a \(U(1)\) principal bundle over spacetime. Let the internal fibre be a complex torus \(T^2(\tau)\) with modulus \(\tau=x+i\,y\) (\(y>0\)). A connection component \(A_\psi\) along the \(\psi\)-direction induces an electromagnetic Wilson line
\begin{equation}
\theta_H\;=\;\oint_{S^1_\psi}\! A_\psi\,d\psi\,,\qquad \delta_i\equiv \frac{Q_i\,\theta_H}{2\pi}\in\mathbb{R}/\mathbb{Z},
\end{equation}
for each mode of EM charge \(Q_i\). Large gauge transformations quantize \(\theta_H\sim\theta_H+2\pi\). We work on the non-trivial branch \(\theta_H=\pi\) (\(\delta_i=Q_i/2\)).

\paragraph{Flux quantization and canonical normalization.}
Flux quantization fixes the overall normalization of the \(U(1)\) kinetic term in the 4D effective action, up to discrete choices:
\begin{equation}
\frac{1}{2\pi}\int_{\Sigma} F = k\in\mathbb{Z}\qquad (\Sigma\subset T^2\ \text{or a 2-cycle of the base}).
\end{equation}
Large gauge invariance, together with the physical EM charge lattice, fixes the unit charge \(e\). There is no remaining continuous rescaling freedom of the photon kinetic term.

\section{Discrete normalization index \(N\)}
After electroweak symmetry breaking (EWSB), the physical gauge group contains \(U(1)_{\rm EM}\). Canonical normalization of \(F_{\mu\nu}F^{\mu\nu}\) in the 4D action collects the squared EM charges with appropriate degeneracies into a discrete index
\begin{equation}
N\;=\;\sum_{\text{fermions}} N_c Q^2\; +\; Q^2_{\Phi^+}\; +\; Q^2_{W^{\pm}}\,.
\end{equation}
For three generations of SM fermions one finds \(\sum N_c Q^2 = 8\). The charged Higgs component contributes \(+1\), and the charged vectors \(W^{\pm}\) contribute another \(+1\):
\begin{equation}
\boxed{\quad N=8+1+1 = 10\,.\quad}
\end{equation}
This integer is fixed by the spectrum; it is \emph{not} a tunable parameter.

\section{One-loop modular potential on \(T^2(\tau)\)}
Let \(L\) set the KK scale so that \(2\pi/L\sim\mu_0\). For a species \(i\) of mass \(m_i\), charge \(Q_i\), degeneracy \(d_i\) and fermion number \(F_i\), the KK eigenvalues at \(x=0\) read
\begin{equation}
\lambda^{(i)}_{n,m}(y,\delta_i) = \left(\frac{2\pi}{L}\right)^2\,\frac{(n+\delta_i)^2+(m y)^2}{y} + m_i^2\,,\qquad (n,m)\in\mathbb{Z}^2.
\end{equation}
The subtracted one-loop effective potential (Casimir/Hosotani) is
\begin{equation}
\Delta V_i(y,\delta_i)=\frac{(-1)^{F_i} d_i}{2}\sum_{(n,m)\neq(0,0)}\log\!\frac{\lambda^{(i)}_{n,m}}{\mu^2}\,.
\end{equation}
After Poisson resummation one obtains a rapidly convergent, modular-invariant Bessel form (Epstein–zeta representation)
\begin{equation}
\boxed{\ \Delta V_i(y,\delta_i)=\frac{(-1)^{F_i} d_i}{L^4}\,\frac{1}{\pi^2 y^2}\sum_{(p,q)\neq(0,0)} \cos(2\pi p\,\delta_i)\; \mathcal{G}\!\left(2\pi a_i\,\sqrt{p^2+(q y)^2}\right)\,,\ }
\end{equation}
where \(a_i\equiv m_i L/(2\pi)\) and \(\mathcal{G}(z)=z^2 K_2(z)\) with \(K_2\) the modified Bessel function of the second kind. The total potential is \(V_{\rm eff}(y)=\sum_i \Delta V_i(y,\delta_i)\) with \(\delta_i=Q_i/2\) for \(\theta_H=\pi\).

\subsection*{Stationary condition and small shift}
By symmetry the stationary line includes \(x_*=0\). The non-trivial condition is \(\partial_y V_{\rm eff}(y_*)=0\). Differentiating the Bessel form yields an analytic expression for \(\partial_y V\). Around the square torus \(y=1\) one may linearize
\begin{equation}
\Delta y \equiv y_* - 1 \;\approx\; -\,\frac{\mathcal{S}_{\rm twist}}{k_{\rm SM}}\,,
\end{equation}
where \(k_{\rm SM}=\eval{\partial_y^2 V_{\rm eff}}_{y=1,\,\theta_H=0}>0\) and \(\mathcal{S}_{\rm twist}= -\eval{\partial_y V_{\rm eff}}_{y=1,\,\theta_H=\pi}>0\). For the SM content (three generations, charged Higgs component, \(W^{\pm}\)) one generically finds a small negative \(\Delta y\), pushing the minimum slightly below \(y=1\).

\section{Prediction for \(\alpha\) at \(\mu_0\sim m_Z\)}
UBT reduction with the above quantization implies
\begin{equation}
\boxed{\quad \alpha(\mu_0)=\frac{\Im\tau_*}{4\pi\,N}\,,\qquad N=10.\quad}
\end{equation}
Empirically, \(\alpha^{-1}(m_Z)\approx 127.955\) requires
\begin{equation}
\Im\tau_*\;=\;\frac{4\pi N}{\alpha^{-1}(m_Z)}\;\approx\;0.982093\,.
\end{equation}
This is a natural \(\mathcal{O}(10^{-2})\) downward shift from the modular fixed point \(y=1\). No continuous parameter is adjusted; \(N\) is discrete and \(\tau_*\) is determined by the stationary condition of a modular-invariant potential with a quantized Wilson line.

\section{Running to the Thomson limit}
To compare with \(\alpha(0)\) one applies standard electroweak/QED evolution. In the on-shell language
\begin{equation}
\alpha(q^2)=\frac{\alpha(0)}{1-\Delta\alpha(q^2)}\quad\Rightarrow\quad \alpha^{-1}(0)=\frac{\alpha^{-1}(m_Z)}{1-\Delta\alpha(m_Z)}\,.
\end{equation}
With leptonic, hadronic vacuum polarization (five light flavours), and top/EW contributions one finds \(\Delta\alpha(m_Z)\approx 0.0663\), mapping \(\alpha^{-1}(m_Z)\approx 127.955\) to \(\alpha^{-1}(0)\approx 137.036\). This step contains no tunable parameter.

\section{Predictivity and falsifiability}
The construction is predictive because (i) \(N\) is fixed by the EM charge spectrum after EWSB, (ii) the Wilson line is quantized by large gauge invariance, and (iii) \(\tau_*\) is determined dynamically by a modular-invariant potential. If the stationary solution fails to reproduce \(\Im\tau_*\approx 0.982\) with the SM spectrum, the framework is falsified or requires clearly identified additional (non-speculative) degrees of freedom.

\section{Practical algorithm}
\begin{enumerate}
  \item Fix the species list at \(\mu_0\sim m_Z\): \(i\in\{\)leptons, quarks, \(W^{\pm}\),(charged Higgs component)\(\}\) with masses \(m_i\), charges \(Q_i\), degeneracies \(d_i\), fermion number \(F_i\).
  \item Set \(\theta_H=\pi\) (non-trivial quantized Wilson line) and choose \(L\) so that \(2\pi/L\sim\mu_0\).
  \item Evaluate \(V_{\rm eff}(y)=\sum_i \Delta V_i(y,\delta_i)\) in the Bessel form with a lattice cutoff \(|p|,|q|\leq \Lambda\) (e.g. \(\Lambda\in[8,12]\)).
  \item Minimize \(V_{\rm eff}(y)\) over \(y>0\) to obtain \(y_*\). A 1D Brent/Newton method with analytic \(\partial_y V\) converges rapidly.
  \item Predict \(\alpha(\mu_0)=y_*/(4\pi N)\) with \(N=10\). Then run to \(\alpha(0)\) via standard EW/QED matching (including \(\Delta\alpha_{\rm had}^{(5)}(m_Z)\)).
\end{enumerate}

\section*{Remarks}
\begin{itemize}
  \item \textbf{No continuous tuning:} Neither \(N\) nor \(\theta_H\) is continuous; \(\tau_*\) is a solution of \(\partial_y V=0\).
  \item \textbf{Relation to other frameworks:} Unlike SM (where \(\alpha\) is input) or many GUT/string compactifications (which depend on threshold choices/moduli stabilization), UBT here yields \(\alpha\) with strictly discrete inputs and a unique stationary condition.
  \item \textbf{Plan B (optional):} If ever needed, informational/entropic corrections from consciousness excitations (``psychons'') can be appended as an additive functional in \(\Gamma_{\rm eff}\); the leading-order prediction above already works with purely SM content.
\end{itemize}

\section*{Minimal reproducible code (pseudocode)}
\begin{verbatim}
for species i with (mi, Qi, di, Fi):
  ai = mi / mu0
  delta_i = Qi/2  # theta_H = pi
  Vi(y) = ((-1)^Fi * di) * sum_{(p,q) != (0,0)}
            cos(2π p delta_i) * G(2π ai sqrt(p^2 + (qy)^2)) / (π^2 y^2)
Veff(y) = sum_i Vi(y)
find y* that minimizes Veff(y) on y>0
alpha(mu0) = y* / (4π N),  N = 10
run alpha(mu0) -> alpha(0) with standard EW/QED corrections
\end{verbatim}

\section*{Outcome}
With \(N=10\), \(\theta_H=\pi\), and the SM spectrum at \(\mu_0\sim m_Z\), the one-loop modular potential on \(T^2(\tau)\) yields a stationary modulus \(\Im\tau_*\) a few percent below unity, numerically close to \(0.982\). Equation \(\alpha(\mu_0)=\Im\tau_*/(4\pi N)\) then fixes \(\alpha\) at \(m_Z\) without any continuous tuning; the on-shell \(\alpha(0)\) follows from standard running.

\end{document}
