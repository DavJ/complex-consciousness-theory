
\section{Appendix E: Standard Model Coupling and QCD Embedding in UBT}
\label{app:sm-qcd-ubt}

\subsection{Overview}
This appendix restores and consolidates the linkage between the Unified Biquaternion Theory (UBT) and the
Standard Model (SM) gauge structure, with special emphasis on the QCD sector. We present a consistent dictionary
from the UBT geometric variables to the SM gauge potentials and field strengths, and we state matching conditions
and running-coupling relations compatible with Appendices~\ref{app:alpha-consolidated} and \ref{app:padic-rigorous}.

\subsection{Gauge bundle and connections}
Let the SM gauge group be
\[
\mathbb{G} \;\cong\; SU(3)_c \times SU(2)_L \times U(1)_Y\,.
\]
We introduce gauge connections (one-forms) and field strengths:
\begin{align}
G_\mu &\;=\; G_\mu^a T^a \in \mathfrak{su}(3), &
G_{\mu\nu} &= \partial_\mu G_\nu - \partial_\nu G_\mu + i g_s\,[G_\mu, G_\nu], \\
W_\mu &\;=\; W_\mu^i \tau^i \in \mathfrak{su}(2), &
W_{\mu\nu} &= \partial_\mu W_\nu - \partial_\nu W_\mu + i g\,[W_\mu, W_\nu], \\
B_\mu &\in \mathfrak{u}(1), &
B_{\mu\nu} &= \partial_\mu B_\nu - \partial_\nu B_\mu.
\end{align}
The covariant derivative acting on a matter field $\Psi$ in a representation $(\mathbf{3},\mathbf{2},Y)$ reads
\begin{equation}
D_\mu \Psi \;=\; \Big(\partial_\mu + i g_s G_\mu^a T^a + i g W_\mu^i \tau^i + i g^\prime Y B_\mu\Big)\Psi.
\end{equation}

\subsection{UBT $\to$ SM dictionary}
UBT provides a unified connection $\mathcal{A}_\mu$ on the $\psi$-fibered spacetime. We assume a block-diagonal projection
\begin{equation}
\mathcal{A}_\mu \;\longmapsto\; (G_\mu,\, W_\mu,\, B_\mu)
\end{equation}
such that the $U(1)$ normalization is fixed by the Chern quantization as in Appendix~\ref{app:alpha-consolidated}.
The electric charge operator obeys $Q = T^3 + Y/2$, and the electroweak mixing is
\begin{equation}
\begin{pmatrix} A_\mu \\ Z_\mu \end{pmatrix} \;=\;
\begin{pmatrix} \cos\theta_W & \sin\theta_W \\ -\sin\theta_W & \cos\theta_W \end{pmatrix}
\begin{pmatrix} B_\mu \\ W^3_\mu \end{pmatrix},
\qquad
e \;=\; g \sin\theta_W \;=\; g^\prime \cos\theta_W.
\end{equation}
At low energies $e$ matches $\alpha$ derived in Appendix~\ref{app:alpha-consolidated}. The determination of $\theta_W$ and $(g,g^\prime)$
requires additional matching conditions (left for future work) or a unification hypothesis.

\subsection{Gauge-invariant Lagrangian}
The gauge kinetic terms are
\begin{equation}
\mathcal{L}_{\rm gauge} \;=\; -\frac{1}{4}\,G_{\mu\nu}^a G^{a\,\mu\nu} \;-\; \frac{1}{4}\,W_{\mu\nu}^i W^{i\,\mu\nu} \;-\; \frac{1}{4}\,B_{\mu\nu} B^{\mu\nu}.
\end{equation}
For QCD with $n_f$ quark flavors the matter part includes
\begin{equation}
\mathcal{L}_{\rm QCD}^{\rm matter} \;=\; \sum_{f=1}^{n_f} \bar{q}_f\,(i\gamma^\mu D_\mu - m_f)\,q_f\,,
\qquad D_\mu q \;=\; (\partial_\mu + i g_s G_\mu^a T^a)q.
\end{equation}

\subsection{Running couplings and matching}
\paragraph{QED.} The low-energy fine-structure constant $\alpha(\mu)$ is fixed by the UBT topological integer $N$ and
vacuum polarization as in Appendix~\ref{app:alpha-consolidated}.

\paragraph{QCD.} The strong coupling runs according to
\begin{equation}
\alpha_s(\mu) \;=\; \frac{g_s^2(\mu)}{4\pi} \;=\; \frac{1}{\beta_0 \ln(\mu^2/\Lambda_{\rm QCD}^2)}\Big(1 - \frac{\beta_1}{\beta_0^2}\frac{\ln\ln(\mu^2/\Lambda^2_{\rm QCD})}{\ln(\mu^2/\Lambda^2_{\rm QCD})} + \cdots\Big),
\end{equation}
with $\beta_0=\tfrac{11}{4\pi}\!-\!\tfrac{n_f}{6\pi}$ and $\beta_1=\tfrac{102}{(4\pi)^2}\!-\!\tfrac{38\,n_f}{(4\pi)^2}$ in the $\overline{\rm MS}$ scheme. Asymptotic freedom ($\beta_0>0$) and confinement at low $\mu$ are consistent with a knotted-flux interpretation in the $\Theta$ sector.

\subsection{Topological interpretation of QCD in UBT}
Color flux tubes correspond to knotted configurations of $\Theta$ with nontrivial linking.
Wilson loops $\langle \mathrm{Tr}\, \mathcal{P}\exp i\oint G\rangle$ map to holonomies of $\mathcal{A}_\mu$ in the UBT fiber;
an area law for large loops is compatible with an energy cost proportional to knotted tube length and curvature.
Instanton sectors ($\pi_3(SU(2))\cong \mathbb{Z}$) mirror Hopf-like textures, providing a common topological language for both EM and QCD sectors.

\subsection{Matching conditions and open tasks}
\begin{itemize}
\item \textbf{Normalization:} $U(1)$ is fixed by Chern quantization (Appendix~\ref{app:alpha-consolidated}). The QCD normalization is anchored by $\Lambda_{\rm QCD}$; in UBT one expects $\Lambda_{\rm QCD}\sim \xi\,\mu_{\rm int}$, with the internal-mode scale $\mu_{\rm int}$ from the electron sector and $\xi=\mathcal{O}(1)$ to be fitted.
\item \textbf{Electroweak mixing:} determining $\theta_W$ from UBT requires an additional symmetry or a unification hypothesis; otherwise it is an independent parameter.
\item \textbf{Anomalies:} the SM matter assignment must satisfy anomaly cancellation; UBT embeddings should preserve this (check fermion content mapping).
\item \textbf{Hadron phenomenology:} flux-tube/knotted-state spectra vs.\ lattice-QCD input is an avenue for quantitative tests.
\end{itemize}

\subsection{Consistency with dark matter appendix}
The interaction portals between the $\Theta$ topological sector and colored matter are suppressed by orthogonality (complex-time fiber)
and higher-dimensional operators. Therefore QCD does not spoil the DM stability discussed in Appendix~\ref{app:dm-consolidated}, while gravitational coupling remains universal.
