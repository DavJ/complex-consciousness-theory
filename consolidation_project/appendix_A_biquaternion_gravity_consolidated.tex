
\section{Biquaternion Gravity}

This appendix presents the gravitational sector of the Unified Biquaternion Theory (UBT).
It combines the core theoretical framework from the original Biquaternion Gravity appendix
with the detailed analysis from the Quantum Gravity solution files, reformulated for
clarity and coherence.

\subsection{Introduction}

The gravitational field in UBT emerges naturally from the covariant formulation of the
biquaternionic tensor-spinor field equations. The metric tensor is propose from the real
part of the scalar product in the biquaternion space, ensuring compatibility with the
principles of General Relativity while extending them to the complexified algebraic
structure.

In this formulation, spacetime is represented by a projection from a higher-dimensional
complex manifold, and curvature is encoded in the covariant derivatives of the
biquaternion field $\Theta(q,\tau)$. The gravitational interaction is therefore not an
independent postulate, but a manifestation of the underlying field geometry.

\subsection{Core Equations}

The line element is expressed as:
\begin{equation}
  ds^2 = g_{\mu\nu} \, dx^\mu dx^\nu ,
\end{equation}
where the metric tensor $g_{\mu\nu}$ is obtained from the biquaternion field via:
\begin{equation}
  g_{\mu\nu} = \Re\left[ \frac{\partial_\mu \Theta \cdot \partial_\nu \Theta^\dagger}{\mathcal{N}} \right],
\end{equation}
with $\mathcal{N}$ a normalisation factor ensuring the correct signature.

The Einstein tensor in this framework takes the standard form:
\begin{equation}
  G_{\mu\nu} = R_{\mu\nu} - \frac{1}{2} g_{\mu\nu} R ,
\end{equation}
but with curvature tensors $R_{\mu\nu}$ and $R$ propose from the biquaternionic connection
coefficients $\Gamma^\rho_{\mu\nu}$ obtained from the extended algebra.

The gravitational field equations couple $G_{\mu\nu}$ to the stress-energy tensor
$T_{\mu\nu}$ constructed from the biquaternion field invariants:
\begin{equation}
  G_{\mu\nu} = 8\pi G \, T_{\mu\nu}[\Theta] .
\end{equation}

\subsection{analysis Summary}

The original analysis proceeds by first defining the biquaternionic connection
compatible with the metric propose from $\Theta(q,\tau)$. The connection coefficients
are computed as:
\begin{equation}
  \Gamma^\rho_{\mu\nu} =
  \frac{1}{2} g^{\rho\sigma} \left( \partial_\mu g_{\nu\sigma}
  + \partial_\nu g_{\mu\sigma}
  - \partial_\sigma g_{\mu\nu} \right),
\end{equation}
where $g_{\mu\nu}$ is substituted from the field definition above.

The curvature tensor $R^\rho_{\ \sigma\mu\nu}$ is then obtained from:
\begin{equation}
  R^\rho_{\ \sigma\mu\nu} =
  \partial_\mu \Gamma^\rho_{\nu\sigma} -
  \partial_\nu \Gamma^\rho_{\mu\sigma} +
  \Gamma^\rho_{\mu\lambda} \Gamma^\lambda_{\nu\sigma} -
  \Gamma^\rho_{\nu\lambda} \Gamma^\lambda_{\mu\sigma} .
\end{equation}

Contracting appropriately yields $R_{\mu\nu}$ and the scalar curvature $R$.

In the quantum gravity extension, fluctuations of the field $\Theta$ are quantised,
leading to corrections to the classical curvature in the form of effective stress-energy
terms arising from vacuum polarisation effects. The semiclassical approximation shows that
these corrections become significant near the Planck scale, modifying the black hole
horizon structure and potentially allowing for stable micro-horizon configurations.

\subsection{Summary}

Biquaternion Gravity provides a natural embedding of General Relativity into a
complexified algebraic framework, unifying gravity with other interactions at the
geometric level. The connection to Quantum Gravity arises from treating $\Theta$ as a
quantum field, where gravitational effects emerge from its covariant structure. This
approach suggest possible deviations from classical GR at very small scales, while
reducing to Einstein's equations in the macroscopic limit.
