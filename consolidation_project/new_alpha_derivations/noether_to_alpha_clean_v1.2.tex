
\documentclass[11pt]{article}
\usepackage[margin=2.2cm]{geometry}
\usepackage{amsmath,amssymb,amsfonts,physics}
\usepackage{tcolorbox}
\usepackage{bm}

\title{Noether $\Rightarrow$ $\alpha$ (clean route, UBT-only)}
\author{One-page derivation}
\date{\today}

\begin{document}
\maketitle
\vspace{-1.2em}

\section*{Setup}
Let $\Theta$ carry a global $U(1)$ phase symmetry generated by $Q$. The 5D background is
$M^4\times S^1_\psi$ with metric $ds^2=e^{2A(\psi)}\eta_{\mu\nu}dx^\mu dx^\nu + e^{2B(\psi)}d\psi^2$.
Noether's theorem gives a conserved current $J^M$; gauging the symmetry amounts to
\begin{equation}
\partial_M \;\to\; D_M \equiv \partial_M + i\,g_5\,Q\,A_M,
\end{equation}
and (either fundamental or emergent) gauge dynamics produces $-\frac{1}{4g_5^2}\!\int\!\sqrt{|g_5|}F_{MN}F^{MN}$.

\section*{Dimensional reduction}
Expand the photon in 4D zero-mode $A_\mu^{(0)}(x)$ with profile $\xi_0(\psi)$:
\begin{equation}
A_\mu(x,\psi) = \xi_0(\psi)\,A_\mu^{(0)}(x) + \cdots,\qquad
Z^\star \equiv \int_0^{L_\psi^\star}\! e^{B-2A}\,|\xi_0(\psi)|^2\,d\psi.
\end{equation}
Canonical normalization gives
\begin{equation}
\frac{1}{g_4^2} \;=\; \frac{Z^\star}{g_5^2}\,,
\qquad
\Rightarrow\quad
\alpha \equiv \frac{g_4^2}{4\pi} \;=\; \frac{g_5^2}{4\pi\,Z^\star}.
\end{equation}

\section*{Holonomy fixes $g_5$}
The vacuum background $A_\psi(\psi)$ is characterized by the gauge-invariant holonomy
\begin{equation}
\theta_H \;=\; g_5\oint_{S^1_\psi} A_\psi\,d\psi \;=\; g_5\,\mathcal I_\psi,\qquad
\mathcal I_\psi \equiv \int_0^{L_\psi} A_\psi(\psi)\,d\psi.
\end{equation}
Large gauge invariance implies $\theta_H\sim\theta_H+2\pi n$. Minimizing $V_{\rm eff}(L_\psi,\theta_H)$ in UBT picks a stationary value $\theta_H^\star$ and a vacuum length $L_\psi^\star$ together with $A_\psi(\psi)$. Hence
\begin{equation}
\boxed{~ g_5 \;=\; \frac{\theta_H^\star}{\mathcal I_\psi^\star}\,,\qquad \mathcal I_\psi^\star=\int_0^{L_\psi^\star}\!A_\psi(\psi)\,d\psi ~.}
\end{equation}

\section*{Final relation}
Combining the pieces,
\begin{tcolorbox}[colback=white,colframe=black,title={\normalsize Noether $\Rightarrow$ $\alpha$ (UBT-only)}]
\begin{equation*}
\boxed{\quad
\alpha(\mu_0) \;=\; \frac{1}{4\pi}\,\frac{\theta_H^{\star 2}}{(\mathcal I_\psi^\star)^2\,Z^\star}\,.
\quad}
\end{equation*}
\end{tcolorbox}
All quantities are determined by the \emph{same} UBT vacuum $(\Theta, A_\psi, A,B)$---no external field list or tunable input.

\section*{Notes and special cases}
\begin{itemize}
\item \textbf{Flat zero-mode:} $A=B=0$, $\xi_0=\text{const}\Rightarrow Z^\star=L_\psi^\star$.
\item \textbf{Mild warp:} $A(\psi)=\varepsilon\cos(2\pi\psi/L)$ gives $Z^\star=L\,I_0(2\varepsilon)$ (modified Bessel), a purely geometric factor.
\item \textbf{Constant $A_\psi$:} $A_\psi=A_0\Rightarrow \mathcal I_\psi^\star=A_0\,L_\psi^\star$.
\item \textbf{Discreteness:} $\theta_H^\star\in 2\pi\mathbb Z$ (large gauge). Nontrivial vacua often select $\theta_H^\star=\pi$.
\end{itemize}

\section*{How UBT fixes the inputs (no tuning)}
\begin{enumerate}
\item Solve vacuum EOM $\Rightarrow$ $(L_\psi^\star,\theta_H^\star,A(\psi),B(\psi),A_\psi(\psi))$.
\item Compute $Z^\star=\int e^{B-2A}|\xi_0|^2 d\psi$ from the Maxwell zero-mode equation on that background.
\item Compute $\mathcal I_\psi^\star=\int_0^{L_\psi^\star}A_\psi\,d\psi$ from the same vacuum $A_\psi(\psi)$.
\item Insert into the boxed formula to get $\alpha$.
\end{enumerate}

\paragraph{Comment.} Pokud má UBT fraktální strukturu ve směru $\psi$, promítne se \emph{jen} do $Z^\star$ a/nebo $\mathcal I_\psi^\star$. Rovnice pro $\alpha$ zůstává stejná; fraktálnost nahradí konstantní profil vhodnou efektivní mírou v integrálech výše.
\end{document}
