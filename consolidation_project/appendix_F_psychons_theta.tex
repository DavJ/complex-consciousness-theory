%% ================================================
%% SPECULATIVE / WIP — Not part of CORE claims.
%% ================================================
\appendix
\section{Appendix F: Psychons and the Theta Field}
% removed addcontentsline
\subsection*{F.1 Scope and Context}
This appendix consolidates the theoretical foundation of \emph{psychons}---elementary excitations of the phase coordinate $\psi$ in complex time $\tau=t+i\psi$---and their coupling to the biquaternionic master field $\Theta(q,\tau)$. It merges and streamlines results from prior notes on psychons, the $\psi$-resonator concept, and the consciousness-model solutions (stationary analysis, time dynamics, and a decision-making toy model).

\subsection*{F.2 Complex Time, Phase Coordinate, and Fields}
We recall the complex-time embedding,
\begin{equation}
\tau = t + i\psi,\qquad \partial_\tau = \partial_t - i\,\partial_\psi,
\end{equation}
and the biquaternionic field $\Theta(q,\tau)\in\mathbb{B}$ decomposed into scalar, vector, and spinorial components. The \emph{theta field} refers to the $\psi$-phase sector and its couplings to physical fields. Psychons are quanta of $\psi$-phase fluctuations, described at low energies by a scalar (or spinor) field $\chi(x,\psi)$.

\subsection*{F.3 Minimal Field-Theoretic Model}
A minimal Lagrangian capturing the $\psi$-sector and its coupling to electromagnetism (QED limit) and matter is
\begin{align}
\mathcal{L}_{\Theta\text{-}\psi} &= \bar{\Psi}(i\gamma^\mu D_\mu - m)\Psi - \frac{1}{4}F_{\mu\nu}F^{\mu\nu}
+ \frac{1}{2}(\partial_\mu \chi)(\partial^\mu \chi) + \frac{1}{2}(\partial_\psi \chi)^2 - U(\chi) \nonumber\\
&\quad + \frac{1}{2}\eta_\psi\,(\partial_\psi \Psi)^\dagger(\partial_\psi \Psi)
+ \frac{1}{2}\kappa_\psi\,(\partial_\psi A_\mu)(\partial_\psi A^\mu)
- g_{\chi A}\,\chi\,F_{\mu\nu}F^{\mu\nu}
- g_{\chi \tilde{A}}\,\chi\,F_{\mu\nu}\tilde{F}^{\mu\nu}
- g_{\chi \psi}\,\chi\,\bar{\Psi}\Psi.
\end{align}
Setting $\partial_\psi(\cdot)=0$ and all $\psi$-couplings to zero recovers standard QED/QFT. The $\psi$-derivative terms encode slow modulation along the phase axis; $U(\chi)$ may be a double-well or periodic potential.

\paragraph{Equations of Motion.} Variation gives
\begin{align}
\partial_\mu F^{\mu\nu} + 2g_{\chi A}\,\partial_\mu(\chi F^{\mu\nu}) + 2g_{\chi \tilde{A}}\,\partial_\mu(\chi \tilde{F}^{\mu\nu}) &= q\,\bar{\Psi}\gamma^\nu\Psi + \kappa_\psi\,\partial_\psi^2 A^\nu,\\
(i\gamma^\mu D_\mu - m)\Psi - g_{\chi \psi}\,\chi\,\Psi - i\,\eta_\psi\,\partial_\psi^2\Psi &= 0,\\
\Box \chi + \partial_\psi^2 \chi + U'(\chi) &= g_{\chi A}\,F_{\mu\nu}F^{\mu\nu} + g_{\chi \tilde{A}}\,F_{\mu\nu}\tilde{F}^{\mu\nu} + g_{\chi \psi}\,\bar{\Psi}\Psi.
\end{align}

\subsection*{F.4 Stationary Analysis in the $\psi$-Sector}
For time-independent backgrounds, set $\partial_t(\cdot)=0$. Linearizing around a homogeneous solution $\chi=\chi_0+\delta\chi$ with $U'(\chi_0)=0$ and $\partial_\psi \chi_0=0$ yields
\begin{equation}
\left(-\nabla^2 + m_\chi^2\right)\delta\chi + \partial_\psi^2 \delta\chi = J_\chi(x,\psi),
\qquad m_\chi^2 \equiv U''(\chi_0),
\end{equation}
with source $J_\chi$ induced by $F^2,\tilde F F$, and $\bar{\Psi}\Psi$. On compact $\psi$ (period $2\pi$), expand $\delta\chi=\sum_{n\in\mathbb{Z}}\delta\chi_n(x)e^{in\psi}$, giving shifted masses $m_{\chi,n}^2=m_\chi^2+n^2$. This leads to discrete sideband structure in observables.

\subsection*{F.5 Time Dynamics and FP/Theta Link}
At mesoscopic scales, phase diffusion in $\psi$ is effectively stochastic. For a probability amplitude $\mathcal{P}(\psi;x,t)$,
\begin{equation}
\frac{\partial \mathcal{P}}{\partial t} = -\frac{\partial}{\partial \psi}\!\big[a(\psi;x)\,\mathcal{P}\big] + \frac{\partial^2}{\partial \psi^2}\!\big[D(\psi;x)\,\mathcal{P}\big],
\end{equation}
whose solutions on the circle organize into Jacobi theta functions. Identifying $i\pi\tau = -\int^t D\,dt' + \tfrac{i}{2}\!\int^t a\,dt'$, we obtain $\mathcal{P}(\psi;t)\propto \theta_3(\psi/2,\tau)$, with modular properties relevant to periodic/torus geometries.

\subsection*{F.6 Toy Model of Decision Dynamics}
A minimal double-well potential $U(\chi)=\tfrac{\lambda}{4}(\chi^2-\chi_0^2)^2$ coupled to a slow control field $h(t)$ produces bistable dynamics with Kramers-like switching rates. The $\psi$-diffusion renormalizes the barrier height:
\begin{equation}
\Gamma_{\mathrm{eff}} \sim \Gamma_0 \exp\!\left[-\frac{\Delta U_{\mathrm{eff}}}{D_\psi}\right],\qquad \Delta U_{\mathrm{eff}}=\Delta U - \delta\Delta U(\partial_\psi, g_{\chi\cdot}).
\end{equation}
This captures decision-like transitions as stochastic resonance in the $\psi$-sector, consistent with observed latency distributions.

\subsection*{F.7 Coupling to the $\psi$-Resonator (Concept)}
In a toroidal resonator (see Appendix E), the EM mode $A_\mu$ couples to $\delta\chi$ through $F^2$ and $\tilde F F$ terms. With a phase drive $\varphi(t)=\varphi_0\sin(2\pi f_m t)$, sidebands at $f_n\pm k f_m$ gain additional $\psi$-dependent weight. A lock-in protocol can extract small coherence peaks correlated with controlled $\psi$-epochs.

\subsection*{F.8 Observables and Scaling Estimates}
\begin{itemize}
\item \textbf{Phase shifts} in interferometric channels: $\Delta\phi \sim \kappa_\psi L\, \langle \partial_\psi^2 A_\parallel\rangle$.
\item \textbf{Sideband asymmetries}: $S_{+k}-S_{-k} \propto \langle a(\psi)\rangle$ for nonzero drift.
\item \textbf{Effective-mass shifts} in fermionic spectra: $\delta m \sim g_{\chi\psi}\langle \chi\rangle$ in controlled environments.
\end{itemize}
Typical magnitudes are $\mathcal{O}(10^{-6}\text{--}10^{-4})$ relative to EM-only predictions under strong-$Q$ and low-noise conditions.

\subsection*{F.9 Consistency and Limits}
All predictions reduce to standard QED/QFT in the limit $\partial_\psi(\cdot)=0$ and vanishing couplings $g_{\chi\cdot}\rightarrow 0$. Energy conditions and causality remain intact for small $\psi$-sector corrections; stability requires $U''(\chi_0)>0$ and $\eta_\psi,\kappa_\psi\ge 0$.

\subsection*{F.10 Summary}
Psychons provide a controlled, symmetry-preserving extension of standard quantum fields via the complex-time phase coordinate $\psi$. Their couplings generate small, structured deviations testable with high-$Q$ toroidal resonators. The theta-based organization of $\psi$-dynamics yields modular selection rules and discrete sidebands, offering clear targets for experimental validation.


\subsection*{F.11 Related Work / Notes (Cross-Appendix Links)}
This appendix should be read alongside:
\begin{itemize}
\item \textbf{Appendix D (QED in UBT)} for the embedding of $U(1)$ and the biquaternionic formulation of gauge fields.
\item \textbf{Appendix E (Theta Resonator)} for the experimental architecture and protocols targeting $\psi$-dependent observables.
\end{itemize}
The present $\psi$-sector model reduces to standard QED/QFT in the limit $\partial_\psi(\cdot)=0$ and vanishing couplings $g_{\chi\cdot}\!\to\!0$.

\subsection*{F.12 Compact Proof Sketch: FP $\Rightarrow$ Theta Spectrum on $S^1$}
Consider the drift--diffusion (Fokker--Planck) equation on the circle $\phi\in[0,2\pi)$:
\begin{equation}
\frac{\partial P}{\partial t} = -a\,\frac{\partial P}{\partial \phi} + D\,\frac{\partial^2 P}{\partial \phi^2}, \qquad P(\phi+2\pi,t)=P(\phi,t).
\end{equation}
Fourier expand $P(\phi,t)=\sum_{n\in\mathbb{Z}} c_n(t) e^{i n \phi}$ to obtain $\dot{c}_n = (-D n^2 + i a n)\, c_n$, hence
\begin{equation}
c_n(t)=c_n(0)\,\exp\!\big(-D n^2 t + i a n t\big).
\end{equation}
With $z=\phi/2$ and $i\pi\tau=-Dt+\tfrac{i}{2} a t$, the solution organizes into a Jacobi theta series
\begin{equation}
P(\phi,t) \;=\; \sum_{n\in\mathbb{Z}} c_n(0)\, e^{i\pi n^2 \tau}\, e^{i n \phi} \;\propto\; \theta_3(z,\tau),
\end{equation}
establishing the $\theta$-basis as the natural spectral representation on compact phase spaces. (Full derivation and modular properties are provided in Appendix E.)

\subsection*{F.13 Schematic: Psychon Coupling to Quantum Matter Field in $(t,i\psi)$}
\begin{figure}[h!]
\centering
\tdplotsetmaincoords{70}{130}
\begin{tikzpicture}[tdplot_main_coords,scale=1.0, line cap=round,line join=round]
  % Axes for complex time
  \draw[->,thick] (-0.5,0,0) -- (5.5,0,0) node[anchor=west] {$t$};
  \draw[->,thick] (0,-0.5,0) -- (0,4.0,0) node[anchor=south] {$i\psi$};

  % Toroidal scaffold (wireframe hint)
  \def\R{2.0}
  \def\r{0.6}
  \foreach \t in {0,15,...,345}{
    \draw[blue!35!black!70, line width=0.25pt, opacity=0.45]
      plot[domain=0:360, samples=60, variable=\u]
      ({(\R+\r*cos(\u))*cos(\t)},
       {(\R+\r*cos(\u))*sin(\t)},
       {0.0});
  }

  % Matter field amplitude (wave along t)
  \draw[very thick, purple!80!black, domain=0:5.2, samples=200, variable=\x]
    plot ({\x}, {0.7 + 0.35*sin(720*\x)}, {0});

  \node[anchor=west, purple!80!black] at (5.5,0.7,0) {$\psi_m(q,t)$};

  % Psychon excitations (glowing points along i psi)
  \foreach \y in {0.6,1.3,2.0,2.8,3.5}{
    \shade[ball color=orange!85!black, opacity=0.95] (2.4,\y,0) circle (1.7pt);
  }
  \node[anchor=west, orange!80!black] at (2.55,3.6,0) {psychons $\ \chi$};

  % Phase factor coupling
  \draw[->, thick, teal!70!black] (2.4,0.7,0) -- (2.4,3.5,0) node[midway, left, scale=0.9, rotate=90] {$e^{\,i\phi_\psi(q,\psi)}$};

  % Annotate combined matter field with phase
  \node[anchor=west, scale=0.95] at (0.2,3.8,0) {$\displaystyle \Psi_m(q,\tau)=\psi_m(q,t)\, e^{\,i\phi_\psi(q,\psi)}$};

  % Interaction arrow to toroidal scaffold
  \draw[->, thick, red!70!black] (2.4,2.6,0) -- (1.0,0.0,0) node[midway, above, scale=0.85] {$F^2,\ \tilde F F$ couplings};

\end{tikzpicture}
\caption{Conceptual 3D schematic of psychon--matter coupling in the $(t,i\psi)$ plane and its link to a toroidal EM scaffold (Appendix E).}
\label{fig:psychon_schematic}
\end{figure}

\subsection*{F.14 Note on Interpretation}
A compact representation for a matter field with psychon phase coupling is
\begin{equation}
\Psi_m(q,\tau) \;=\; \psi_m(q,t)\, e^{\,i\phi_\psi(q,\psi)},
\end{equation}
which directly yields FP-driven $\theta$-spectra on compact $\psi$ and small but structured deviations from standard QED/QFT observables. For an extended discussion of experimental tests and thresholds, see Appendix E. Informally, one might say that \emph{theta} acts as a unifying organizing principle across both physical and phase-sector dynamics.

