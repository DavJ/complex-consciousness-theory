
\appendix
\section*{Appendix K: Fundamental constants consolidated}

\subsection*{K.1 Existing content}
[Existing material preserved here...]

\subsection*{K.2 Existing content}
[Existing material preserved here...]

\subsection*{K.3 Electron mass as an internal mode}

Building on earlier self–energy models (Refs.~[...]), we show that the electron 
mass arises as the lowest non-trivial eigenvalue of the Dirac operator on 
$T^2(\tau)$ with Hosotani phase $\theta_H=\pi$. 

\paragraph{Eigenvalue problem.}
The internal Dirac operator on the torus reads
\begin{equation}
D = i\gamma^\psi \left(\partial_\psi + i Q \theta_H/L_\psi\right) 
  + i\gamma^\phi \partial_\phi ,
\end{equation}
whose eigenmodes are labelled by $(n,m)\in\mathbb{Z}^2$ and acquire shifts 
$(n+\delta, m+\delta')/R$ due to the Hosotani background. 
The corresponding eigenenergies are
\begin{equation}
E_{n,m} = \frac{1}{R}\sqrt{(n+\delta)^2 + (m+\delta')^2}.
\end{equation}

\paragraph{Electron as the first excitation.}
For $Q=-1$ and $\theta_H=\pi$, the lowest non-zero mode $(n,m)=(0,1)$ yields 
\begin{equation}
m_e \;=\; \frac{1}{R}\sqrt{\delta^2+1}\;\simeq\;\frac{1}{R},
\end{equation}
where $R$ is tied to the compactification scale fixed in Appendix~V. 

\paragraph{Implication.}
Thus $m_e$ is not an input parameter but an output of the same geometry that 
fixes $\alpha$. Higher modes $(n,m)=(0,2)$ and $(1,0)$ naturally provide 
candidates for $m_\mu$ and $m_\tau$, suggesting that lepton mass ratios arise 
from the same internal structure.

\subsection*{K.4 Consolidation}
Together with Appendix~V, this establishes that both $\alpha$ and $m_e$ (and 
potentially higher lepton masses) emerge from the same toroidal geometry.
