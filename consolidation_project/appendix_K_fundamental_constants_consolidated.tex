
\appendix
\section*{Appendix K: Consolidation of Fundamental Constants}

\subsection*{K.1 Input vs. Output Constants}
Within the Unified Biquaternion Theory (UBT), we distinguish between input parameters that define the scale and structure, and output constants that the theory predicts. 
The goal is to minimize the number of true inputs.

\textbf{Inputs:} $c, \hbar$ (definitions of units), integer parameters (e.g. $n_\mu, n_\tau$ in early exploratory models).  
\textbf{Outputs:} $\alpha$, $m_e$, $m_\mu$, $m_\tau$, $\Lambda_{\text{QCD}}$, $G$, etc.

\paragraph{Reference values.}  
The following table summarizes empirical reference values relevant to UBT (as used for benchmarking the results in this work):

\begin{center}
\begin{tabular}{lll}
\hline
Constant & Value & Provenance \\
\hline
$\alpha^{-1}(M_Z)$ & $127.955 \pm 0.010$ & (EW fit, 5-flavor) \\
$m_e$ & $0.51099895000(15)$ MeV & CODATA \\
$m_\mu$ & $105.6583755(23)$ MeV & PDG \\
$m_\tau$ & $1776.86(12)$ MeV & PDG \\
$\Lambda_{\text{QCD}}^{(5)}$ & $\sim 0.21 \pm 0.01$ GeV & PDG \\
$G$ & $6.67430(15)\times 10^{-11}$ m$^3$ kg$^{-1}$ s$^{-2}$ & CODATA \\
\hline
\end{tabular}
\end{center}

These numbers serve only as \emph{external} benchmarks; in UBT, $\alpha$ and lepton masses are \emph{outputs} of the same internal geometry.

\subsection*{K.2 The Puzzle of Lepton Mass Ratios (Historical Context)}
Experimentally observed ratios of charged lepton masses are
\begin{align*}
\frac{m_\mu}{m_e} &\simeq 206.77, \qquad
\frac{m_\tau}{m_e} &\simeq 3477.2.
\end{align*}
The near-integer values ($207$ and $3477$) were long viewed as a numerological curiosity. 
Early drafts used these integers as a clue; the present work replaces this with a \emph{spectral} mechanism (Appendix~W).

\subsection*{K.3 Electron Mass as an Internal Eigenmode (New)}
Using the internal toroidal structure, the electron mass arises as the lowest non-trivial eigenvalue of the Dirac operator on $T^2(\tau)$ with Hosotani background $\theta_H=\pi$.

\paragraph{Eigenvalue problem.}
The internal Dirac operator on the torus is
\begin{equation}
D \;=\; i\gamma^\psi \!\left(\partial_\psi + i Q\, \theta_H/L_\psi\right) \;+\; i\gamma^\phi \partial_\phi,
\end{equation}
with eigenmodes $(n,m)\in\mathbb{Z}^2$ shifted by the Hosotani background. The eigen-energies are
\begin{equation}
E_{n,m} \;=\; \frac{1}{R}\sqrt{(n+\delta)^2 + (m+\delta')^2}.
\end{equation}

\paragraph{Electron as the first excitation.}
For $Q=-1$ and $\theta_H=\pi$, the lowest nonzero mode $(n,m)=(0,1)$ gives
\begin{equation}
m_e \;=\; E_{0,1} \;=\; \frac{1}{R}\sqrt{\delta^2+1} \;\simeq\; \frac{1}{R},
\end{equation}
where $R$ is tied to the compactification scale fixed in Appendix~V. 
Higher modes such as $(0,2)$ and $(1,0)$ naturally provide candidates for $m_\mu$ and $m_\tau$ (see Appendix~W).

\subsection*{K.4 Consolidation with $\alpha$ (New)}
Together with Appendix~V, this shows that both $\alpha$ and $m_e$ (and potentially the heavier lepton masses) emerge from the same toroidal geometry.  
This consolidation eliminates the apparent ``two-scale problem'' by linking the determination of $\alpha$ at $M_Z$ with the prediction of $m_e$ as the first eigenmode.

\subsection*{K.5 Other Fundamental Constants (Restored Notes)}
\paragraph{QCD scale $\Lambda_{\text{QCD}}$.}
The running of the strong coupling introduces a scale $\Lambda_{\text{QCD}}$. Its empirical magnitude is consistent with the compactification radius $R$ inferred from $\alpha$ and $m_e$, suggesting a deeper link to be developed.

\paragraph{Weinberg angle $\theta_W$.}
Preliminary considerations indicate that the electroweak mixing angle may be geometrically related to phases/holonomies in the internal torus. A full derivation is left for future work.

\paragraph{Gravitational constant $G$.}
Earlier sketches suggest $G$ may be emergent from large-scale averaging of the biquaternionic field $\Theta$ with appropriate normalization. Although incomplete, this motivates a program toward gravity as an effective coupling of the same underlying geometry.
