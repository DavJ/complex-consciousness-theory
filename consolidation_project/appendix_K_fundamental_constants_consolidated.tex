%% ----------------------------------------------
%% SPECULATIVE / WIP: The following material is exploratory and not part of CORE claims.
%% ----------------------------------------------
\section{Appendix K: suggest of Fundamental Constants (Beyond $\alpha$)}
\label{app:fundamental-constants}

This appendix consolidates UBT-propose constraints and predictions for fundamental constants beyond the fine-structure constant $\alpha$ (treated in Appendix~\ref{app:alpha-consolidated}). We distinguish \emph{inputs} (assumptions/fixed scales) from \emph{outputs} (predictions) and cross-check against CODATA/PDG values.

\subsection{Methodological outline}
\begin{enumerate}
\item Identify UBT relations linking constants to topological/geometry scales (internal-mode frequency, holonomy integers, curvature radii).
\item Separate \emph{dimensionful} vs.\ \emph{dimensionless} constants; fix only units where needed.
\item Propagate quantum corrections (1–2 loop) consistent with Appendix~\ref{app:alpha-consolidated} and SM/QCD running (Appendix~\ref{app:sm-qcd-ubt}).
\end{enumerate}

\subsection{Speed of light $c$ and Planck's constant $\hbar$}
UBT treats $c$ and $\hbar$ as unit-defining invariants; no dynamics is assigned. We keep them as \emph{inputs} (SI-defining constants).

\subsection{Newton's constant $G$}
\paragraph{UBT relation.} $G$ enters through the biquaternionic curvature scale $R_{\rm grav}$ (Appendix~A). We will express $G$ in terms of a UBT length $L_{\rm UBT}$ and internal-mode scale $\mu_{\rm int}$:
\begin{equation}
G \;\sim\; \frac{c^3}{\mu_{\rm int}^2}\,\Xi_{\rm grav}(L_{\rm UBT}) \quad\text{(model-dependent factor $\Xi_{\rm grav}$ to be fixed from gravitational sector fits).}
\end{equation}

\subsection{Lepton masses $m_e,m_\mu,m_\tau$}
\paragraph{UBT pipeline.} Internal-mode spectrum of the $\Theta$-sector (Appendix~D) yields
\begin{equation}
m_\ell \;=\; \hbar\, \omega_{\ell}(\mu_{\rm int},\text{holonomy},\text{boundary})\,[1+\Delta_{\rm loop}],
\end{equation}
with $\Delta_{\rm loop}$ computed in the same renormalization scheme as for $\alpha$.
We will import electron-mode details from Appendix~D and extend to $(\mu,\tau)$ by mode indexing.

\subsection{QCD scale $\Lambda_{\rm QCD}$}
\paragraph{Matching to UBT.}
\begin{equation}
\Lambda_{\rm QCD} \;\simeq\; \xi\, \mu_{\rm int}\, \exp\!\Big(-\frac{2\pi}{\beta_0\,\alpha_s(\mu_{\rm int})}\Big)\,,
\end{equation}
with $\beta_0$ as in Appendix~\ref{app:sm-qcd-ubt} and $\xi=\mathcal{O}(1)$ encoding normalization. This ties the strong scale to the internal-mode physics.

\subsection{Electroweak angle $\theta_W$ and gauge couplings}
Given $e=g\sin\theta_W=g'\cos\theta_W$ and $e$ fixed by Appendix~\ref{app:alpha-consolidated}, one needs either a unification condition or an extra UBT constraint to determine $\theta_W$; otherwise it remains an input parameter.

\subsection{Summary table (placeholders)}
\begin{center}
\begin{tabular}{lccc}
\hline
Constant & UBT Status & suggest/Input & Comparison (CODATA/PDG)\\
\hline
$c$ & unit & input & (exact)\\
$\hbar$ & unit & input & (exact)\\
$G$ & model-dependent & TBD & value $\pm$ unc.\\
$m_e$ & from Appendix D & TBD & PDG\\
$m_\mu$ & internal mode & TBD & PDG\\
$m_\tau$ & internal mode & TBD & PDG\\
$\Lambda_{\rm QCD}$ & running/matching & TBD & PDG\\
$\theta_W$ & needs constraint & TBD & PDG\\
\hline
\end{tabular}
\end{center}
