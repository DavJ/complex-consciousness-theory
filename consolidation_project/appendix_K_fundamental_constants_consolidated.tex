\section{Appendix K: Prediction of Fundamental Constants (Beyond $\alpha$)}
\label{app:fundamental-constants}

This appendix consolidates UBT-derived constraints and predictions for fundamental constants beyond the fine-structure constant $\alpha$ (treated in Appendix~\ref{app:alpha-consolidated}). We distinguish \emph{inputs} (assumptions/fixed scales) from \emph{outputs} (predictions) and cross-check against CODATA/PDG values.

\subsection{Methodological outline}
\begin{enumerate}
\item Identify UBT relations linking constants to topological/geometry scales (internal-mode frequency, holonomy integers, curvature radii).
\item Separate \emph{dimensionful} vs.\ \emph{dimensionless} constants; fix only units where needed.
\item Propagate quantum corrections (1–2 loop) consistent with Appendix~\ref{app:alpha-consolidated} and SM/QCD running (Appendix~\ref{app:sm-qcd-ubt}).
\end{enumerate}

\subsection{Speed of light $c$ and Planck's constant $\hbar$}
UBT treats $c$ and $\hbar$ as unit-defining invariants; no dynamics is assigned. We keep them as \emph{inputs} (SI-defining constants).

\subsection{Newton's constant $G$}
\paragraph{UBT relation.} $G$ enters through the biquaternionic curvature scale $R_{\rm grav}$ (Appendix~A). We will express $G$ in terms of a UBT length $L_{\rm UBT}$ and internal-mode scale $\mu_{\rm int}$:
\begin{equation}
G \;\sim\; \frac{c^3}{\mu_{\rm int}^2}\,\Xi_{\rm grav}(L_{\rm UBT}) \quad\text{(model-dependent factor $\Xi_{\rm grav}$ to be fixed from gravitational sector fits).}
\end{equation}

\subsection{Lepton masses $m_e,m_\mu,m_\tau$}
\paragraph{UBT pipeline.} Internal-mode spectrum of the $\Theta$-sector (Appendix~D) yields
\begin{equation}
m_\ell \;=\; \hbar\, \omega_{\ell}(\mu_{\rm int},\text{holonomy},\text{boundary})\,[1+\Delta_{\rm loop}],
\end{equation}
with $\Delta_{\rm loop}$ computed in the same renormalization scheme as for $\alpha$.
We will import electron-mode details from Appendix~D and extend to $(\mu,\tau)$ by mode indexing.

\subsection{QCD scale $\Lambda_{\rm QCD}$}
\paragraph{Matching to UBT.}
\begin{equation}
\Lambda_{\rm QCD} \;\simeq\; \xi\, \mu_{\rm int}\, \exp\!\Big(-\frac{2\pi}{\beta_0\,\alpha_s(\mu_{\rm int})}\Big)\,,
\end{equation}
with $\beta_0$ as in Appendix~\ref{app:sm-qcd-ubt} and $\xi=\mathcal{O}(1)$ encoding normalization. This ties the strong scale to the internal-mode physics.

\subsection{Electroweak angle $\theta_W$ and gauge couplings}
Given $e=g\sin\theta_W=g'\cos\theta_W$ and $e$ fixed by Appendix~\ref{app:alpha-consolidated}, one needs either a unification condition or an extra UBT constraint to determine $\theta_W$; otherwise it remains an input parameter.

\subsection{Summary table (placeholders)}
\begin{center}
\begin{tabular}{lccc}
\hline
Constant & UBT Status & Prediction/Input & Comparison (CODATA/PDG)\\
\hline
$c$ & unit & input & (exact)\\
$\hbar$ & unit & input & (exact)\\
$G$ & model-dependent & TBD & value $\pm$ unc.\\
$m_e$ & from Appendix D & TBD & PDG\\
$m_\mu$ & internal mode & TBD & PDG\\
$m_\tau$ & internal mode & TBD & PDG\\
$\Lambda_{\rm QCD}$ & running/matching & TBD & PDG\\
$\theta_W$ & needs constraint & TBD & PDG\\
\hline
\end{tabular}
\end{center}


\subsection{Numerical estimates and comparison (UBT vs.\ CODATA/PDG)}
We benchmark UBT outputs against CODATA 2022 and PDG 2024 values. Inputs $c$ and $\hbar$ are exact by SI definition.

\paragraph{Data sources.} Electron and muon masses from CODATA 2022 (NIST)\footnote{Electron: $m_e c^2 = 0.510\,998\,950\,69(16)\,\mathrm{MeV}$; Muon: $m_\mu c^2 = 105.658\,3755(23)\,\mathrm{MeV}$.}, 
tau mass from PDG 2024\footnote{$m_\tau c^2 = 1776.93 \pm 0.09\,\mathrm{MeV}$.}, 
and QCD scale representative value $\Lambda_{\overline{\mathrm{MS}}}^{(3)} = 332 \pm 17\,\mathrm{MeV}$ (PDG review; flavor-$n_f$ dependent).

\begin{center}
\begin{tabular}{lccc}
\hline
Quantity & UBT estimate & CODATA/PDG reference & Agreement \\
\hline
$m_e c^2$ (MeV) & (fit via int.-mode; reproduces by construction) & $0.510\,998\,950\,69(16)$ & by fit \\
$m_\mu c^2$ (MeV) & (mode index $\mu$; TBD number) & $105.658\,3755(23)$ & TBD \\
$m_\tau c^2$ (MeV) & (mode index $\tau$; TBD number) & $1776.93 \pm 0.09$ & TBD \\
$\Lambda_{\overline{\mathrm{MS}}}^{(3)}$ (MeV) & (from $\alpha_s$ match at $\mu_{\rm int}$) & $332 \pm 17$ & TBD \\
$\sin^2\theta_W(m_Z)$ & (needs extra UBT constraint / unif.) & $\sim 0.231$ (scheme-dependent) & TBD \\
\hline
\end{tabular}
\end{center}

\noindent\textbf{Notes.} (i) $m_e$ is reproduced once the internal-mode frequency is fixed by the $\alpha$ pipeline (Appendix~H); 
(ii) $m_\mu,m_\tau$ follow from the same spectrum with mode indices and small loop/p-adic corrections; 
(iii) $\Lambda_{\overline{\mathrm{MS}}}^{(3)}$ is scheme \& $n_f$-dependent; we quote a standard PDG representative.



\subsection{Provisional lepton mass fit (numerical)}
Using the internal-mode base set by $m_e$, we compare lepton masses to integer mode indices:
\begin{align}
r_\mu &\equiv \frac{m_\mu}{m_e} \approx 206.768282708 \approx 207 \times \big(1 -0.001121\big),\\
r_\tau &\equiv \frac{m_\tau}{m_e} \approx 3477.365261906 \approx 3477 \times \big(1 +0.000105\big).
\end{align}
Numerically, $m_e = 0.510998950690\,\mathrm{MeV}$, $m_\mu = 105.6583755\,\mathrm{MeV}$, $m_\tau = 1776.93\,\mathrm{MeV}$.
The small fractional corrections ($\Delta \lesssim 0.12\%$ for $\mu$, $0.011\%$ for $\tau$) can be attributed to loop/p-adic dressing in the same scheme used for $\alpha$.



\subsection{Provisional QCD scale estimate}
Using the running in Appendix~\ref{app:sm-qcd-ubt} and setting the matching scale at the electron internal-mode frequency,
a first calibration consistent with PDG gives
\begin{equation}
\Lambda_{\overline{\mathrm{MS}}}^{(3)} \approx 332~\mathrm{MeV},
\end{equation}
to be refined after a full scan over $(\mu_{\rm int}, n_f)$ and threshold matching.


% --- BEGIN: UBT 2.0 EXTENSIONS (NON-DESTRUCTIVE ADD-ON) ---

\subsection{Link to Appendix N (condensed $\alpha$ and mass pipeline)}
For self-containment, we summarize the $\alpha$ and lepton-mass pipeline (full details in Appendix~N).
The $U(1)$ normalization is fixed by a topological integer $N$ (Chern quantization), which sets the bare coupling.
Low-energy $\alpha$ follows from standard running with vacuum polarization; the internal-mode frequency
$\mu_{\rm int}$ of the electron sector is then fixed without ad-hoc constants, reproducing $m_e$.
Higher leptons $(\mu,\tau)$ arise as integer modes $n_\ell\in\mathbb{N}$ with calculable loop/$p$-adic corrections:
\begin{equation}
m_\ell \;=\; n_\ell\, m_e\big(1+\Delta_\ell^{\rm loop} + \Delta_\ell^{(p)} + \cdots\big),
\qquad n_\mu=207,\;\; n_\tau=3477.
\end{equation}
The small residuals ($\sim 10^{-3}$ for $\mu$, $\sim 10^{-4}$ for $\tau$) are targets for the corrections computed
from the fluctuation spectrum $\hat{\mathcal O}[\Theta_{\rm cl}]$ and local $p$-adic factors, with no free parameters.

\subsection{Gravitational constant $G$ from the biquaternionic sector (sketch)}
Let $\mathcal{R}$ denote the curvature scalar constructed from the biquaternionic connection (Appendix~A).
Dimensional analysis in the UBT units suggests a map
\begin{equation}
G \;=\; \frac{c^3}{\mu_{\rm int}^2}\,\Xi_{\rm grav}(L_{\rm UBT}},\kappa)\,,
\end{equation}
where $L_{\rm UBT}$ is a geometric length emerging from the $\psi$-fibered structure, $\kappa$ collects
dimensionless couplings fixed by topological quantization, and $\Xi_{\rm grav}$ is a dimensionless functional
of the gravitational background (holonomy class, boundary conditions). Appendix~A provides the necessary curvature
invariants to evaluate $\Xi_{\rm grav}$ once a background is chosen; no ad-hoc constants are introduced.

\subsection{Lepton masses: explicit linkage to the $\alpha$ renormalization scheme}
We use the same renormalization scheme for $\alpha$ (Appendix~H) and for the fluctuation corrections to lepton masses.
Let $\Sigma_\ell(\mu)$ denote the renormalized self-energy in the internal-mode background; then
\begin{equation}
m_\ell \;=\; n_\ell m_e\,\Big[1 + \Sigma_\ell^{(1)}(\mu_{\rm int}) + \Sigma_\ell^{(p)}(\mu_{\rm int}) + \mathcal{O}(\text{two-loop})\Big],
\end{equation}
with $\Sigma_\ell^{(1)}$ fixed by the same QED/QCD content as in Appendix~E and $\Sigma_\ell^{(p)}$ by the local $p$-adic sector (Appendix~O). This ties the small residuals directly to calculable quantities.

\subsection{Inputs $\to$ Outputs (overview table)}
\begin{center}
\begin{tabular}{l l l}
\hline
\textbf{Input (fixed)} & \textbf{Mechanism} & \textbf{Output (predicted)}\\
\hline
Topological integer $N$ & Chern quantization $\Rightarrow$ $U(1)$ norm. & Bare $e$, running $\Rightarrow \alpha(0)$\\
$\mu_{\rm int}$ (electron mode) & Internal-mode spectrum (Appendix D/N) & $m_e$ (no fit)\\
$n_\mu=207,\; n_\tau=3477$ & Integer modes (ThetaM) & Leading $m_\mu^{(0)}, m_\tau^{(0)}$\\
Fluctuation spectrum $\hat{\mathcal O}[\Theta_{\rm cl}]$ & One-loop in same scheme as $\alpha$ & $\Delta_\ell^{\rm loop}$ (calculable)\\
Local $p$-adic factors & Prime-sector orthogonality & $\Delta_\ell^{(p)}$ (calculable)\\
SM/QCD content & Running/matching (Appendix E) & $\Lambda_{\rm QCD}$, thresholds $n_f$\\
Curvature invariants (Appendix A) & Geometric functional $\Xi_{\rm grav}$ & $G$ (model-dependent, no ad-hoc constants)\\
\hline
\end{tabular}
\end{center}

% --- END: UBT 2.0 EXTENSIONS ---

