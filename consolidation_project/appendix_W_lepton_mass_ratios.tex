
\appendix
\section*{Appendix V: Emergent $\alpha$}

\subsection*{V.1 Dynamical fixation of geometry}
In UBT the value of the fine-structure constant $\alpha$ is not postulated 
but arises from the dynamics of the internal space. We assume that the 
extra dimensions are compactified on a torus $T^2$, whose shape (modulus 
$\tau=i y$) and size ($R$) are determined by minimizing the one-loop 
effective potential $V_{\rm eff}$ (Hosotani/Casimir mechanism).

The potential is given by a sum over all Standard Model (SM) particles 
with mass $m_i$ that couple to the internal space:
\begin{equation}
V_{\rm eff}(y) = \sum_{i \in \text{SM}} (-1)^{F_i} d_i 
  \sum_{p,q \in \mathbb{Z}} \frac{m_i^2}{8\pi^2 L^2 y} 
  K_2\!\left(2\pi m_i L \sqrt{p^2 y^2 + (q+\delta_i)^2}\right),
\end{equation}
where $F_i$ is the fermion number, $d_i$ the number of degrees of freedom, 
and $\delta_i$ the shift induced by the charge.

Minimizing this potential at scale $\mu_0\sim m_Z$ dynamically fixes 
a stable value of the torus modulus $y=y_\ast$. The value of $\alpha$ 
is then given by
\begin{equation}
\alpha(M_Z)^{-1} = \frac{4\pi N}{y_\ast},
\end{equation}
where $N=10$ is the normalization factor determined by the sum of squared 
charges of SM particles.

\subsection*{V.2 Cross-reference to electron mass}
As shown in Appendix~K.3, the electron mass can be interpreted as the first 
internal eigenmode on $T^2(\tau)$ with Hosotani background. This directly ties 
the emergent value of $\alpha$ to the lepton mass spectrum, providing a 
non-circular and unified account of both quantities.

\subsection*{V.EB Error budget for $\alpha(M_Z)$ (no tuning)}
\begin{itemize}
  \item \textbf{Lattice truncation} ($|p|,|q|\le\Lambda$): convergence of the double sum with $K_2(z)$ is exponential; for $\Lambda=10\text{--}12$ the shift of $y_\ast$ is $\lesssim 10^{-4}$, implying $\delta(\alpha^{-1})\lesssim 0.01$.
  \item \textbf{Gauge bookkeeping} (vectors vs. Goldstones+ghosts): net DOF vs. explicit treatment agree within $\delta(\alpha^{-1})\sim 0.01$ when done consistently.
  \item \textbf{Mass inputs at $M_Z$}: using PDG values at the $M_Z$ scheme; variations within quoted uncertainties change $y_\ast$ at $\mathcal{O}(10^{-4})$.
  \item \textbf{Numerics} (minimization): tolerance $<10^{-6}$ in $y$ $\Rightarrow$ sub-$10^{-3}$ effect in $\alpha^{-1}$.
\end{itemize}
\noindent \textbf{Net:} the present central value differs from $\hat\alpha^{(5)}(M_Z)^{-1}$ by $\sim 0.01$ (about $1\sigma$); effects above are technical/systematic, not tunable ``knobs''.
