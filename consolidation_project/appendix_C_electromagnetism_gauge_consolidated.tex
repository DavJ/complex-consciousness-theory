
\appendix{C}{Electromagnetism in the Unified Biquaternion Framework}

\section{Overview}
This appendix consolidates the original derivations, solutions, and conceptual notes on electromagnetism within the Unified Biquaternion Theory (UBT). The aim is to present a coherent, non-duplicative treatment, while preserving the original reasoning paths that led to the key results. Links to related appendices on gravitational coupling, $\alpha$-phase effects, and $p$-adic extensions are provided where relevant.

\section{Biquaternion Formulation of Electromagnetism}
We start from the covariant field equation for the electromagnetic potential $A_\mu$ embedded into the biquaternion algebra $\mathbb{B}$. This allows the electric and magnetic fields to be represented as bivector components of a unified field tensor $\mathcal{F}$,
\begin{equation}
    \mathcal{F} = \nabla \wedge A \quad \in \quad \mathbb{B} \otimes \Lambda^2.
\end{equation}
In explicit biquaternion form:
\begin{equation}
    \mathcal{F} = (\mathbf{E} + i\,\mathbf{B}) \cdot \boldsymbol{\sigma},
\end{equation}
where $\boldsymbol{\sigma}$ are the Pauli-like basis elements in $\mathbb{B}$ and $i$ is the scalar imaginary unit commuting with the quaternion units.

\section{Maxwell's Equations in Curved Spacetime}
Using the covariant derivative $\nabla_\mu$ compatible with the UBT metric $g_{\mu\nu}$ derived in Appendix B (Gravitation), Maxwell's equations generalize to:
\begin{align}
    \nabla_\mu \mathcal{F}^{\mu\nu} &= \mu_0 J^\nu, \\
    \nabla_{[\alpha} \mathcal{F}_{\beta\gamma]} &= 0.
\end{align}
This retains gauge invariance and introduces curvature coupling terms, which in the biquaternion formalism appear as commutator terms $[\Gamma, \mathcal{F}]$ in the connection representation.

\section{Original Derivation Notes}
In the early stages of this work, the electromagnetic sector was explored via analogy with the Dirac equation in $\mathbb{B}$. The key insight was that the EM field could be treated as the curvature of a $U(1)$ connection embedded in the right-multiplication sector of $\mathbb{B}$, while the left-multiplication sector described spinorial matter. This decomposition naturally explains charge conjugation symmetry and provides a pathway for coupling to the $\alpha$-phase field (see Appendix F).

It was also noted that the Fokker--Planck-type diffusion of the $\psi$-phase in complex time $\tau = t + i\psi$ could modulate the effective permittivity and permeability of the vacuum. This idea connects to the $p$-adic hierarchical scaling of field strengths (Appendix G).

\section{Wave Solutions}
From the biquaternion Maxwell equations in flat spacetime, one recovers the familiar wave equation:
\begin{equation}
    \Box A_\mu = 0,
\end{equation}
for free fields. In the curved UBT metric, the wave operator becomes the Laplace--Beltrami operator $\Box_g$, leading to redshift and lensing of electromagnetic waves.

Original solution work (see historical ``solutions'' notes) examined toroidal standing wave configurations, relevant for the Theta Resonator experimental proposal. Such solutions are characterized by localized energy densities and quantized circulation numbers $n$, potentially linked to the $\alpha$-quantization of phase space.

\section{Field Invariants and Duality}
Two Lorentz- and gauge-invariant scalars can be formed:
\begin{align}
    \mathcal{I}_1 &= \frac{1}{2} \mathcal{F}_{\mu\nu} \mathcal{F}^{\mu\nu} 
        = |\mathbf{B}|^2 - |\mathbf{E}|^2, \\
    \mathcal{I}_2 &= \frac{1}{2} \mathcal{F}_{\mu\nu} \tilde{\mathcal{F}}^{\mu\nu} 
        = 2\,\mathbf{E} \cdot \mathbf{B}.
\end{align}
In $\mathbb{B}$ representation, duality rotations correspond to multiplication by $e^{i\theta}$ in the scalar imaginary sector, providing a natural geometric interpretation.

\section{Links to Other Appendices}
\begin{itemize}
    \item \textbf{Appendix B}: Gravitational coupling and metric derivations.
    \item \textbf{Appendix F}: $\alpha$-phase modulation of EM fields.
    \item \textbf{Appendix G}: $p$-adic scaling and hierarchical structure of field amplitudes.
    \item \textbf{Appendix H}: Toroidal resonator applications and standing wave quantization.
\end{itemize}

This appendix should be read alongside these sections to obtain a complete picture of electromagnetism in the UBT framework.
