\documentclass{article}
\usepackage{amsmath}
\title{Symbolické odvození jemné struktury}
\begin{document}
\maketitle

\section*{Shrnutí}
Odvozujeme hodnotu jemné struktury $\alpha = \frac{e^2}{\hbar c}$ na základě vztahů mezi parametry bikvaternionového pole $\Theta$ a jeho vnitřními módy.

\section*{Základní vztahy}
Z teorie máme:
\[
\Theta(q, \tau) = \rho(q, \tau) e^{i \phi(q, \tau)}
\]
kde fáze $\phi$ odpovídá vnitřní rotaci (spin) a interakčním efektům. Pole nese vnitřní frekvenci $\omega_\Theta$ analogickou k $mc^2/\hbar$.

\section*{Odvození}
Z dimenzionální analýzy a energetické rovnosti pole a jeho elektromagnetické interakce odvodíme:
\[
\alpha = \frac{e^2}{\hbar c} \approx \frac{\omega_q}{\omega_\Theta}
\]
kde $\omega_q$ je frekvence elektromagnetické vazby (součet fázových posunů napříč topologií toru).

\section*{Závěr}
Výraz pro $\alpha$ je přirozeným důsledkem geometrie a vlastností vnitřních módů pole $\Theta$.

\end{document}
