
\appendix
\section*{Appendix 15: Theoretical Simulations and Model Predictions}
\addcontentsline{toc}{section}{Appendix 15: Theoretical Simulations and Model Predictions}

This appendix presents a selection of theoretical simulations and model predictions based on the Unified Biquaternion Theory (UBT). The goal is to demonstrate the concrete quantitative consequences of the theory in selected scenarios.

\subsection*{1. Complex Diffusion of the \(\Theta(q,\tau)\) Field}

By applying the complex Fokker–Planck equation derived in Appendix A to a localized initial distribution of the form:

\[
\Theta(q, \tau = 0) = \exp\left(-\frac{|q|^2}{2\sigma^2}\right) \cdot e^{i \phi_0},
\]

we numerically simulate the evolution of the probability amplitude and its phase across both real time \(t\) and internal phase time \(\psi\).

Results confirm the emergence of spiral structures in phase space and show a drift of the peak probability density along the imaginary direction, consistent with conscious directional flow.

\subsection*{2. Curvature in Toroidal Coordinate Systems}

A toroidal geometry is introduced with metric coefficients derived in Appendix B. By computing the imaginary part of the curvature tensor \(\mathbf{R}_{\text{imag}}\), we observe resonance peaks corresponding to quantized modes:

\[
\omega_n \sim \frac{n c}{2 \pi R},
\]

where \(R\) is the major radius of the torus and \(n\) is the mode number. This supports the hypothesis of phase-resonant excitations in the \(\Theta\) field.

\subsection*{3. Scalar Field \(G(q)\) Dynamics and Energy Localization}

Simulations of the scalar field \(G(q)\) under the equation:

\[
\Box G - \alpha G + \beta G^3 = \gamma |\partial_\tau \Theta|^2,
\]

demonstrate spontaneous symmetry breaking and localized soliton formation. These solutions are stable in both Minkowski and weakly curved space, suggesting a mechanism for storing conscious phase energy.

\subsection*{4. Composite Structure of the \(\Theta\) Field}

Decomposing \(\Theta\) into spinor and scalar components reveals interference patterns under coupled evolution. Numerical simulations show robust phase locking under weak perturbations and phase decoherence in the presence of topological noise.

\subsection*{5. Gravitational Feedback of Phase Flow}

Including backreaction from \(\mathbf{R}_{\text{imag}}\) into the real part of the metric via:

\[
g_{\mu\nu} \rightarrow g_{\mu\nu} + \lambda \Re[\mathbf{R}_{\text{imag}, \mu\nu}],
\]

leads to measurable curvature distortions in scenarios with high phase drift. These effects remain below current detection thresholds but may become observable with precise interferometry.

\subsection*{Conclusion}

Theoretical simulations based on UBT yield a wide spectrum of novel phenomena, many of which are amenable to numerical study and may guide future experimental setups. These results motivate further investigation into the physical manifestations of phase-encoded information within the biquaternionic manifold.
