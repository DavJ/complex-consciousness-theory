
\appendix

\section*{Appendix 8: Kvadratick\'a akce a Maxwellovy rovnice}

\addcontentsline{toc}{section}{Appendix 8: Kvadratick\'a akce a Maxwellovy rovnice}

\subsection*{1. Roz\v{s}\'{i}\v{r}en\'a akce}

Abychom doplnili chyb\'ej\'ic\'i dynamiku pro imagin\'arn\'i slo\v{z}ku k\v{r}ivostn\'iho tenzoru \( \mathbf{R}_{ab} \), p\v{r}id\'av\'ame k akci kvadratick\'y \v{c}len:

\begin{equation}
S = \text{Re} \left[ \int \text{ScalarPart} \left( e^a \wedge e^b \wedge \mathbf{R}_{ab} \right) \right] + \lambda \cdot \text{Re} \left[ \int \text{ScalarPart} \left( \mathbf{R}_{ab} \wedge \star \mathbf{R}^{ab} \right) \right]
\end{equation}

Kde \( \star \) je Hodgeova dualita, a \( \lambda \) je konstanta s jednotkami takov\'ymi, aby \( S \) bylo bezrozm\v{e}rn\'e.

\subsection*{2. Odd\v{e}len\'i re\'aln\'e a imagin\'arn\'i \v{c}\'asti}

Rozlo\v{z}me \( \mathbf{R}_{ab} = \mathbf{R}^R_{ab} + \mathbf{R}^I_{ab} \) a vyu\v{z}ijme ortogonalitu re\'aln\'e a imagin\'arn\'i \v{c}\'asti:

\begin{equation}
\text{Re} \left[ \text{ScalarPart}( \mathbf{R}_{ab} \wedge \star \mathbf{R}^{ab} ) \right] = \text{ScalarPart}( \mathbf{R}^R_{ab} \wedge \star \mathbf{R}^{R ab} ) + \text{ScalarPart}( \mathbf{R}^I_{ab} \wedge \star \mathbf{R}^{I ab} )
\end{equation}

Prvn\'i \v{c}len je kvadratick\'y v gravita\v{c}n\'i k\v{r}ivosti a m\v{u}\v{z}e p\v{r}isp\v{e}t nap\v{r}. jako Gauss--Bonnetova invariantn\'i oprava. Druh\'y \v{c}len je nov\'y a odpov\'id\'a kvadratick\'emu term\'inu pro elektromagnetick\'e pole:

\begin{equation}
\mathbf{R}^I_{ab} = F_{\mu\nu ab} \cdot \mathbf{i} \quad \Rightarrow \quad \mathbf{R}^I_{ab} \wedge \star \mathbf{R}^{I ab} = (F \wedge \star F) \cdot \mathbf{i}^2 = - F \wedge \star F
\end{equation}

Tak\v{z}e celkov\'y p\v{r}\'ispevek imagin\'arn\'i \v{c}\'asti je:

\begin{equation}
\text{Re} \left[ \text{ScalarPart}( \mathbf{R}^I_{ab} \wedge \star \mathbf{R}^{I ab} ) \right] = - F \wedge \star F
\end{equation}

\subsection*{3. Variace akce a Maxwellovy rovnice}

Provedeme variaci kvadratick\'eho \v{c}lenu podle \( \omega^I \), resp. podle \( A = \omega^I \) (elektromagnetick\'e spojkov\'e pole):

\begin{equation}
\delta S_{EM} = - 2 \lambda \int \delta F \wedge \star F = - 2 \lambda \int d (\delta A) \wedge \star F = 2 \lambda \int \delta A \wedge d (\star F)
\end{equation}

Z \v{c}eho\v{z} plyne rovnice pole:

\begin{equation}
\boxed{d (\star F) = 0 \quad \Rightarrow \quad \nabla^\mu F_{\mu\nu} = 0}
\end{equation}

To jsou p\v{r}\'esn\v{e Maxwellovy rovnice v zak\v{r}iven\'em prostoro\v{c}ase.

\subsection*{4. Z\'av\v{e}r}

Roz\v{s}\'{i}\v{r}en\'im akce o kvadratick\'y \v{c}len \( \mathbf{R}^I \wedge \star \mathbf{R}^I \) zachov\'av\'ame kompatibilitu s Einsteinovou rovnic\'i a p\v{r}irozen\v{e z}\'isk\'av\'ame Maxwellovy rovnice jako pohybov\'e rovnice imagin\'arn\'i slo\v{z}ky konexe. Va\v{s}e teorie tak sjednocuje gravitaci s elektromagnetismem v \'upln\'em a elegantn\'im smyslu.
