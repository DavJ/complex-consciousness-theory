\documentclass{article}
\usepackage[utf8]{inputenc}
\title{P4 – Odvození jemné struktury z bikvaternionové teorie}
\author{}
\date{}

\begin{document}
\maketitle

\section*{Cíl}
Odvodit jemnou strukturu (fine-structure constant) $\alpha$ čistě z principů bikvaternionové teorie pole $\Theta(q, \tau)$.

\section*{Pozadí}
Většina teorií považuje $\alpha$ za empirickou konstantu, avšak v rámci UBT lze předpokládat, že $\alpha$ vzniká jako poměr frekvence vnitřního módu, škálovací konstanty a geometrických faktorů toroidální struktury.

\section*{Zadání}
Ukázat, že poměr $\alpha = \frac{e^2}{\hbar c}$ může být v této teorii odvozen symbolicky.

\end{document}
