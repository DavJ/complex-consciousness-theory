
\documentclass[12pt]{article}
\usepackage{amsmath,amssymb,geometry}
\geometry{margin=1in}
\title{Research Proposal: Topological Origin of Dark Matter in UBT}
\author{Unified Biquaternion Theory Initiative}
\date{\today}

\begin{document}

\maketitle

\section*{Motivation}
The nature of dark matter remains one of the most profound mysteries in modern physics. Despite extensive searches, no direct detection of dark matter particles has been confirmed. This motivates alternative explanations rooted in geometry and topology of the fundamental fields.

\section*{Hypothesis}
We propose that dark matter arises from topologically nontrivial, electromagnetically neutral configurations of the unified field \( \Theta(q, \tau) \) defined on the complexified spacetime \( \mathbb{C}^4 \). These configurations should:

\begin{itemize}
  \item Possess gravitational mass-energy,
  \item Be stable due to topological protection,
  \item Not interact electromagnetically with visible matter.
\end{itemize}

\section*{Research Goals}
\begin{enumerate}
  \item Identify classes of topologically stable field configurations in UBT,
  \item Analyze their energy-momentum contributions to the Einstein field equations,
  \item Compare the resulting mass-energy distribution with observed dark matter effects,
  \item Investigate implications for early universe structure formation.
\end{enumerate}

\section*{Expected Outcomes}
A successful theoretical derivation of dark matter effects from geometric field structures would provide a particle-free explanation consistent with observations. This may also bridge dark matter and dark energy within a unified topological framework.

\end{document}
