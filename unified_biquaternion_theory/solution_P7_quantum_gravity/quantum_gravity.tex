
\documentclass[12pt]{article}
\usepackage[a4paper, margin=2.5cm]{geometry}
\usepackage{amsmath, amssymb}
\usepackage{hyperref}
\usepackage{graphicx}
\usepackage{titlesec}
\usepackage{authblk}

\titleformat{\section}{\normalfont\Large\bfseries}{\thesection.}{0.5em}{}
\titleformat{\subsection}{\normalfont\large\bfseries}{\thesubsection.}{0.5em}{}

\title{\textbf{Quantum Gravity in the Unified Biquaternion Theory (UBT)}}
\author{David Jaroš}
\affil{\texttt{jdavid.cz@gmail.com}}
\date{July 2025}

\begin{document}

\maketitle

\begin{abstract}
We demonstrate how the Unified Biquaternion Theory (UBT), with its fundamental field $\Theta(q, \tau)$ defined over a biquaternionic spacetime $q \in \mathbb{B}$ and complex time $\tau = t + i\psi$, naturally leads to a theory of quantum gravity. Unlike conventional approaches that attempt to quantize spacetime or the metric tensor $g_{\mu\nu}$ directly, UBT derives spacetime geometry and gravitational interaction as emergent phenomena from the quantum properties of the $\Theta$ field.
\end{abstract}

\section{Introduction}

A consistent theory of quantum gravity remains one of the major unsolved problems in physics. Traditional approaches like string theory or loop quantum gravity attempt to quantize the metric structure of spacetime, often leading to formidable mathematical difficulties. UBT provides a fundamentally different and conceptually elegant route.

\section{Field $\Theta(q, \tau)$ as Quantum Substrate}

The field $\Theta$ is a quantum field defined over a non-commutative, algebraically rich space---the space of biquaternions $\mathbb{B}$---and a complexified time variable $\tau$. It obeys a generalized Fokker–Planck-like equation and supports excitations that correspond to physical observables.

\subsection{No Need to Quantize the Metric}

In UBT, the classical spacetime metric $g_{\mu\nu}$ is not fundamental. It arises from derivatives and bilinear forms of the field $\Theta$, via constructs like:
\[
g_{\mu\nu} = \Re\left(\partial_\mu \Theta^\dagger \cdot \partial_\nu \Theta \right)
\]
Thus, quantizing $\Theta$ is sufficient to produce quantum properties of geometry.

\section{Emergent Geometry and Gravitation}

Spacetime and gravity appear as macroscopic manifestations of the underlying dynamics of $\Theta$. This mirrors how fluid dynamics emerges from molecular motion.

\subsection{Analogy}

\begin{itemize}
  \item \textbf{Molecular Level:} Discrete quantum excitations of $\Theta$
  \item \textbf{Macroscopic Level:} Smooth geometry with classical gravitational field
\end{itemize}

\section{Implications for Planck Scale Physics}

At small scales, fluctuations of $\Theta$ generate metric fluctuations---providing a mechanism for quantum spacetime foam and a UV-complete description of gravity.

\section{Conclusion}

UBT avoids the pitfalls of direct metric quantization by focusing on the quantization of a fundamental, algebraic field $\Theta$. This naturally yields a viable framework for quantum gravity.

\section*{License}
This work is licensed under a Creative Commons Attribution 4.0 International License.

\end{document}
