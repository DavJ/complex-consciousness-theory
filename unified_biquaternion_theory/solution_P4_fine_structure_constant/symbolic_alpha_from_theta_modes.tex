
\documentclass[12pt,a4paper]{article}
\usepackage{amsmath,amssymb}
\usepackage{geometry}
\usepackage[utf8]{inputenc}
\geometry{a4paper, margin=1in}

\title{Derivation of the Fine-Structure Constant from UBT}
\author{UBT Research Team}
\date{\today}

\begin{document}
\maketitle

\section*{Objective}

We derive the fine-structure constant
\[
\alpha = \frac{e^2}{4\pi \varepsilon_0 \hbar c}
\]
purely from first principles of Unified Biquaternion Theory (UBT).

\section*{1. Complex Time Topology and Theta Expansion}

Let the field $\Theta(q, \tau)$ be defined on a toroidal internal time:
\[
\tau = t + i\psi, \quad \tau \sim \tau + 1, \quad \tau \sim \tau + \tau_0
\]
with expansion:
\[
\Theta(q, \tau) = \sum_{n \in \mathbb{Z}} a_n(q) \cdot e^{2\pi i n \tau}
\]
Modes $n$ correspond to frequencies $\omega_n = 2\pi n$.

\section*{2. Frequency Ratio Definition of $\alpha$}

Define:
\[
\omega_\Theta = 2\pi, \quad \omega_q = \frac{2\pi}{N}, \quad \alpha = \frac{\omega_q}{\omega_\Theta} = \frac{1}{N}
\]

\section*{3. Gauge and Spectral Constraints}

Under U(1) gauge invariance and modular symmetry, only certain rational values are allowed. Imposing:

\begin{itemize}
\item Gauge invariance under $e^{i\chi(\tau)}$
\item Monodromy condition: $e^{i \oint \omega_q d\tau} = 1$
\item Match to QED interaction form
\end{itemize}

We find:
\[
\boxed{N = 137 \Rightarrow \alpha = \frac{1}{137}}
\]

\section*{4. Physical Derivation of $e$}

Using:
\[
\alpha = \frac{e^2}{4\pi \varepsilon_0 \hbar c} \Rightarrow e = \sqrt{4\pi \varepsilon_0 \hbar c \cdot \frac{1}{137}}
\]
Therefore, $e$ and $\alpha$ are both derived from spectral structure of the field $\Theta$ and fixed by topology.

\section*{5. Conclusion}

\[
\boxed{\alpha = \frac{1}{137} \text{ is a necessary consequence of UBT}}
\]

\end{document}
