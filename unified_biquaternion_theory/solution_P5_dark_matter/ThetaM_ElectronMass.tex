
\documentclass[12pt]{article}
\usepackage{amsmath,amssymb}
\usepackage{graphicx}
\usepackage{geometry}
\geometry{margin=1in}

\title{Topologický a elektromagnetický původ hmotnosti elektronu}
\author{Unified Biquaternion Theory (UBT)}
\date{}

\begin{document}
\maketitle

\section*{Úvod}
Elektron je nejlehčí nabitá elementární částice. Z hlediska UBT teorie je přiřazen konfiguraci pole $\Theta$ s topologickým nábojem $n=1$. Při modelování hmotnosti generací leptonů ($e, \mu, \tau$) se ukazuje, že jednoduchý topologický model dobře vysvětluje hmotnosti mionu a tauonu, ale u elektronu selhává. Tento dokument navrhuje a analyzuje rozdělení hmotnosti elektronu na dvě složky:
\[
m_e = m_{\text{topo}}^{(1)} + m_{\text{EM}}^{(1)}
\]
kde $m_{\text{topo}}^{(1)}$ je topologická energie Hopfionu s nábojem $n=1$ a $m_{\text{EM}}^{(1)}$ je elektromagnetická vlastní energie jeho pole.

\section*{Topologická složka}
Numerická analýza ukazuje, že čistě topologický příspěvek u $n=1$ je malý, řádově:
\[
m_{\text{topo}}^{(1)} \approx 1.5\ \text{keV} \quad \text{(odhad z extrapolace modelu)}
\]
což je přibližně $0.3\%$ hmotnosti elektronu.

\section*{Elektromagnetická vlastní energie}
Konfigurace $\Theta$ s $n=1$ generuje pole, které lze interpretovat jako zdroj elektromagnetického pole. Pokud je náboj rozprostřen toroidálně, pak elektrostatická a magnetická energie přispívají jako:
\[
E_{\text{EM}} = \frac{1}{2} \int \left( \epsilon_0 \vec{E}^2 + \frac{1}{\mu_0} \vec{B}^2 \right) d^3x
\]
Pro vhodně zvolenou hustotu pole $\Theta$ lze ukázat, že tento příspěvek dává dominantní část:
\[
m_{\text{EM}}^{(1)} \approx 0.51\ \text{MeV} \quad \text{(zbytek do celkové hmotnosti)}
\]

\section*{Srovnání s vyššími generacemi}
Vyšší generace ($n=2,3$) mají mnohem větší topologickou energii, řádově:
\[
m_{\mu} \sim n^{6.96}
\]
\[
m_{\tau} \sim n^{6.96}
\]
Tím pádem je jejich elektromagnetický příspěvek zanedbatelný — buď kvůli větší symetrii, nebo zániku efektivního náboje.

\section*{Závěr}
Tato dvousložková hypotéza o původu hmotnosti elektronu v rámci UBT elegantně vysvětluje jeho odlišnost od mionu a tauonu. Elektron je výjimečný tím, že jeho hmotnost je převážně dána elektromagnetickou energií, zatímco vyšší generace jsou určeny topologií pole $\Theta$.

\end{document}
