\documentclass{article}
\usepackage{amsmath,amssymb}
\begin{document}

\section*{Model Elektronu jako Mód Pole \(\Theta\)}

Navrhujeme model, v němž elektron vzniká jako specifická excitace pole \(\Theta(q, \tau)\). Tato excitace má tvar:
\[
\Theta_e(q, \tau) = \psi(q) \otimes s,
\]
kde \(\psi(q)\) je prostorově-časová vlnová funkce a \(s\) je interní spinorová složka.

\subsection*{Hmotnost jako Vnitřní Frekvence}
Předpokládáme periodickou závislost v imaginární složce komplexního času \(\tau = t + i\psi\):
\[
\Theta(q, \tau) = e^{i\omega \psi} \Psi(q).
\]
Potom máme vztah mezi frekvencí a hmotností:
\[
m = \frac{\hbar \omega}{c^2}.
\]

\subsection*{Spin jako Algebraická Struktura}
Uvažujeme komponenty \(\Theta\) jako operátory splňující algebru:
\[
[\hat{s}_i, \hat{s}_j] = i \hbar \epsilon_{ijk} \hat{s}_k,
\]
což odpovídá spin-1/2 reprezentaci.

\subsection*{Interakce s Elektromagnetickým Polem}
V klasickém limitu generuje \(\Theta\) proud:
\[
j^\mu = \bar{\Theta} \gamma^\mu \Theta,
\]
což odpovídá QED interakci s potenciálem \(A_\mu\).


\section*{Author's Note}

This work was developed solely by Ing. David Jaroš.  
Large language models (ChatGPT-4o by OpenAI and Gemini 2.5 Pro by Google) were used strictly as assistive tools for calculations, LaTeX formatting, and critical review.  
All core ideas, equations, theoretical constructs and conclusions are the intellectual work of the author.

\end{document}
