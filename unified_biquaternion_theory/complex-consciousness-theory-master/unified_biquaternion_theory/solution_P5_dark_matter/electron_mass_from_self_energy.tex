
\documentclass[11pt]{article}
\usepackage{amsmath, amssymb}
\usepackage[a4paper, margin=2.5cm]{geometry}
\title{Derivation of the Electron Mass from Electromagnetic Self-Energy}
\author{
Ing.~David Jaroš \\
\textit{UBT Research Team} \\
\textbf{AI Assistants:} ChatGPT-4o (OpenAI), Gemini 2.5 Pro (Google) \\
Unified Biquaternion Theory}
\date{}

\begin{document}
\maketitle

\section*{Overview}

In this document, we present a derivation of the electron mass as arising from its own electromagnetic self-energy, in line with the hypothesis of dual mass origin proposed in the Unified Biquaternion Theory (UBT).

\section*{1. Self-Energy of a Smeared Charge Distribution}

We start from the classical expression for the electrostatic self-energy:
\[
\delta m_e = \frac{1}{2} \int \rho(\vec{x}) \phi(\vec{x})\, d^3x
\]
Assuming a Gaussian charge distribution for the topological field configuration (Hopfion), we solve Poisson's equation and obtain the electrostatic potential. The resulting self-energy is:
\[
\delta m_e = \frac{e^2}{\sqrt{\pi} R}
\]
where \( R \) is the effective "size" of the charge distribution.

\section*{2. Total Energy of the Hopfion Field \(\Theta_1\)}

We consider the topological solution:
\[
\Theta_1(\vec{x}) = \frac{1}{R} \cdot \frac{1}{(1 + r^2)^2}
\]
Computing the total energy density of this configuration:
\[
T_{00}(\vec{x}) = |\nabla \Theta_1|^2
\]
we obtain the total energy:
\[
E = \frac{\pi^2}{2 R^3}
\]

\section*{3. Effective Radius from Energy Density}

From the normalized energy density we compute the effective spatial variance:
\[
R_{\text{eff}}^2 = \frac{\int r^2 T_{00}(\vec{x})\, d^3x}{\int T_{00}(\vec{x})\, d^3x} = 5R^2
\quad \Rightarrow \quad R = \frac{R_{\text{eff}}}{\sqrt{5}}
\]

\section*{Conclusion}

The parameter \( R \) used in the self-energy calculation is not a free constant. It is uniquely determined by the topological field solution \(\Theta_1\), completing the prediction of the electron mass from first principles.


\section*{Author's Note}

This work was developed solely by Ing. David Jaroš.  
Large language models (ChatGPT-4o by OpenAI and Gemini 2.5 Pro by Google) were used strictly as assistive tools for calculations, LaTeX formatting, and critical review.  
All core ideas, equations, theoretical constructs and conclusions are the intellectual work of the author.

\end{document}

