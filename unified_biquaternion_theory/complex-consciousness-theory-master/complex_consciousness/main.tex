
\documentclass[12pt]{article}
\usepackage{amsmath,amsfonts}
\usepackage{graphicx}
\usepackage{authblk}
\usepackage{hyperref}
\usepackage{geometry}
\geometry{margin=1in}

\title{A Complex-Time Theory of Consciousness:\\ Drift, Diffusion, and Phase Collapse in Toroidal Cognitive Space}
\author{
Ing.~David Jaroš \\
\textit{UBT Research Team} \\
\textbf{AI Assistants:} ChatGPT-4o (OpenAI), Gemini 2.5 Pro (Google) \\
Ing. David Jaroš\thanks{Email: jdavid.cz@gmail.com} \\
\textit{Independent Researcher, Prague, Czech Republic}}

\date{}

\begin{document}

\maketitle

\begin{abstract}
This paper proposes a novel theoretical framework for consciousness in which time is treated as a complex quantity, with its real part representing conscious awareness and its imaginary part encoding progression and entropy. We model consciousness as a dynamic point on a complex torus, with its evolution governed by modulations in frequency and orientation. This perspective provides an intuitive yet mathematically tractable way of linking subjective experience, quantum decoherence, and entropy flow. In particular, we argue that perception corresponds to a selection mechanism---a projection---onto a quasi-stable trajectory through complexified configuration space. Extreme dynamical changes in conscious orientation may correspond to death and rebirth, where the torus collapses and reinitializes at light-speed transitions. This model naturally invites a reinterpretation of known structures, such as the time-frequency duality and theta functions. We propose that the evolution of perceptual orientation may be governed by a theta function-based predictive equation. Notably, we discuss the potential role of Jacobi's identity in describing the internal consistency of these transformations. Implications for multiverse structure, entropy, and conscious interference are explored.
\end{abstract}

\section{Introduction}

Despite extensive research spanning biological, quantum, and information-based models, there is a shortage of mathematically rigorous theories merging subjective experience with physical evolution. Here, we fill this gap by extending time $t$ and space $x$ into the complex domain:
\[
T = t + i\psi, \quad Z = x + i\chi,
\]
where $\psi$ encodes awareness phase, and $\chi$ encodes perceptual depth. Using the Fokker--Planck framework and Jacobi theta functions, we develop a probabilistic model of conscious evolution on a toroidal phase space. Drift corresponds to directed movement in time, while diffusion corresponds to mental variability.

\section{Complex Time, Space, and Awareness}

Time $T = t + i\psi$ combines physical progression with a subjective awareness phase. Space $Z = x + i\chi$ represents both position and perceptual orientation. These complex-valued coordinates offer a natural representation for the interaction between external causality and internal experience.

\section{Toroidal Phase Space and Theta Functions}

The toroidal topology provides a compact, periodic structure suitable for representing recurrent conscious states. The Jacobi theta function $\vartheta(z, \tau)$ \cite{mumford1983tata} describes the distribution of these states:
\[
P(Z, T) = \vartheta\left(\frac{\pi Z}{L}, e^{-T}\right),
\]
where $L$ is the spatial period and $T$ is complex time.

\section{Modified Fokker--Planck Evolution}

A modified Fokker--Planck equation governs the evolution of conscious probability densities in complex space. The standard equation
\[
\frac{\partial p(x,t)}{\partial t} = -\frac{\partial}{\partial x}[\mu(x,t)p] + D \frac{\partial^2 p}{\partial x^2}
\]
\cite{risken1996fokker}
is extended by replacing $t \rightarrow T$ and $x \rightarrow Z$, allowing drift and diffusion within a toroidal cognitive space.

\section{Relativistic Collapse and Cognitive Velocity}

By adopting a Lorentz-like transformation \cite{einstein1905elektrodynamik} in $(T, Z)$:
\[
T' = \gamma(T - vZ), \quad Z' = \gamma(Z - vT), \quad \gamma = \frac{1}{\sqrt{1 - v^2/c^2}},
\]
we show that as the velocity $v \rightarrow c$, the toroidal structure collapses ($L \rightarrow 0$), potentially corresponding to a point of cognitive singularity.

\section{Philosophical Interpretation}

We speculate that such singularities may represent transitions such as death, birth, or non-dual awareness states. The collapse of toroidal geometry may align with subjective reports of timelessness, unity, or ego dissolution. This resonates with multiverse theories where reality branches and conscious experience selects one consistent narrative path, a concept aligned with Deutsch's interpretation of the multiverse and the role of consciousness within it \cite{deutsch1997fabric}.

\section{Prediction and Duality}

Jacobi identities relate $\vartheta(z, \tau)$ to its transformed counterpart under $\tau \mapsto -1/\tau$, suggesting a time--frequency duality analogous to the Fourier transform:
\[
\vartheta(z, \tau) = \sqrt{-i\tau} \, e^{i\pi z^2/\tau} \, \vartheta\left(\frac{z}{\tau}, -\frac{1}{\tau}\right)
\]
This hints at a speculative possibility: if awareness exhibits such a dual structure, then future states may be encoded in current phase information. Rigorously, this may not yield deterministic "premonitions", but rather a field of constrained future probabilities shaped by current internal structure.

\section{Conclusion and Future Work}

This theory provides a new formal language to relate consciousness to physics. Future directions include simulation, application to AI attention models, EEG correlation with $\psi(t)$, and exploration of Jacobi identities for predictive modeling.

\bibliographystyle{plain}
\bibliography{bibliography}


\section*{Author's Note}

This work was developed solely by Ing. David Jaroš.  
Large language models (ChatGPT-4o by OpenAI and Gemini 2.5 Pro by Google) were used strictly as assistive tools for calculations, LaTeX formatting, and critical review.  
All core ideas, equations, theoretical constructs and conclusions are the intellectual work of the author.

\end{document}

