
\documentclass{article}
\usepackage{amsmath}
\begin{document}

\section*{Odvození Efektivního Potenciálu}

Cílem je ukázat, jak může efektivní potenciál pole $\chi$ vzniknout z interakcí s ostatními složkami pole $\Theta$.

Začněme Lagrangiánem obsahujícím interakční člen:
\[
\mathcal{L}_{\text{int}} = -\lambda (\chi^2 - \alpha)^2.
\]
Tento člen přirozeně generuje bistabilní potenciál:
\[
V(\chi) = \lambda (\chi^2 - \alpha)^2 = \lambda \chi^4 - 2\lambda \alpha \chi^2 + \lambda \alpha^2.
\]
Pro $\lambda = \frac{1}{4}$, $\alpha = 1$ dostáváme:
\[
V(\chi) = \frac{1}{4} \chi^4 - \frac{1}{2} \chi^2 + \text{konst.}
\]

\end{document}
