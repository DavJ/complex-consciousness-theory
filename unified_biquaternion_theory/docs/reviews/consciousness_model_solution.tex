
\documentclass{article}
\usepackage{amsmath}
\begin{document}

\section*{Toy Model Vědomí – Přehled}

Tento dokument shrnuje hlavní ideu "toy modelu" vědomí založeného na dynamice pole $\chi(t)$ v bistabilním potenciálu.

Model předpokládá, že kódování mentálního stavu může být popsáno jako evoluce skalárního pole v potenciálu tvaru
\[
V(\chi) = \frac{1}{4} \chi^4 - \frac{1}{2} \chi^2.
\]
Tento potenciál má dvě minima, reprezentující dva různé mentální stavy nebo rozhodnutí.

Evoluce pole je popsána stochastickou diferenciální rovnicí nebo Fokker-Planckovou rovnicí:
\[
\frac{\partial P(\chi, t)}{\partial t} = \frac{\partial}{\partial \chi} \left( \frac{\partial V(\chi)}{\partial \chi} P(\chi, t) \right) + D \frac{\partial^2 P(\chi, t)}{\partial \chi^2}.
\]

\end{document}
