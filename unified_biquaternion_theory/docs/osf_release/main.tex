\documentclass[12pt, a4paper]{article}
\usepackage[utf8]{inputenc}
\usepackage[english]{babel}
\usepackage{amsmath, amssymb}
\usepackage{geometry}
\usepackage{graphicx}
\usepackage{hyperref}
\usepackage{siunitx}

\geometry{a4paper, margin=1in}
\linespread{1.15}

\title{\textbf{The Unified Biquaternion Theory: A Framework for Fundamental Constants, Dark Matter, and Mass Hierarchy}}

\author{
  Ing. David Jaroš\thanks{Primary author and theorist. Contact: \texttt{jdavid.cz@gmail.com}} \\
  \textit{with AI assistance from ChatGPT-4o and Gemini 2.5 Pro (as technical aides)}
}

\date{July 4, 2025}


\begin{document}
\maketitle


\begin{abstract}
We present the Unified Biquaternion Theory (UBT), a theoretical framework based on a complexified spacetime and a fundamental biquaternion-valued field, \( \Theta(q) \). UBT aims to reconcile quantum field theory and general relativity by providing natural derivations for fundamental constants and particle properties from the geometry and topology of the underlying manifold. The theory predicts the emergence of known particles as topological excitations and introduces new fields associated with consciousness and dark matter. We demonstrate that UBT reduces to known physical theories in their respective limits and offers falsifiable predictions. This paper outlines the theory's core tenets and its successful application to three major unsolved problems: the origin of the fine-structure constant, the nature of dark matter, and the lepton mass hierarchy.
\end{abstract}

\tableofcontents
\newpage

\section{Introduction}

\subsection{The Limits of the Standard Model and General Relativity}
Modern physics is built upon two incredibly successful pillars: General Relativity (GR), which describes gravity, and the Standard Model (SM) of particle physics, which describes the other three fundamental forces via Quantum Field Theory (QFT). Despite their triumphs, a complete picture of reality remains elusive. Key open problems include the incompatibility of GR and QFT, the nature of dark matter, and the fact that the SM contains ~20 free parameters (e.g., the fine-structure constant \(\alpha\), particle masses) whose values are determined by experiment but not explained by theory.

\subsection{The UBT Proposal: A New Foundation}
The Unified Biquaternion Theory (UBT) proposes a new foundation to address these challenges. It postulates that physical reality unfolds on a complexified manifold where all four spacetime coordinates are complex-valued. The fundamental entity is a single biquaternion-valued field, \( \Theta(q) \), whose algebraic and topological properties are hypothesized to give rise to all known particles, forces, and even spacetime itself.

\subsection{Structure of the Paper}
This paper is structured as follows. Section 2 introduces the mathematical foundations of UBT. Section 3 demonstrates how the geometry of GR emerges from the theory. Section 4 details the derivation of fundamental constants and particle masses. Section 5 shows the compatibility of UBT with the Standard Model. Section 6 presents the theory's predictions, including a model for dark matter and new particles. Finally, Section 7 summarizes the results and outlines future work.

\section{Foundations of the Unified Biquaternion Theory}

\subsection{The Biquaternionic Manifold}
The foundational postulate of UBT is that physical reality unfolds not on a real-valued manifold, but on a **biquaternionic manifold**, which we can denote as \( \mathcal{M}_{\mathbb{B}} \). In this framework, each of the four fundamental coordinates \(q^\mu\) (\(\mu=0,1,2,3\)) is itself a biquaternion (a complexified quaternion), an element of \( \mathbb{B} \cong \mathbb{C} \otimes \mathbb{H} \).

This means each coordinate can be expanded on the quaternion basis \( \{1, \mathbf{i}, \mathbf{j}, \mathbf{k}\} \) with complex coefficients:
\begin{equation}
    q^\mu = z_0^\mu \cdot 1 + z_1^\mu \cdot \mathbf{i} + z_2^\mu \cdot \mathbf{j} + z_3^\mu \cdot \mathbf{k}
\end{equation}
where each coefficient \( z_\nu^\mu = x_\nu^\mu + i y_\nu^\mu \) is a complex number.

The classical spacetime coordinates \(x^\mu\) that we observe are identified with the **real-scalar part** of this rich structure:
\begin{equation}
    x^\mu \equiv x_0^\mu \quad (\text{for } \mu=0,1,2,3)
\end{equation}
All other components (\(y_0^\mu\), and all \(x_{1,2,3}^\mu\), \(y_{1,2,3}^\mu\)) represent internal degrees of freedom of spacetime itself, encoding phase, spin, and gauge information directly into the geometry of the manifold. The simplified notation \( \tau = t + i\psi \), which we use elsewhere, refers specifically to the complexification of the scalar part of the time coordinate \(q^0\).

\subsection{The Fundamental Field \( \Theta(q) \)}
The central object is a biquaternion-valued spinor field \( \Theta(q) \in \mathbb{B} \otimes \mathbb{C}^4 \). This single field acts as a universal precursor to all other fields. Its various components and excitational modes correspond to the particles of the Standard Model. It can be locally decomposed into its amplitude and phase: \( \Theta(q) = \rho(q) e^{i\phi(q)} \).

\subsection{The Equation of Motion}
The dynamics of \( \Theta \) are governed by a generalized covariant wave equation. A simplified form can be expressed as:
\begin{equation}
    \left( \mathcal{D}_\mu \mathcal{D}^\mu + m^2 \right) \Theta(q) = 0
\end{equation}
where \( \mathcal{D}_\mu = \partial_\mu + \Omega_\mu \) is the biquaternionic covariant derivative. This equation unifies wave-like propagation in the real part of the coordinates with diffusion-like dynamics in the imaginary part. One of the constraints emerging from this equation is a condition of orthogonality between the gradients of the amplitude and phase: \( \eta^{\mu\nu} \partial_\mu \rho \, \partial_\nu \phi = 0 \).

\section{Emergent Physics I: Spacetime and Gravity}

\subsection{The Emergent Metric}
The effective metric tensor \( g_{\mu\nu} \) of spacetime is not fundamental but emerges from the correlations of the \( \Theta \) field:
\begin{equation}

