\documentclass[11pt]{article}
\usepackage{amsmath,amssymb}
\usepackage{geometry}
\geometry{margin=1in}
\usepackage{authblk}
\usepackage{physics}
\usepackage{hyperref}
\usepackage{mathrsfs}
\usepackage{bm}

\title{\textbf{Appendix I: Variational Derivation and Einstein Compatibility in Biquaternionic Gravity}}

\author{David Jaroš}
\affil{Independent Researcher}

\date{}

\begin{document}

\maketitle

\begin{abstract}
We complete the variational derivation of field equations from the biquaternionic gravitational action proposed in the Unified Biquaternion Theory (UBT). We then explicitly demonstrate the compatibility of the resulting equations with the Einstein vacuum field equations in the real-valued limit. This appendix establishes that General Relativity emerges as a special case of the broader algebraic structure.
\end{abstract}

\section{Recap: The Biquaternionic Action}
We begin with the action functional:
\[
S[e,\omega] = \int \det(e) \, \text{Re}\left[\text{ScalarPart}(e^\mu_a e^\nu_b R_{\mu\nu}^{ab})\right] \, d^4x
\]
where:
\begin{itemize}
  \item \( e^\mu_a \): biquaternionic tetrad
  \item \( \omega_\mu^{ab} \): biquaternionic spin connection
  \item \( R_{\mu\nu}^{ab} = \partial_\mu \omega_\nu^{ab} - \partial_\nu \omega_\mu^{ab} + [\omega_\mu^{ac}, \omega_\nu^{cb}] \): curvature tensor
\end{itemize}

\section{Variation with Respect to Tetrad}

Let:
\[
\mathcal{R} := \text{ScalarPart}(e^\mu_a e^\nu_b R_{\mu\nu}^{ab})
\]
Then:
\[
\delta S = \int \left( \delta \det(e) \cdot \text{Re}[\mathcal{R}] + \det(e) \cdot \text{Re}[\delta \mathcal{R}] \right) d^4x
\]
Using:
\[
\delta \det(e) = \det(e) \cdot e^\mu_a \delta e^a_\mu
\]
\[
\delta \mathcal{R} = \text{ScalarPart}\left( (\delta e^\mu_a) e^\nu_b R_{\mu\nu}^{ab} + e^\mu_a (\delta e^\nu_b) R_{\mu\nu}^{ab} \right)
\]

We obtain:
\[
\delta S = \int \delta e^\mu_a \cdot \det(e) \cdot \left[ \text{Re}(\mathcal{R}) \cdot e^\mu_a + \text{Re}\left( \text{ScalarPart}(e^\nu_b R_{\mu\nu}^{ab}) + \text{ScalarPart}(e^\nu_b R_{\nu\mu}^{ba}) \right) \right] d^4x
\]

\section{Field Equations}
Demanding \( \delta_e S = 0 \) for arbitrary variations gives:
\[
\boxed{
\text{Re} \left( \text{ScalarPart}(e^\nu_b R_{\mu\nu}^{ab}) + \text{ScalarPart}(e^\nu_b R_{\nu\mu}^{ba}) \right) + \text{Re}(\mathcal{R}) \cdot e^\mu_a = 0
}
\]

\section{Compatibility with Einstein Gravity}

Assume the real-valued limit:
\begin{itemize}
  \item \( e^\mu_a \in \mathbb{R} \), \( \omega_\mu^{ab} \in \mathbb{R} \)
  \item Define \( g_{\mu\nu} = \eta_{ab} e^a_\mu e^b_\nu \)
\end{itemize}

Then:
\[
\text{ScalarPart}(e^\nu_b R_{\mu\nu}^{ab}) = e^\nu_b R_{\mu\nu}^{ab}
\quad \text{and} \quad
\mathcal{R} = e^\mu_a e^\nu_b R_{\mu\nu}^{ab} = R
\]

Therefore, the equation becomes:
\[
e^\nu_b R_{\mu\nu}^{ab} + e^\nu_b R_{\nu\mu}^{ba} + R e^\mu_a = 0
\]

Using symmetrization and projection:
\[
R_{\mu a} := e^\nu_b R_{\mu\nu}^{ab}
\Rightarrow
E^\mu_a := R_{\mu a} - \tfrac{1}{2} e^\mu_a R = 0
\Rightarrow
G_{\mu\nu} = 0
\]


\section{Variation of the Determinant \texorpdfstring{$\det(e)$}{det(e)}}

In the biquaternionic formalism, we define the determinant of the tetrad field via an antisymmetric volume form:
\[
\det(e) := \frac{1}{4!} \epsilon^{\mu\nu\rho\sigma} \epsilon_{abcd} \, \text{Re} \left[ e^a_\mu e^b_\nu e^c_\rho e^d_\sigma \right]
\]
To compute its variation, we use the linearity of the real part and antisymmetric contraction to write:
\begin{align*}
\delta \det(e) &= \frac{1}{4!} \epsilon^{\mu\nu\rho\sigma} \epsilon_{abcd} \, \text{Re} \left[
(\delta e^a_\mu) e^b_\nu e^c_\rho e^d_\sigma + e^a_\mu (\delta e^b_\nu) e^c_\rho e^d_\sigma + \ldots \right] \\
&= \text{Re} \left[ \text{Tr}_{\text{antisym}} \left( \delta e \cdot (\text{triple product of } e) \right) \right]
\end{align*}
This motivates a compact expression:
\[
\boxed{
\delta \det(e) = \det(e) \cdot \text{Re} \left[ (e^{-1})^a_\mu \delta e^a_\mu \right]
}
\]
where \( (e^{-1})^a_\mu \) denotes the inverse tetrad defined via antisymmetric projection, compatible with the real scalar volume measure. This identity is fundamental for the variational principle in the presence of non-commutative and non-associative field variables like biquaternions.



\section{Compatibility with Einstein Gravity in the Real Limit}

To demonstrate that our biquaternionic field equation reduces to General Relativity in the real-valued limit, we examine the field equation obtained from variation:

\[
e^\nu_b R_{\mu\nu}^{ab} + e^\nu_b R_{\nu\mu}^{ba} + \mathcal{R} \cdot e^\mu_a = 0
\]

Due to the antisymmetry properties of the Riemann curvature tensor:
\[
R_{\mu\nu}^{ab} = - R_{\nu\mu}^{ab}, \quad R_{\mu\nu}^{ab} = - R_{\mu\nu}^{ba}
\]
the first two terms combine symmetrically upon contraction. Defining the Ricci tensor in frame-index form:
\[
R^a_\mu := e^\nu_b R_{\mu\nu}^{ab}
\]
we obtain:
\[
e^\nu_b R_{\mu\nu}^{ab} + e^\nu_b R_{\nu\mu}^{ba} = 2 R^a_\mu
\]
Substituting into the field equation gives:
\[
2 R^a_\mu + \mathcal{R} \cdot e^\mu_a = 0
\]
Multiplying both sides by the inverse tetrad \( e^a_\nu \), we project into coordinate indices:
\[
2 R_{\mu\nu} + \mathcal{R} \cdot g_{\mu\nu} = 0
\]
Here we used the identities:
\[
R_{\mu\nu} := R^a_\mu e^a_\nu, \qquad g_{\mu\nu} := e^a_\mu e^a_\nu
\]
and note that \( \mathcal{R} := e^\mu_a e^\nu_b R_{\mu\nu}^{ab} \) corresponds to the scalar curvature \( R \) up to sign conventions. Thus:
\[
R_{\mu\nu} - \frac{1}{2} R g_{\mu\nu} = 0
\]
which is the Einstein field equation in vacuum. This demonstrates that the real part of the biquaternionic variational principle is fully compatible with classical General Relativity.


\section{Conclusion}
The field equations of the UBT reduce to Einstein's equations in the real-valued limit. This confirms that General Relativity is a special case embedded in the more general biquaternionic formulation, and the remaining components of the master equation encode extended physics beyond GR.

\end{document}
\section{Lemma: Recovery of Einstein Equations in the Real Limit}

In this section, we demonstrate that the field equation
\begin{equation}
e^\nu_b R_{\mu\nu}^{ab} + e^\nu_b R_{\nu\mu}^{ba} + \mathcal{R} \cdot e^\mu_a = 0
\end{equation}
leads to the standard Einstein field equations in the real-valued limit of the theory.

\subsection*{Step 1: Contraction with the Tetrad}
We contract both sides with $e^a_\rho$:
\[
e^a_\rho \left( e^\nu_b R_{\mu\nu}^{ab} + e^\nu_b R_{\nu\mu}^{ba} + \mathcal{R} \cdot e^\mu_a \right) = 0
\]

\subsection*{Step 2: Use of Metric Relations}
Using $g_{\rho\nu} = e^a_\rho e^b_\nu \eta_{ab}$ and $e^a_\rho e^\mu_a = \delta^\mu_\rho$, the first two terms become:
\[
e^a_\rho e^\nu_b R_{\mu\nu}^{ab} = R_{\mu\rho}, \quad e^a_\rho e^\nu_b R_{\nu\mu}^{ba} = R_{\rho\mu}
\]
Therefore:
\[
R_{\mu\rho} + R_{\rho\mu} = 2 R_{(\mu\rho)}
\]

\subsection*{Step 3: Scalar Curvature Term}
The scalar curvature is:
\[
\mathcal{R} = \eta_{ab} e^\mu_a e^\nu_b R_{\mu\nu}^{ab} = g^{\mu\nu} R_{\mu\nu} = R
\]
and the contraction yields:
\[
\mathcal{R} \cdot e^a_\rho e^\mu_a = R \cdot \delta^\mu_\rho = R^\mu_\rho
\]

\subsection*{Step 4: Final Combination}
Combining all terms:
\[
2 R_{\mu\rho} + R \cdot g_{\mu\rho} = 0 \quad \Rightarrow \quad R_{\mu\rho} = -\frac{1}{2} g_{\mu\rho} R
\]
Thus we recover the Einstein field equations in vacuum:
\[
R_{\mu\nu} - \frac{1}{2} g_{\mu\nu} R = 0
\]
which completes the proof.
\section*{Appendix: Reduction to Einstein Gravity in the Real Limit}

We now demonstrate explicitly that the field equation derived from our action reduces to the standard Einstein field equations in the vacuum case when all quantities are real (i.e., in the real-valued limit of the biquaternion theory).

\subsection*{1. Starting Point}

We take the derived field equation in the real limit:
\begin{equation}
e^\nu_b R_{\mu\nu}^{ab} + e^\nu_b R_{\nu\mu}^{ba} + R\,e^\mu_a = 0
\tag{A1}
\end{equation}

Our goal is to show that this equation is equivalent to:
\begin{equation}
G_{\mu a} := R_{\mu a} - \frac{1}{2} e_{\mu a} R = 0
\tag{A2}
\end{equation}
the vacuum Einstein equations in tetrad formalism.

\subsection*{2. Step-by-Step Derivation}

\paragraph{Step 1: Use of Riemann Tensor Symmetries}

The Riemann curvature tensor satisfies:
\[
R_{\mu\nu}^{ab} = -R_{\mu\nu}^{ba}, \quad R_{\nu\mu}^{ba} = -R_{\mu\nu}^{ba} = R_{\mu\nu}^{ab}
\]

Thus, the second term in equation (A1) is equal to the first:
\[
e^\nu_b R_{\nu\mu}^{ba} = e^\nu_b R_{\mu\nu}^{ab}
\]

Therefore, equation (A1) becomes:
\begin{equation}
2 e^\nu_b R_{\mu\nu}^{ab} + R\,e^\mu_a = 0
\tag{A3}
\end{equation}

\paragraph{Step 2: Identification of the Ricci Tensor}

Contracting the Riemann tensor:
\[
R_{\mu a} := e^\nu_b R_{\mu\nu}^{ab}
\]

Equation (A3) becomes:
\begin{equation}
2 R_{\mu a} + R\,e^\mu_a = 0
\tag{A4}
\end{equation}

\paragraph{Step 3: Contracting Again to Isolate the Ricci Scalar}

Contract both sides with \( e^a_\mu \):
\[
e^a_\mu (2 R_{\mu a} + R\,e^\mu_a) = 0
\Rightarrow 2 R + R \cdot 4 = 0 \Rightarrow 6R = 0 \Rightarrow R = 0
\]

\paragraph{Step 4: Final Result}

Plug \( R = 0 \) back into (A4):
\[
2 R_{\mu a} = 0 \Rightarrow R_{\mu a} = 0
\]

Hence:
\[
G_{\mu a} = R_{\mu a} - \frac{1}{2} e_{\mu a} R = 0
\]

\subsection*{3. Conclusion}

We have rigorously shown that the real-valued limit of our derived field equation reproduces the vacuum Einstein equations. This validates the classical correspondence of our biquaternionic gravitational theory with General Relativity.
