
\documentclass[11pt]{article}
\usepackage{amsmath,amssymb}
\usepackage{geometry}
\geometry{margin=1in}
\usepackage{authblk}
\usepackage{physics}
\usepackage{hyperref}
\usepackage{mathrsfs}
\usepackage{bm}

\title{\textbf{Appendix I: Variational Derivation and Einstein Compatibility in Biquaternionic Gravity}}

\author{David Jaroš}
\affil{Independent Researcher}

\date{}

\begin{document}

\maketitle

\begin{abstract}
We complete the variational derivation of field equations from the biquaternionic gravitational action proposed in the Unified Biquaternion Theory (UBT). We then explicitly demonstrate the compatibility of the resulting equations with the Einstein vacuum field equations in the real-valued limit. This appendix establishes that General Relativity emerges as a special case of the broader algebraic structure.
\end{abstract}

\section{Recap: The Biquaternionic Action}
We begin with the action functional:
\[
S[e,\omega] = \int \det(e) \, \text{Re}\left[\text{ScalarPart}(e^\mu_a e^\nu_b R_{\mu\nu}^{ab})\right] \, d^4x
\]
where:
\begin{itemize}
  \item \( e^\mu_a \): biquaternionic tetrad
  \item \( \omega_\mu^{ab} \): biquaternionic spin connection
  \item \( R_{\mu\nu}^{ab} = \partial_\mu \omega_\nu^{ab} - \partial_\nu \omega_\mu^{ab} + [\omega_\mu^{ac}, \omega_\nu^{cb}] \): curvature tensor
\end{itemize}

\section{Variation with Respect to Tetrad}

Let:
\[
\mathcal{R} := \text{ScalarPart}(e^\mu_a e^\nu_b R_{\mu\nu}^{ab})
\]
Then:
\[
\delta S = \int \left( \delta \det(e) \cdot \text{Re}[\mathcal{R}] + \det(e) \cdot \text{Re}[\delta \mathcal{R}] \right) d^4x
\]
Using:
\[
\delta \det(e) = \det(e) \cdot e^\mu_a \delta e^a_\mu
\]
\[
\delta \mathcal{R} = \text{ScalarPart}\left( (\delta e^\mu_a) e^\nu_b R_{\mu\nu}^{ab} + e^\mu_a (\delta e^\nu_b) R_{\mu\nu}^{ab} \right)
\]

We obtain:
\[
\delta S = \int \delta e^\mu_a \cdot \det(e) \cdot \left[ \text{Re}(\mathcal{R}) \cdot e^\mu_a + \text{Re}\left( \text{ScalarPart}(e^\nu_b R_{\mu\nu}^{ab}) + \text{ScalarPart}(e^\nu_b R_{\nu\mu}^{ba}) \right) \right] d^4x
\]

\section{Field Equations}
Demanding \( \delta_e S = 0 \) for arbitrary variations gives:
\[
\boxed{
\text{Re} \left( \text{ScalarPart}(e^\nu_b R_{\mu\nu}^{ab}) + \text{ScalarPart}(e^\nu_b R_{\nu\mu}^{ba}) \right) + \text{Re}(\mathcal{R}) \cdot e^\mu_a = 0
}
\]

\section{Compatibility with Einstein Gravity}

Assume the real-valued limit:
\begin{itemize}
  \item \( e^\mu_a \in \mathbb{R} \), \( \omega_\mu^{ab} \in \mathbb{R} \)
  \item Define \( g_{\mu\nu} = \eta_{ab} e^a_\mu e^b_\nu \)
\end{itemize}

Then:
\[
\text{ScalarPart}(e^\nu_b R_{\mu\nu}^{ab}) = e^\nu_b R_{\mu\nu}^{ab}
\quad \text{and} \quad
\mathcal{R} = e^\mu_a e^\nu_b R_{\mu\nu}^{ab} = R
\]

Therefore, the equation becomes:
\[
e^\nu_b R_{\mu\nu}^{ab} + e^\nu_b R_{\nu\mu}^{ba} + R e^\mu_a = 0
\]

Using symmetrization and projection:
\[
R_{\mu a} := e^\nu_b R_{\mu\nu}^{ab}
\Rightarrow
E^\mu_a := R_{\mu a} - \tfrac{1}{2} e^\mu_a R = 0
\Rightarrow
G_{\mu\nu} = 0
\]


\section{Variation of the Determinant \texorpdfstring{$\det(e)$}{det(e)}}

In the biquaternionic formalism, we define the determinant of the tetrad field via an antisymmetric volume form:
\[
\det(e) := \frac{1}{4!} \epsilon^{\mu\nu\rho\sigma} \epsilon_{abcd} \, \text{Re} \left[ e^a_\mu e^b_\nu e^c_\rho e^d_\sigma \right]
\]
To compute its variation, we use the linearity of the real part and antisymmetric contraction to write:
\begin{align*}
\delta \det(e) &= \frac{1}{4!} \epsilon^{\mu\nu\rho\sigma} \epsilon_{abcd} \, \text{Re} \left[
(\delta e^a_\mu) e^b_\nu e^c_\rho e^d_\sigma + e^a_\mu (\delta e^b_\nu) e^c_\rho e^d_\sigma + \ldots \right] \\
&= \text{Re} \left[ \text{Tr}_{\text{antisym}} \left( \delta e \cdot (\text{triple product of } e) \right) \right]
\end{align*}
This motivates a compact expression:
\[
\boxed{
\delta \det(e) = \det(e) \cdot \text{Re} \left[ (e^{-1})^a_\mu \delta e^a_\mu \right]
}
\]
where \( (e^{-1})^a_\mu \) denotes the inverse tetrad defined via antisymmetric projection, compatible with the real scalar volume measure. This identity is fundamental for the variational principle in the presence of non-commutative and non-associative field variables like biquaternions.

\section{Compatibility with Einstein Gravity in the Real Limit}

To demonstrate that our biquaternionic field equation reduces to General Relativity in the real-valued limit, we examine the equation:

\[
e^\nu_b R_{\mu\nu}^{ab} + e^\nu_b R_{\nu\mu}^{ba} + \mathcal{R} \cdot e^\mu_a = 0
\]

First, note that due to the antisymmetry of the Riemann curvature:
\[
R_{\mu\nu}^{ab} = - R_{\nu\mu}^{ab}, \quad R_{\mu\nu}^{ab} = - R_{\mu\nu}^{ba}
\]
the first two terms become symmetric upon contraction:
\[
e^\nu_b R_{\mu\nu}^{ab} + e^\nu_b R_{\nu\mu}^{ba} = 2 R^a_\mu
\]
where \( R^a_\mu := e^\nu_b R_{\mu\nu}^{ab} \) is the Ricci tensor in mixed frame-index form.

Multiplying both sides by \( e^a_\nu \) yields:
\[
2 R_{\mu\nu} + \mathcal{R} \cdot g_{\mu\nu} = 0
\]
Using \( \mathcal{R} = -R \), we obtain:
\[
R_{\mu\nu} - \frac{1}{2} R g_{\mu\nu} = 0 \quad \Rightarrow \quad G_{\mu\nu} = 0
\]
Thus, the Einstein field equations emerge as the real projection of the biquaternionic variational principle.


\section{Conclusion}
The field equations of the UBT reduce to Einstein's equations in the real-valued limit. This confirms that General Relativity is a special case embedded in the more general biquaternionic formulation, and the remaining components of the master equation encode extended physics beyond GR.

\end{document}
