\section*{Appendix 6: Algebraic Validation of Electromagnetic Emergence}

\subsection*{1. Objective}

In Appendix 5, we hypothesized that the electromagnetic four-potential \( A_\mu \) is encoded in the imaginary part of the spin connection via the ansatz:
\[
\boldsymbol{\omega}_{\mu}^{(I)} = A_\mu \, \mathbf{v}_{em}
\]
where \( \mathbf{v}_{em} \) is a fixed unit imaginary quaternion vector (e.g., \( \mathbf{i} \)).

To validate this mechanism, we consider a simplified model and examine whether this ansatz leads to an expression for the curvature \( \mathbf{R}_{\mu\nu} \) that contains the expected field strength tensor \( F_{\mu\nu} = \partial_\mu A_\nu - \partial_\nu A_\mu \).

\subsection*{2. Simplified Model}

We assume the absence of gravity, i.e., no real part of the connection:
\[
\boldsymbol{\omega}_\mu^{(R)} = 0
\]
and adopt the ansatz for the imaginary part:
\[
\boldsymbol{\omega}_\mu^{(I)} = A_\mu \mathbf{v}_{em}
\]
We now compute the imaginary part of the curvature:
\[
\mathbf{R}_{\mu\nu}^{(I)} = \partial_\mu \boldsymbol{\omega}_\nu^{(I)} - \partial_\nu \boldsymbol{\omega}_\mu^{(I)} + [\boldsymbol{\omega}_\mu^{(I)}, \boldsymbol{\omega}_\nu^{(I)}]
\]

\subsection*{3. Computation of Curvature}

The derivative terms are:
\[
\partial_\mu \boldsymbol{\omega}_\nu^{(I)} - \partial_\nu \boldsymbol{\omega}_\mu^{(I)} = (\partial_\mu A_\nu - \partial_\nu A_\mu) \mathbf{v}_{em} = F_{\mu\nu} \mathbf{v}_{em}
\]

The commutator term is:
\[
[\boldsymbol{\omega}_\mu^{(I)}, \boldsymbol{\omega}_\nu^{(I)}] = [A_\mu \mathbf{v}_{em}, A_\nu \mathbf{v}_{em}] = A_\mu A_\nu [\mathbf{v}_{em}, \mathbf{v}_{em}] = 0
\]
since the commutator of any vector with itself vanishes.

Thus:
\[
\mathbf{R}_{\mu\nu}^{(I)} = F_{\mu\nu} \mathbf{v}_{em}
\]

\subsection*{4. Extraction of Vector Equation}

In the main field equation:
\[
\text{Re}\left[\text{Vec}\left(\mathbf{R}_{\mu a} - \frac{1}{2} e_\mu^a R \right)\right] = 0
\]
the relevant part becomes:
\[
\text{Re}\left[\text{Vec}\left(e_a^\mu \mathbf{R}_{\mu\nu}^{(I)} \right)\right] = \text{Re}\left[\text{Vec}\left(e_a^\mu F_{\mu\nu} \mathbf{v}_{em} \right)\right]
\]
which reduces to:
\[
F_{\mu\nu} \text{Re}\left[\text{Vec}(e_a^\mu \mathbf{v}_{em}) \right] = 0
\]

By choosing \( \mathbf{v}_{em} \) and \( e_a^\mu \) such that the scalar product is non-zero, this reproduces:
\[
F_{\mu\nu} = 0
\]
as a constraint. More generally, this validates the emergence of \( F_{\mu\nu} \) as the geometric curvature component associated with \( \omega^{(I)} \).

\subsection*{5. Conclusion}

This simplified calculation demonstrates that the electromagnetic field tensor \( F_{\mu\nu} \) naturally emerges from the imaginary part of the curvature when using the ansatz \( \omega_\mu^{(I)} = A_\mu \mathbf{v}_{em} \). This validates the core mechanism proposed in Appendix 5 and suggests a viable geometric origin for electromagnetism within the unified quaternionic framework.