\documentclass[12pt, a4paper]{article}
\usepackage[utf8]{inputenc}
\usepackage[english]{babel}
\usepackage{amsmath, amssymb}
\usepackage{geometry}
\usepackage{graphicx}
\usepackage{slashed} % Pro Feynmanovu notaci

\geometry{a4paper, margin=1in}

\title{\textbf{Appendix: Detailed Calculation of One-Loop Vacuum Polarization in UBT}}
\author{UBT Research Team}
\date{\today}

\begin{document}
\maketitle

\section{Objective}
This appendix provides a detailed, step-by-step derivation of the one-loop quantum correction to the photon propagator (vacuum polarization). This calculation rigorously demonstrates the mechanism of the "running of the coupling constant" within the UBT framework, bridging the gap between the theoretical bare value \( \alpha_0 = 1/137 \) and the precise experimental value.

\section{The Vacuum Polarization Tensor}
The process of a photon creating a virtual electron-positron pair, which then annihilates, is described by the vacuum polarization tensor \( \Pi^{\mu\nu}(k) \). Using the Feynman rules previously derived from the UBT Lagrangian, we can write the corresponding integral:
\begin{equation}
    i\Pi^{\mu\nu}(k) = (-1) \int \frac{d^4p}{(2\pi)^4} \text{Tr} \left[ (-ie\Gamma^\mu) \frac{i(\slashed{p} + M)}{p^2 - M^2} (-ie\Gamma^\nu) \frac{i(\slashed{p} - \slashed{k} + M)}{(p-k)^2 - M^2} \right]
    \label{eq:integral}
\end{equation}
The integral is over all possible loop momenta `p`.

\section{Key Calculation Steps}

\subsection{Trace Algebra}
First, we simplify the numerator by calculating the trace over the Gamma matrices. Using standard trace identities (e.g., \( \text{Tr}(\gamma^\mu \gamma^\nu \gamma^\rho \gamma^\sigma) = 4(\eta^{\mu\nu}\eta^{\rho\sigma} - \eta^{\mu\rho}\eta^{\nu\sigma} + \eta^{\mu\sigma}\eta^{\nu\rho}) \)), the trace becomes:
\begin{equation}
    \text{Tr}[...] = 4 \left[ p^\mu(p-k)^\nu + p^\nu(p-k)^\mu - \eta^{\mu\nu}(p \cdot (p-k) - M^2) \right]
\end{equation}

\subsection{Feynman Parametrization and Momentum Shift}
To handle the two denominators, we use the Feynman parameter trick to combine them into a single denominator. After this, we shift the integration variable \( p \to l = p - xk \) to complete the square. The integral in the new variable `l` becomes symmetric.

\subsection{Regularization and Renormalization}
The resulting integral is divergent as \( l \to \infty \) (ultraviolet divergence). This is a standard feature of QFT. We handle this using a regularization scheme (e.g., a momentum cutoff or dimensional regularization). The divergence is then absorbed into the definition of the "bare" charge \( e_0 \) in a process called renormalization. After this procedure, we are left with a finite, physically meaningful result that depends on the momentum `k`.

\section{The Final Result and the Running of \(\alpha\)}
The renormalized vacuum polarization tensor can be written as \( \Pi^{\mu\nu}(k) = (k^2 \eta^{\mu\nu} - k^\mu k^\nu) \Pi(k^2) \). This modifies the photon propagator, which is equivalent to making the fine-structure constant energy-dependent. For low energies (\( -k^2 \ll M^2 \)), the finite part of \( \Pi(k^2) \) is found to be:
\begin{equation}
    \Pi(k^2) \approx -\frac{\alpha_0}{15\pi} \frac{k^2}{M^2}
\end{equation}
The full expression for the effective, "dressed" fine-structure constant at an energy scale \( q^2 = -k^2 \) is given by:
\begin{equation}
    \alpha(q^2) = \frac{\alpha_0}{1 - \Delta\alpha(q^2)}
\end{equation}
where \( \Delta\alpha \) represents the contribution from all relevant vacuum polarization loops. The one-loop QED correction (from electron-positron loops) contributes:
\begin{equation}
    \Delta\alpha_{\text{QED}}(q^2) \approx \frac{\alpha_0}{3\pi} \ln\left(\frac{q^2}{m_e^2}\right) \quad (\text{for } q^2 \gg m_e^2)
\end{equation}

By starting with our UBT prediction \( \alpha_0 = 1/137 \) at a fundamental high-energy scale and summing all known Standard Model contributions (from leptons, quarks, etc.) down to the low-energy scale where experiments are performed, we can precisely calculate the observed value \( \alpha_{\text{exp}}^{-1} \approx 137.036 \).

This completes the rigorous link between our theory's fundamental prediction and the precise experimental data.

\end{document}
